\documentclass{article}
\usepackage{amsmath, amssymb, amsfonts, amsthm}
\usepackage[T1]{fontenc}
\usepackage[french]{babel}
\usepackage{xspace}
\usepackage{geometry}[textwidth=10cm]
\usepackage{tikz-cd}
\usepackage{tikz}
\usepackage{fontawesome5}
\usepackage{nicefrac}
\usepackage{cancel}
\usepackage{enumitem}
\usepackage{hyperref}
\usetikzlibrary{positioning}
\usetikzlibrary{decorations.pathmorphing}

\newcommand{\id}{\mathrm{id}}
\newcommand{\op}{\mathrm{op}}
\newcommand{\sep}{\mathrm{sep}}
\newcommand{\inj}{\mathrm{inj}}
\newcommand{\sing}{\mathrm{sing}}
\newcommand{\N}{\mathbb{N}}
\newcommand{\Z}{\mathbb{Z}}
\newcommand{\Q}{\mathbb{Q}}
\newcommand{\Qbar}{\overline{\mathbb{Q}}}
\newcommand{\R}{\mathbb{R}}
\newcommand{\C}{\mathbb{C}}
\newcommand{\K}{\mathbb{K}}
\newcommand{\U}{\mathbb{U}}
\newcommand{\F}{\mathbb{F}}
\renewcommand{\H}{\mathbb{H}}

\newcommand{\prem}{\mathfrak{p}}
\newcommand{\maxid}{\mathfrak{m}}

\renewcommand{\P}{\mathbb{P}}
\renewcommand{\d}{{\rm \, d}}

\newcommand{\M}{\mathcal{M}}
\newcommand{\base}{\mathcal{B}}
\newcommand{\cat}{\mathcal{C}}
\newcommand{\catt}{\mathcal{D}}
\newcommand{\rel}{\mathcal{R}}
\newcommand{\Jcat}{\mathcal{J}}
\newcommand{\Acat}{\mathcal{A}}

\newcommand{\Set}{\mathbf{Set}}
\newcommand{\Top}{\mathbf{Top}}
\newcommand{\Grp}{\mathbf{Grp}}
\newcommand{\Mod}{\mathbf{Mod}}
\newcommand{\Ab}{\mathbf{Ab}}
\newcommand{\Vectcat}{\mathbf{Vect}}
\newcommand{\Rep}{\mathbf{Rep}}
\newcommand{\Ring}{\mathbf{Ring}}

%Operators
\DeclareMathOperator{\Acc}{Acc}
\DeclareMathOperator{\arccosh}{arccosh}
\DeclareMathOperator{\degtr}{degtr}
\DeclareMathOperator{\Card}{Card}
\DeclareMathOperator{\GL}{GL}
\DeclareMathOperator{\SL}{SL}
\DeclareMathOperator{\PSL}{PSL}
\DeclareMathOperator{\PGL}{PGL}
\DeclareMathOperator{\SO}{SO}
\DeclareMathOperator{\pgcd}{pgcd}
\DeclareMathOperator{\Tr}{Tr}
\DeclareMathOperator{\rg}{rg}
\DeclareMathOperator{\Ch}{Ch}
\DeclareMathOperator{\sh}{sh}
\DeclareMathOperator{\Vect}{Vect}
\DeclareMathOperator{\Gal}{Gal}
\DeclareMathOperator{\Frob}{Frob}
\DeclareMathOperator{\rk}{rk}
\DeclareMathOperator{\car}{car}
\DeclareMathOperator{\Fr}{Fr}
\DeclareMathOperator{\Fix}{Fix}
\DeclareMathOperator{\Ob}{Ob}
\DeclareMathOperator{\Aut}{Aut}
\DeclareMathOperator{\Int}{Int}
\DeclareMathOperator{\Mor}{Mor}
\DeclareMathOperator{\Hom}{Hom}
\DeclareMathOperator{\Fun}{Fun}
\DeclareMathOperator{\Nat}{Nat}
\DeclareMathOperator{\colim}{colim}
\DeclareMathOperator{\cod}{cod}
\DeclareMathOperator{\End}{End}
\DeclareMathOperator{\coker}{coker}
\DeclareMathOperator{\im}{im}
\DeclareMathOperator{\coim}{coim}
\DeclareMathOperator{\Frac}{Frac}
\renewcommand{\th}{\mathop{\rm th}}
\renewcommand{\Re}{\mathop{\rm Re}}
\renewcommand{\Im}{\mathop{\rm Im}}

\newcommand{\todo}{\textbf{TODO}}

\renewcommand{\epsilon}{\varepsilon}
\newcommand{\invlim}{\varprojlim}
\newcommand{\dirlim}{\varinjlim}

\newcommand{\esh}{{\textstyle \int}}
\newcommand{\midbar}{\, \middle|\,}

%Macros
\newcommand{\applic}[4]{\begin{array}[t]{rcl}
#1 & \to & #2 \\
#3 & \mapsto & #4
\end{array}}

\newcommand{\norm}[1]{\left\lVert #1 \right\rVert}

\newcommand{\warning}{\faExclamationTriangle \hspace{3pt}}

\renewcommand{\subset}{\subseteq}

\setlength{\parindent}{0pt}

\theoremstyle{plain}
\newtheorem{theorem}{Théorème}[subsection]
% TODO : améliorer la numérotation des machins
\newtheorem{proposition}[theorem]{Proposition}
\newtheorem{lemma}[theorem]{Lemme}
\newtheorem{corollary}[theorem]{Corollaire}

\theoremstyle{definition}
\newtheorem{definition}[theorem]{Définition}
\newtheorem{example}[theorem]{Exemple}

\theoremstyle{remark}
\newtheorem*{remark}{Remarque}
\newtheorem*{exercise}{Exercice}

\title{Algèbre 2}
\author{Cours de Vlerë Mehmeti\footnote{Notes originales trouvables à \url{https://webusers.imj-prg.fr/~vlere.mehmeti/teaching.html}}}
\date{}

\setlist[enumerate]{label=(\arabic*)}

\begin{document}

\maketitle

\setcounter{section}{-1}
\section{Rappels sur les anneaux}

\subsection{Anneaux}

\begin{definition}
    Un \emph{anneau} $(A,+,\cdot)$ est un ensemble muni de deux lois de composition internes $+$ et $\cdot$ telles que
    \begin{enumerate}
        \item $(A,+)$ est un groupe abélien (on note son élément neutre $0_A$)
        \item $\cdot$ est associative
        \item $\cdot$ se distribue sur $+$.
    \end{enumerate}
    S'il existe un élément neutre pour $\cdot$, on le note $1_A$ et on dit que $A$ est \emph{unitaire}. Si $\cdot$ est commutatif, on dit que $A$ est \emph{commutatif}. 
\end{definition}

\begin{remark}
    Dans ce cours, sauf mention explicite, tout anneau considéré sera unitaire et commutatif.
\end{remark}

\begin{definition}
    Un élément $a \in A \setminus\{0\}$ est dit \emph{un diviseur de zéro} s'il existe $b \in A \setminus \{0\}$ tel que $ab = 0$. Si $A$ ne possède pas de diviseurs de zéros, on dit que $A$ est \emph{intègre}. Un élément $a \in A$ est dit \emph{inversible} s'il existe $b \in A$ tel que $ab = ba = 1$. On notera l'ensemble des éléments inversibles $A^\times$.
\end{definition}

\begin{remark} \leavevmode
    \begin{enumerate}
        \item Si $a \in A^\times$ est inversible, l'élément $b \in A$ tel que $ab = 1$ est unique\footnote{En effet, si $ab = b'a = 1$, alors $b' = b'ab = b$}. On le note $a^{-1}$.
        \item Un élément inversible n'est pas un diviseur de zéro.
    \end{enumerate}
\end{remark}

Dans un anneau intègre $A$, pour tous $a,b \in A$, $ab = ac$ implique $a = 0$ ou $b = c$. En particulier, si $a \ne 0$, on peut simplifier par $a$.

\begin{example}
    \begin{enumerate}
        \item $(\Z,+,\cdot)$ est un anneau.
        \item $(\Z[\sqrt{d}],+,\cdot)$ est un anneau.
        \item $(\Z/n\Z,+_n,\cdot_n)$ l'est aussi.
        \item $(\M_n(\R),+,\cdot)$ est un anneau non commutatif.
        \item L'anneau des quaternions $(\H_\C,+,\cdot)$ est aussi non commutatif.
    \end{enumerate}
    Tous ces exemples sont des anneaux intègres.
    \begin{enumerate}
        \setcounter{enumi}{5}
        \item Si $A,B$ sont des anneaux, alors $A\times B$ muni des lois produits (composante par composante) est un anneau en général non intègre car $(1_A,0) \cdot (0,1_B) = (0,0)$. On peut généraliser à un produit quelconque d'anneaux.
        \item Soit $A$ un anneau. Alors $(A[X],+,\cdot)$ est un anneau où $+$,$\cdot$ sont l'addition et la multiplication usuelle des polynômes à coefficients dans $A$. On peut définir récursivement, pour $n \in \N^*$, $A[X_1,\dots,X_n] = A[X_1,\dots,X_{n-1}][X_n]$, et les lois internes $+$ et $\cdot$ induites de celles de $A[X_1,\dots,X_{n-1}]$. Par convention, pour $n = 0, A[X_1,\dots,X_n] = A$.
        \item Soit $A$ un anneau et $X$ un ensemble. Alors $A^X = \{f \mid f : X \to A\}$ muni des lois données par
        \[\begin{cases}
            (f+g)(x) & = f(x) +_A g(x) \\
            (f\cdot g)(x) & = f(x) \cdot_A g(x)
        \end{cases}\]
        est un anneau.
        \item L'ensemble $\{f : \R \to \R \mid \exists [a_f,b_f] \subseteq \R, f_{\mid [a_f,b_f]^c} = 0\}$ est un anneau.
        \item L'anneau nul $\{0\}$ est l'unique anneau où $0 = 1$.
    \end{enumerate}
\end{example}

\begin{proposition}
    Soit $A$ un anneau. Alors
    \begin{enumerate}
        \item $a 0 = 0 a = 0$ pour tout $a \in A$
        \item $(-a)b = a(-b) = -(ab)$ pour tous $a,b\in A$
        \item $(-a)(-b) = ab$ pour tous $a,b \in A$
        \item $-a = (-1)a$ pour tout $a \in A$.
    \end{enumerate}
\end{proposition}

\begin{definition}
    Un \emph{sous-anneau} $R$ d'un anneau $(A,+,\cdot)$ est un sous-groupe $R$ de $(A,+)$ tel que pour tous $a,b \in R$, $ab \in R$.
\end{definition}

\begin{remark}
    $R$ est un sous-anneau si et seulement si $0 \in R$ (ou $R \ne \varnothing$) et pour tous $a,b \in R$, $a-b \in R$ et $ab \in R$.
\end{remark}

\begin{remark}
    L'anneau nul n'est pas intègre.
\end{remark}

\begin{example} \leavevmode
    \begin{enumerate}
        \item $n\Z$ est un sous-anneau de $\Z$ pour tout $n \in \Z$.
        \item $(\Q,+,\cdot)$ est un anneau et $\{\frac{m}{n} \mid m,n \in \Z \setminus\{0\}, m \wedge n = 1, n\text{ impair}\} \cup \{0\}$ est un sous-anneau. Si on remplace \og $n$ impair \fg par \og $m$ impair \fg, ce n'est plus le cas.
    \end{enumerate}
\end{example}

\begin{definition}
    Un \emph{morphisme d'anneaux unitaires} est une application $\varphi : A \to B$ telle que
    \begin{enumerate}
        \item $\varphi(a+b) = \varphi(a)+\varphi(b)$
        \item $\varphi(ab) = \varphi(a)\varphi(b)$
        \item $\varphi(1_A) = 1_B$
    \end{enumerate}
    Le morphisme $\varphi$ est dit un \emph{isomorphisme} s'il est bijectif. \\
    Si $\varphi$ est un isomorphisme et $A=B$, on dit que $\varphi$ est un \emph{automorphisme de $A$}.
\end{definition}

\begin{remark}
    On notera par $\Aut(A)$ l'ensemble des automorphismes de $A$. Muni de la composition, c'est un groupe.
\end{remark}

\begin{proposition}
    Soit $\varphi : A \to B$ un morphisme d'anneaux. Alors
    \begin{enumerate}
        \item $\Im(\varphi)$ est un sous-anneau de $B$.
        \item $\varphi$ est injectif si et seulement si
        \[\ker (\varphi) = \{a \in A \mid \varphi(a) = 0_B\} = \{0_A\}\]
    \end{enumerate}
\end{proposition}

\begin{proposition}
    Si $\varphi : A \to B$ et $\psi : B \to C$ sont deux morphismes d'anneaux, alors $\psi \circ \varphi : A \to C$ en est un aussi.
\end{proposition}

\begin{example} \leavevmode
    \begin{enumerate}
        \item $\pi : \applic{\Z}{\Z/n\Z}{m}{m \mod n}$ est un morphisme d'anneaux (surjectif, mais pas injectif).
        \item $\psi_n : \applic{\Z}{\Z}{x}{nx}$ n'est pas un morphisme d'anneaux unitaires si $n \ne 1$.
        \item $\theta : \applic{\C}{\C}{z}{\overline{z}}$ est un automorphisme de $\C$.
        \item $\applic{\Q[X]}{\Q}{P(X)}{P(0)}$ est un morphisme d'anneaux (surjectif, non injectif).
        \item $\applic{\Z\times\Z}{\Z\times\Z}{(a,b)}{(a,0)}$ n'est pas un morphisme d'anneaux unitaires.
    \end{enumerate}
\end{example}

\begin{definition}
    Un \emph{corps} est un anneau commutatif unitaire non nul où tout élément non nul est inversible.
\end{definition}

\begin{example} \leavevmode
    \begin{enumerate}
        \item $\Q,\R,\C$ sont des corps (de nombres).
        \item Le corps des fractions rationnelles $\Q(T) = \{\frac{P}{Q} \mid P,Q \in \Q[T]\}$ est un corps.
        \item Le corps des nombres algébriques $\overline{Q} = \{x \in \C \mid \exists P \in \Q[X] \setminus\{0\}, P(x) = 0\}$ est un corps.
    \end{enumerate}
\end{example}

\begin{definition}
    Soit $A$ un anneau sans diviseur de zéro. Le \emph{corps des fractions de $A$}, noté $\Frac A$, est
    \[\left\{\frac{a}{b} \midbar a \in A, b \in A\setminus\{0\}\right\}/\sim \quad \text{où} \quad \frac{a}{b} \sim \frac{c}{d} \iff ad-bc = 0\]
\end{definition}

\begin{proposition}
    Les lois suivantes sont bien définies sur $\Frac A$ et en font un corps :
    \begin{align*}
        \frac{a}{b} + \frac{c}{d} & = \frac{ad+bc}{bd} \\
        \frac{a}{b} \frac{c}{d} & = \frac{ac}{bd}
    \end{align*}
\end{proposition}

\begin{proposition}
    L'application $\varphi : \applic{A}{\Frac A}{a}{\frac{a}{1}}$ est un morphisme d'anneaux injectif. Si $A$ est un corps, c'est un isomorphisme.
\end{proposition}

\begin{remark}
    $\Frac A$ est le plus petit corps contenant $A$.
\end{remark}

\begin{example} \leavevmode
    \begin{enumerate}
        \item $\Frac \Z = \Q$
        \item Pour $n\in \N^*$, $\Z/n\Z$ a des diviseurs de zéro, sauf si $n$ est premier et dans ce cas c'est un corps.
        \item $\Frac \Q[X] = \Q(X)$. Plus généralement, on définit $K(X_1,\dots,X_n) = \Frac K[X_1,\dots,X_n]$ pour $K$ un corps.
    \end{enumerate}
\end{example}

\subsection{Idéaux}

\begin{definition}
    Soit $A$ un anneau (commutatif, unitaire). Un idéal de $A$ est un sous-ensemble $I \subseteq A$ tel que
    \begin{enumerate}
        \item $(I,+)$ est un sous-groupe de $(A,+)$
        \item Pour tous $a \in A$ et $b \in I$, $ab \in I$.
    \end{enumerate}
    On note $I \lhd A$.
\end{definition}

\begin{remark}
    Comme $(A,+)$ est un groupe abélien et $I$ en est un sous-groupe, il est distingué, donc $A/I$ est un groupe abélien muni de l'addition $(a+I) + (b+I) = (a+b) + I$.
\end{remark}

\begin{proposition}
    Soit $I$ un idéal de l'anneau $A$. Alors $(A/I,+,\cdot)$ est un anneau de loi multiplicative $(a+I) (b+I) = ab + I$. En particulier, cette opération est bien définie. On appelle $A/I$ \emph{l'anneau quotient de $A$ par $I$}. De plus, la projection canonique $\pi : \applic{A}{A/I}{x}{x+I}$ est un morphisme d'anneaux surjectif.
\end{proposition}

\begin{remark}
    La définition d'idéal est précisément faite pour que la proposition ci-dessus fonctionne.
\end{remark}

\begin{theorem}[Premier théorème d'isomorphisme] \leavevmode
    \begin{enumerate}
        \item Si $\varphi : A \to B$ est un morphisme d'anneaux, alors $\ker \varphi$ est un idéal de $A$ et $A/\ker \varphi \simeq \im \varphi$.
        \item Si $I$ est un idéal de $A$ alors $\pi : A \to A/I$ est surjective de noyau $I$.
    \end{enumerate}
\end{theorem}

\begin{remark}
    Les autres théorèmes d'isomorphisme fonctionnent aussi, par exemple si $I \subset J$ sont deux idéaux de $A$, alors $A/J \simeq (A/I)(J/I)$.
\end{remark}

\begin{theorem}[Propriété universelle du quotient]
    Soit $\varphi : A \to B$ un morphisme d'anneaux. Soit $I$ un idéal de $A$ tel que $\varphi(I) = \{0_B\}$. Alors il existe un unique $\overline{\varphi} : A/I \to B$ tel que $\varphi = \overline{\varphi} \circ \pi$ avec $\pi : A \to A/I$ la projection canonique.
    \begin{center}
        \begin{tikzcd}[column sep=tiny]
            A \arrow[rr, "\varphi"] \arrow[rd,"\pi" '] & & B \\
            & A/I \arrow[ru, dashed, "\exists ! \overline{\varphi}" ']
        \end{tikzcd}
    \end{center}
\end{theorem}

\begin{example} \leavevmode
    \begin{enumerate}
        \item $n\Z$ est un idéal de $\Z$ pour $n \in \N$
        \item $\{0\}$ et $A$ sont des idéaux de $A$
        \item $\{X P(X) \mid P \in \Q[X]\}$ est un idéal de $Q[X]$
    \end{enumerate}
\end{example}

\begin{proposition}
    Soit $I$ un idéal de $A$. L'application
    \[\applic{\{\text{idéaux de $A$ contenant $I$}\}}{\{\text{idéaux de $A/I$}\}}{J}{J/I}\]
    est une bijection.
\end{proposition}

\begin{proof}
    \todo
\end{proof}

\begin{example}
    Les idéaux de $\Z/n\Z$ sont $m\Z/n\Z$ tel que $m\Z \supseteq n\Z$ c'est-à-dire tels que $m \mid n$.
\end{example}

\begin{proposition}
    Un anneau $A$ est un corps si et seulement si ses seuls idéaux sont $\{0\}$ et $A$.
\end{proposition}

\begin{proof}
    Si $A$ est un corps et $J \subseteq A$ un idéal non nul, alors $J$ contient un élément $x \in A \setminus\{0\}$. Ainsi pour tout $a \in A$, $a = ax^{-1}x \in J$, donc $J=A$. Réciproquement, si les seuls idéaux de $A$ sont $\{0\}$ et $A$, pour $x\in A\setminus\{0\}$, l'ensemble $\{ax \mid a \in A\}$ est un idéal non nul de $A$, donc égal à $A$, d'où il existe $a \in A$ tel que $ax = 1$.
\end{proof}

\begin{corollary}
    Soit $A \ne 0$ un anneau et $K$ un corps. Tout morphisme d'anneaux $\varphi : K \to A$ est injectif.
\end{corollary}

\begin{proof}
    Le noyau $\ker \varphi$ est un idéal de $K$ et $1_K \notin \ker \varphi$ car $\varphi(1_K) = 1_A$ donc $\ker \varphi = \{0\}$.
\end{proof}

\begin{proposition}
    Une intersection d'idéaux est un idéal.
\end{proposition}

\begin{definition}
    Soit $A$ un anneau et $S \subseteq A$. \emph{L'idéal engendré par $S$} est le plus petit idéal de $A$ contenant $S$. On le note $(S)$.
\end{definition}

\begin{remark} \leavevmode
    \begin{enumerate}
        \item $(S) = \bigcap\limits_{\begin{subarray}{c} I \lhd A \\ S \subseteq I \end{subarray}} I$
        \item $(S) = \left\{\sum\limits_{\text{finie}} as \midbar a \in A, s \in S\right\}$
        \item Si $S = \{a\}$, alors $(a) = \{ax \mid x\in A\}$ et $(a) = A$ si et seulement si $a \in A^\times$ et $(a) = \{0\}$ si et seulement si $a = 0$.
    \end{enumerate}
\end{remark}

\begin{definition}
    Un idéal $I$ de $A$ est dit \emph{principal} s'il existe $a \in A$ tel que $I = (a)$. Si $A$ est intègre et que tout idéal de $A$ est principal, $A$ est dit \emph{principal}. L'idéal $I$ est dit de \emph{type fini} s'il existe $S \subseteq A$ finie telle que $I = (S)$.
\end{definition}

\begin{remark}
    Soit $A$ intègre et $a,b \in A \setminus \{0\}$. Alors
    \[(a) = (b) \iff \exists x \in A^\times, a=bx\]
\end{remark}

\begin{example} \leavevmode
    \begin{enumerate}
        \item L'idéal $(X,Y^2 +1)$ de $\Q[X,Y]$ est celui des polynômes de la forme $P(X,Y)X + Q(X,Y)(Y^2+1)$ avec $P,Q \in \Q[X,Y]$.
        \item On a $(X,X-1) = \Q[X]$ car $1 = X - (X-1)$.
    \end{enumerate}
\end{example}

\begin{remark}
    Il est en général difficile de déterminer si un idéal est de type fini et de trouver un ensemble générateur. Un ensemble générateur d'un idéal n'est pas unique, par exemple $(1) = (X,X-1)$ ou $(X-Y,Y) = (X,Y) = (X,Y,X+Y)$.
\end{remark}

\subsection{Idéaux premiers et maximaux}
\begin{definition} \leavevmode
    \begin{enumerate}
        \item Un idéal $\prem$ de $A$ est dit \emph{premier} si $\prem \ne A$ et pour tous $a,b \in A$, $ab \in \prem$ implique $a \in \prem$ ou $b \in \prem$.
        \item Un idéal $\maxid$ de $A$ est dit \emph{maximal} si $\maxid \ne A$ et que les seuls idéaux de $A$ qui le contiennent sont $\maxid$ et $A$.
    \end{enumerate}
\end{definition}

\begin{theorem} \leavevmode
    \begin{enumerate}
        \item Un idéal $\prem$ de $A$ est premier si et seulement si $A/\prem$ est intègre.
        \item Un idéal $\maxid$ de $A$ est maximal si et seulement si $A/\maxid$ est un corps.
    \end{enumerate}
    En particulier, tout idéal maximal est premier.
\end{theorem}

\begin{proof}
    Soient $a,b \in A$. Alors, $(a+\prem)(b+\prem) = \prem$ si et seulement si $ab \in \prem$ si et seulement si $a \in \prem$ ou $b \in \prem$ si et seulement si $(a+\prem) = \prem$ ou $b +\prem = \prem$. De cela découle $(1)$.\\
    On sait qu'on a une bijection entre les idéaux de $A$ qui contiennent $\maxid$ et les idéaux de $A/\maxid$. De cette bijection et du fait que $A/\maxid$ est un corps si et seulement si ses seuls idéaux sont $\{0\}$ et lui-même, on déduit $(2)$.
\end{proof}

\begin{example} \leavevmode
    \begin{enumerate}
        \item $n\Z$ est un idéal premier (ou maximal) de $\Z$ si et seulement si $n$ est premier.
        \item $(P) \subseteq \Q[X_1,\dots,X_n]$ est premier si et seulement si $P$ nul ou irréductible c'est-à-dire non constant et que si $P=AB$, alors $A$ ou $B$ est constant.
        \item Si $K$ est un corps, $(X)$ est un idéal maximal de $K[X]$ car $K[X]/(X) \simeq K$ (premier théorème d'isomorphisme appliqué au morphisme d'évaluation en $0_K$).
        \item Pour $\alpha_1,\dots,\alpha_n \in K$, l'idéal $(X_1-\alpha_1,\dots,X_n-\alpha_n) \subseteq K[X_1,\dots,X_n]$ est maximal.
    \end{enumerate}
\end{example}

\begin{proposition}
    Soit $A$ un anneau commutatif unitaire. Tout idéal propre de $A$ est inclus dans un idéal maximal.
\end{proposition}

\begin{proof}
    Soit $I$ un idéal de $A$ distinct de $A$. On considère $\Jcat = \{J \lhd $A$ \mid I \subseteq J, J \ne A\}$. Par propreté de $I$, cet ensemble est non vide. Il est ordonné pour l'inclusion. Soit $\Acat \subset \Jcat$ une partie de $\Jcat$ totalement ordonnée pour l'inclusion. On pose $\bigcup\limits_{J \in \Acat} J$. C'est un idéal car $\Acat$ est totalement ordonnée pour l'inclusion. De plus, pour tout $J \in \Acat$, $1 \notin J$, donc $1 \notin \bigcup\limits_{J \in \Acat} J$. Ainsi, $\bigcup\limits_{J \in \Acat} J$ est une borne supérieure de $\Acat$ dans $\Jcat$. Cela démontre que $\Jcat$ est un ensemble inductif. Ainsi, le lemme de Zorn donne l'existence d'un élément maximal $\maxid \in \Jcat$. Par l'absurde, si $\maxid$ n'est pas un idéal maximal, on dispose de $J \ne A$ idéal qui contient strictement $\maxid$. Cela contredit la maximalité de $\maxid$ dans $(\Jcat,\subseteq)$.
\end{proof}

\begin{example} \leavevmode
    \begin{enumerate}
        \item $(X)$ est un idéal premier mais non maximal de $\Z[X]$.
        \item $(X,2)$ est un idéal maximal de $\Z[X]$.
        \item $(\Q,+,\cdot_0)$ où $\cdot_0$ est définie par $a\cdot_0 b = 0$ pour tous $a,b \in \Q$ n'a pas d'idéaux maximaux. C'est un anneau non unitaire.
    \end{enumerate}
\end{example}

\subsection{Anneaux principaux}

Remarquons que $\Z$ est un exemple d'anneau principal et $\Z[X]$ un exemple d'anneau non-principal, car $(X,2) \lhd \Z[X]$ n'est pas principal.

\begin{proposition}
    Soit $A$ un anneau principal. Alors, un idéal propre (distinct de $A$) et non-trivial (distinct de $\{0\}$) est premier si et seulement s'il est maximal.
\end{proposition}

\begin{proof}
    Un idéal maximal est toujours premier. Soit $(a)$ avec $a \in A$ un idéal premier. On a $(a) \ne A$. Soit $(b) \supseteq (a)$ un autre idéal le contenant. Alors, $a = bx$ avec $x \in A$. Par primalité de $(a)$, ou bien $b \in (a)$, dans ce cas $(a) = (b)$, ou bien $x \in (a)$, dans ce cas $x = ay$ avec $y \in A$, donc $a = aby$, donc $by = 1$ par intégrité, donc $(b) = A$. Ainsi $(a)$ est maximal.
\end{proof}

\begin{definition}
    Pour $a,b \in A$, on dit que \emph{$a$ divise $b$} et on écrit $a \mid b$ s'il existe $x\in A$ tel que $b = ax$.
\end{definition}

\begin{remark}
    Pour $a,b \in A$, $a \mid b$ équivaut à $(b) \subseteq (a)$. Si $x \in A^\times$, alors $x \mid a$ pour tout $a \in A$.
\end{remark}

Brièvement, quelques rappels sur $K[X]$.

\begin{theorem}
    Soit $K$ un corps. Alors $K[X]$ est un anneau principal.
\end{theorem}

\begin{theorem}
    Soit $K$ un corps. Pour tout polynôme $P \in K[X] \setminus \{0\}$, il existe une unique décomposition $P = \alpha \prod\limits_{i=1}^d P_i^{n_i}$ (unique à permutation près des $P_i$) avec $\alpha \in K^\times$, $P_i$ unitaires et irréductibles et deux à deux distincts.
\end{theorem}

\begin{lemma}
    Un idéal $(P)$ est premier si et seulement si $P = 0$ ou $P$ est irréductible.
\end{lemma}

Quelques rappels :

\begin{enumerate}
    \item L'algorithme de division euclidienne fonctionne dans $K[X]$.
    \item Si $P,Q \in K[X]$ alors leur PGCD est bien défini (à multiplication par un élément de $K^\times$ près) et $(P,Q) = (\pgcd(P,Q))$.
    \item Si $K \subseteq L$ avec $L$ un corps, pour $P,Q \in K[X]$, on a $\pgcd_K(P,Q) = \pgcd_L (P,Q)$.
\end{enumerate}

Référence : Abstract Algebra de Dummit and Foote.

\section{Théorie des corps}

\subsection{Extensions de corps}

\begin{definition}
    Soit $K$ un corps. Une \emph{$K$-algèbre} est un anneau $A$ muni d'un morphisme d'anneaux $i : K \to A$ (forcément injectif car $K$ est un corps). Si $K \xrightarrow{i} A$ et $K \xrightarrow{j} B$ sont deux $K$-algèbres, un \emph{morphisme de $K$-algèbres} (parfois appelé un \emph{$K$-morphisme}) est un morphisme d'anneaux $\varphi : A \to B$ tel que $\varphi \circ i = j$.
\end{definition}

\begin{definition}
    Soit $K$ un corps. Une \emph{extension de corps de $K$} est un corps $L$ muni d'un morphisme de corps $K \xrightarrow{\theta} L$. On écrit $L/K$. Une \emph{sous-extension de $L/K$} est un corps $E$ muni de morphismes de corps $K \xrightarrow{i} E \xrightarrow{j} L$ tels que $j \circ i = \theta$. On écrit $L/E/K$ et on parle de \emph{tour de corps}.
\end{definition}

\begin{remark}
    Comme le morphisme $\theta$ dans la définition précédente est injectif, on fait régulièrement l'abus de notation d'identifier $K$ à $\theta(K)$ pour écrire $K \subseteq L$.
\end{remark}

\begin{example} \leavevmode
    \begin{enumerate}
        \item $\R/\Q$, $\C/\Q$, $\C/\R$ sont des extensions de corps.
        \item $\applic{\Q(i)}{\C}{i}{i}$ et $\applic{\Q(i)}{\C}{i}{-i}$ sont deux $\Q$-morphismes (où $\Q(i) = \{a+bi \mid a,b \in \Q\}$).
        \item $\applic{\Q(\sqrt[3]{2})}{\C}{\sqrt[3]{2}}{\sqrt[3]{2}}$ et $\applic{\Q(\sqrt[3]{2})}{\C}{\sqrt[3]{2}}{j\sqrt[3]{2}}$  sont deux $\Q$-morphismes, où $j^3 = 1$ et $j \ne 1$ et $\Q(\sqrt[3]{2}) = \{a + b \sqrt[3]{2} + c \sqrt[3]{4} \mid a,b,c \in \Q\}$.
        \item Pour $\F_p = \Z/p\Z$ avec $p$ premier, l'application $\Frob : \applic{\F_p(T)}{\F_p(T)}{x}{x^p}$ est un morphisme de corps (donc est injective) mais n'est pas surjective. Le fait que $\Frob(1) = 1$ et $\Frob(xy) = \Frob(x)\Frob(y)$ est clair. De plus,
        \[(x+y)^p = \sum_{k=0}^p \binom{p}{k} x^k y^{p-k}\]
        Et par primalité de $p$ et comme $p = 0$ dans $\Z/p\Z$, pour $0 < k < p$, $\binom{p}{k} = 0$. Ainsi, $(x+y)^p = x^p + y^p$.
        \item Soit $f : \R \to \R$ un morphisme de corps. Comme $f(1) = 1$, on a $f(n) = n$ pour tout $n \in \Z$, puis $f(r) = r$ pour tout $r \in \Q$, donc $f$ stabilise $\Q$. Pour $x \ge 0$, $x = \sqrt{x}^2$, donc $f(x) = f(\sqrt{x})^2 \ge 0$, donc $f$ préserve $\R^+$. Ainsi, si $x \ge y$, $x-y \ge 0$, donc $f(x - y) \ge 0$, donc $f(x) \ge f(y)$, donc $f$ est croissante. Soit $(x_n)$ une suite réelle telle que $x_n \to x \in \R$. Alors, $x-x_n \to 0$, donc on peut trouver deux suites de rationnels $(r_n), (s_n)$ de limites nulles telles que pour tout $n$, $r_n \le x-x_n \le s_n$. Par croissance de $f$ et comme $f$ préserve $\Q$, $r_n \le f(x) - f(x_n) \le s_n$. Ainsi, $f(x_n) \to f(x)$, ce qui démontre que $f$ est continue. Par densité de $\Q$ dans $\R$ et comme $f_{\mid \Q} = \id$, on en déduit que $f = \id$. Ainsi, le seul automorphisme de corps de $\R$ est l'identité.
    \end{enumerate}
\end{example}

\begin{definition}
    Soit $I$ un ensemble et $S = (X_i)_{i\in I}$ une famille d'indéterminées indexée par $I$. On définit :
    \[K[X_i \mid i \in I] = \{P(X_{i_n}, \dots,X_{i_m}) \mid \exists \{i_1,\dots,i_n\} \subseteq I, P \in K[X_{i_1},\dots,X_{i_n}]\}\]
\end{definition}

Remarquons que  $K[X_i \mid i\in I] = \bigcup\limits_{E \subseteq I \text{ fini}} K[X_i \mid i \in E]$, ce qui permet de définir les lois internes $+$ et $\cdot$ sur $K[X_i \mid i\in I]$. On construit ainsi une structure de $K$-algèbre sur $K[X_i \mid i\in I]$. De plus, $K[X_i \mid i\in I]$ est intègre.

\begin{definition}
    On définit $K(X_i \mid i \in I) = \Frac K[X_i \mid i\in I]$. C'est une extension de corps de $K$.
\end{definition}

\begin{example}
    Soit $I = \N$. Le $K$-morphisme $\applic{K(X_i \mid i \in I)}{X_i}{K(X_i \mid i \in I)}{X_{i+1}}$ est un morphisme de corps non surjectif.
\end{example}

Si $I$ est fini, on écrit $I = \{1,2,\dots,n\}$ et on a $K[X_i \mid i\in I] = K[X_1,\dots,X_n]$ et $K(X_i \mid i \in I) = K(X_1,\dots,X_n)$.

\begin{proposition}
    Si $L/K$ est une extension de corps, alors $L$ est muni d'une structure de $K$-espace vectoriel par la loi externe $\applic{K\times L}{L}{(k,v)}{k \cdot_L v}$ (ou si on ne fait pas l'identification, $(k,v) \mapsto i(k)\cdot_L v$).
\end{proposition}

\begin{definition}
    Le \emph{degré} d'une extension de corps $L/K$ est la dimension de $L$ vu comme $K$-espace vectoriel. On le note $[L : K]$. Si $[L : K] < \infty$, on dira que $L/K$ est une \emph{extension finie}. Si $[L : K] = \infty$, elle est dite \emph{infinie}.
\end{definition}

\begin{remark}
    Le résultat bien connu d'algèbre linéaire qui dit qu'en dimension finie, toutes les bases ont même cardinal est vrai en général (deux bases dans un espace vectoriel sont toujours en bijection). Ainsi, on peut parler du degré d'une extension infinie de manière plus précise en parlant du cardinal des bases.
\end{remark}

\begin{example} \leavevmode
    \begin{enumerate}
        \item On a $[\C : \R] = 2 = [\Q(i) : \Q]$ et $[\Q(\sqrt[3]{2}) : \Q] = 3$.
        \item $[\R : \Q]$ est non dénombrable sinon $\R$ serait dénombrable.
        \item $[K(X): K] = \max (\Card(K),\Card(\N))$. Une base est donnée par
        \[\left\{X^j, \frac{X^i}{P^n} \midbar j \in \N, i,n \in \N^*, i < \deg P, P \text{ unitaire irréductible dans } K[X]\right\}\]
    \end{enumerate}
\end{example}

\begin{theorem} \label{thmdeg}
    Soit $M/L/K$ une tour de corps. Alors
    \[[M : K] = [M : L] [L : K]\]
    En particulier, $M/K$ est finie si et seulement si $M/L$ et $L/K$ le sont.
\end{theorem}

\begin{remark}
    La relation est à lire : si $\base_1$ est une base de $M/L$ et $\base_2$ une base de $L/K$, alors $\dim_K M = \Card(\base_1 \times \base_2)$.
\end{remark}

\begin{proof}
    Soit $(e_i)_{i\in I}$ une base du $L$-espace vectoriel $M$ et $(f_j)_{j \in J}$ une base du $K$-espace vectoriel $L$. On montre que $(e_i f_i)_{(i,j) \in I \times J}$ est une base de $M/K$. \\
    Cette famille est libre : si $\sum\limits_{i,j} \lambda_{ij} e_i f_j = 0$ avec $\lambda_{ij} \in K$ presque nulle, alors $\sum\limits_{i \in I} e_i \left(\sum\limits_{j \in J} \lambda_{ij} f_j\right) = 0$. Comme $(e_i)$ est libre sur $L$, on a pour tout $i \in I$, $\sum\limits_{j \in J} \lambda_{ij} f_j = 0$. Comme $(f_j)$ est libre sur $K$, les $\lambda_{ij}$ sont nuls, d'où la liberté. Cette famille est génératrice : pour $x \in M$,
    \[x = \sum_{j\in J} \lambda_j f_j = \sum_{j \in J} \sum_{i \in I} \mu_i e_i f_j\]
    Où les $\lambda_j \in L$ existent car $(f_j)$ engendre $M$ et les $\mu_i \in K$ existent car $(e_i)$ engendre $L$.
\end{proof}

\begin{definition}
    Soit $L/K$ une extension de corps et $S \subseteq L$.
    \begin{enumerate}
        \item Le \emph{sous-anneau de $L/K$ engendré par $S$}, noté $K[S]$, est défini par
        \[K[S] = \bigcap\limits_{\begin{subarray}{c} K \cup S \subseteq A \subseteq L \\ A \text{ sous-anneau} \\ \text{de } L\end{subarray}} A\]
        C'est le plus petit sous-anneau de $L$ qui contient $K$ et $S$.
        \item La \emph{sous-extension $L/K$ engendrée par $S$}, notée $K(S)$ est définie par
        \[K(S) = \bigcap\limits_{\begin{subarray}{c} K \cup S \subseteq E \subseteq L \\ E \text{ sous-corps} \\ \text{de } L\end{subarray}} E\]
        C'est le plus petit sous-corps de $L$ qui contient $K$ et $S$.
        \item On dira que $L/K$ est \emph{de type fini} s'il existe $S \subseteq L$ fini tel que $L = K(S)$.
    \end{enumerate}
\end{definition}

\begin{remark} \leavevmode
    \begin{enumerate}
        \item On a $K[S] = \Vect_K (\base)$ avec $\base = \left\{\prod\limits_{\text{fini}} \alpha \midbar \alpha \in S\right\}$.
        \item On a $K(S) = \Frac K[S]$. Si $S = \{a_1,\dots,a_n\} \subseteq L$, on écrira $K[a_1,\dots,a_n]$ et $K(a_1,\dots,a_n)$ pour $K[S]$ et $K(S)$, respectivement. \\
        On a alors $K[a_1,\dots,a_n] = \{P(a_1,\dots,a_n) \mid P \in K[X_1,\dots,X_n]\}$.
        \item $K(S) = \bigcup\limits_{\begin{subarray}{c} A \subseteq S \\ \text{finie}\end{subarray}} K(A)$ car $\bigcup\limits_{A \text{ finie}} K(A)$ est un corps.
        \item Toute extension de corps finie est de type fini (car une base est un ensemble générateur).
    \end{enumerate}
\end{remark}

\begin{example} \leavevmode
    \begin{enumerate}
        \item $\Q(\mu_n)/\Q$ avec $\mu_n \in \C$ une racine primitive $n$-ième de l'unité où $n \in \N^*$ est une extension finie. On remarque que $\Q(\mu_n) = \Q[\mu_n]$ car $\mu_n^n = 1$. On dit que $\Q(\mu_n)/\Q$ est une \emph{extension cyclotomique}.
        \item $\Q(\sqrt[3]{2},j)/\Q$ est finie.
        \item $K(X)/K$ est de type fini, mais pas finie.
        \item $K(X_n \mid n \in \N)/K$ n'est pas de type fini. $\R/\Q$, $\C/\Q$ ne le sont pas non plus. On verra cela en utilisant des bases de transcendance.
    \end{enumerate}
\end{example}

\subsection{Extensions algébriques et transcendantes}

\begin{definition}
    Soit $L/K$ une extension de corps. Un élément $\alpha \in L$ est dit \emph{algébrique sur $K$} s'il existe $P \in K[X] \setminus \{0\}$ tel que $P(\alpha) = 0$. Sinon, il est dit \emph{transcendant sur $K$}. L'extension $L/K$ est dite \emph{algébrique} si tout élément de $L$ est algébrique sur $K$. Sinon, on dira qu'elle est \emph{transcendante}.
\end{definition}

\begin{example} \leavevmode
    \begin{enumerate}
        \item $e$ est transcendant dans $\C/\Q$ (Hermite 1873). $\pi$ est transcendant dans $\C/\Q$ (Lindemann 1882).
        \item $\sqrt{2}$ et $\sqrt[n]{n}$ sont algébriques dans $\C/\Q$.
        \item $2^{\sqrt{2}}$ est transcendant dans $\R/\Q$ (Gelfond-Schneider 1934).
        \item On ne sait pas si $e+\pi$ ou $e-\pi$ sont transcendants.
    \end{enumerate}
\end{example}

\begin{theorem} \label{thmpolmin}
    Soit $\alpha \in L$. Soit $E_\alpha : \applic{K[X]}{L}{P(X)}{P(\alpha)}$. C'est un $K$-morphisme d'anneaux. De plus :
    \begin{enumerate}
        \item L'élément $\alpha$ est transcendant dans $L/K$ si et seulement si $E_\alpha$ est injectif. Dans ce cas, il induit un isomorphisme de $K$-algèbres $K[X] \simeq K[\alpha]$ qui se prolonge en un $K$-isomorphisme $K(X) \to K(\alpha)$. Dans ce cas, $\dim_K K[\alpha]$ et $\dim_K K(\alpha)$ sont infinies.
        \item L'élément $\alpha$ est algébrique dans $L/K$ si et seulement si $E_\alpha$ n'est pas injectif. Dans ce cas, il existe un unique polynôme unitaire $P_\alpha$ dans $K[X]$ de degré minimal tel que $P_\alpha (\alpha) = 0$. Ce polynôme est irréductible sur $K$. De plus, $K[\alpha] = K(\alpha) \simeq K[X]/(P_\alpha)$ et $[K(\alpha) : K] = \deg P_\alpha$.
    \end{enumerate}
\end{theorem}

\begin{definition}
    Le polynôme $P_\alpha \in K[X]$ issu du théorème précédent est appelé le \emph{polynôme minimal de $\alpha \in L$ sur $K$}.
\end{definition}

\begin{proof}[Preuve du théorème \ref{thmpolmin}] \leavevmode
    \begin{enumerate}
        \item L'élément $\alpha$ est transcendant si et seulement si $E_\alpha$ est injectif par définition. Alors, $E_\alpha (K[X]) \simeq \Im(E_\alpha) = K[\alpha]$, donc $E_\alpha$ induit un $K$-isomorphisme $K[X] \xrightarrow{\sim} K[\alpha]$ et il se prolonge à $K(X) \xrightarrow{\sim} K(\alpha)$ par $\frac{P}{Q} \mapsto P(\alpha)Q(\alpha)^{-1}$ (bien défini car $Q(\alpha) \ne 0$ pour $Q \in K[X]\setminus\{0\}$).
        \item L'élément $\alpha$ est algébrique si et seulement si $\ker E_\alpha \ne \{0\}$ par définition. Comme $K[X]$ est principal, $\ker E_\alpha = (P_\alpha)$ pour un certain polynôme $P_\alpha \in K[X]$ qu'on peut choisir unitaire. Ce polynôme est de degré minimal dans $\ker E_\alpha \setminus \{0\}$. De plus, $\ker E_\alpha$ est premier, donc $P_\alpha$ est irréductible. Comme $K[X]$ est principal, $\ker E_\alpha$ est maximal, donc $K[X]/(\ker E_\alpha)$ est un corps, isomorphe à $K[\alpha]$ par théorème d'isomorphisme. En particulier, $K(\alpha) = K[\alpha]$. En notant $n = \deg P_\alpha$, $\{1,\alpha,\dots,\alpha^{n-1}\}$ est une base de $K[\alpha]/K$. En effet, elle est libre, sinon on aurait un polynôme de $K[X]$ annulateur de $\alpha$ de degré strictement plus petit que $\deg P_\alpha$. De plus, pour $x \in K[\alpha]$, il existe $T \in K[X]$ tel que $x = T(\alpha)$. Par division euclidienne, on a $T = QP_\alpha + R$ avec $\deg R < n$, donc $T(\alpha) = R(\alpha)$, donc $x \in \Vect(1,\alpha,\dots,\alpha^{n-1})$. On a donc $[K(\alpha) : K] = \dim_K K[\alpha] = \deg P_\alpha$.
    \end{enumerate}
\end{proof}

\begin{remark} \leavevmode
    \begin{enumerate}
        \item On a aussi montré que $[K[X]/(P_\alpha) : K] = \deg P_\alpha$ et que $K[\alpha] = \{a_0 + a_1 \alpha + \cdots + a_{n-1}\alpha^{n-1} \mid a_i \in K\}$, où $\alpha \in L$ est algébrique sur $K$ et $n = \deg P_\alpha$.
        \item Si $P$ est un polynôme irréductible dans $K[X]$, alors $(P)$ est un idéal maximal, donc $L = K[X]/(P)$ est un corps et $L/K$ une extension. Soit $\alpha = X + (P) \in L$. Alors, le morphisme projection $\applic{K[X]}{L}{X}{\alpha}$ coïncide avec $E_\alpha$. Dans ce cas, $P_\alpha = uP$ où $u \in K^\times$ est tel que $P_\alpha$ soit unitaire. On a alors $[L : K] = \deg P$.
        \item Pour $\alpha \in L$ algébrique sur $K$ et $Q \in K[X]$, on a $Q(\alpha) = 0$ si et seulement si $P_\alpha \mid Q$ dans $K[X]$.
    \end{enumerate}
\end{remark}

\begin{example} \leavevmode
    \begin{enumerate}
        \item Dans $\C/\R$, tout polynôme minimal d'un élément de $\C$ est de degré au plus $2$.
        \item Dans $K(X)/K$, l'élément transcendant $X$ est transcendant sur $K$.
        \item Dans $\Q(\sqrt{2})/\Q$, le polynôme minimal de $a + b \sqrt{2}$ avec $b \ne 0$ est $(X-a)^2 - 2b^2 \in \Q[X]$.
    \end{enumerate}
\end{example}

\begin{proposition}
    Tout extension de corps finie est algébrique.
\end{proposition}

\begin{proof}
    Soit $L/K$ une extension finie. Soit $\alpha \in L$. Alors, $K(\alpha)/K$ est aussi finie, donc $\alpha$ est algébrique.
\end{proof}

\begin{example}
    On verra plus tard que $\overline{\Q} = \{\alpha \in \C \mid \alpha \text{ algébrique dans } \C/\Q\}$ est dénombrable. Remarquons que par définition, $\overline{\Q}/\Q$ est algébrique. De plus, on démontrera que $\overline{\Q}/\Q$ n'est pas finie et donc pas de type fini.
\end{example}

\begin{corollary}
    Soit $L/K$ une extension de type fini. Si $L$ est engendré par des éléments algébriques, alors $L/K$ est finie et algébrique.
\end{corollary}

\begin{proof}
    Soit $L/E/K$ une tour de corps telle que $E/K$ soit finie et algébrique et $x \in L$ algébrique sur $K$. Alors, $x$ est algébrique sur $E$ donc $E(x)/E$ est finie donc $E(x)/K$ est finie (donc algébrique) par le théorème \ref{thmdeg}. Le corollaire suit par récurrence sur le cardinal de l'ensemble générateur.
\end{proof}

\begin{corollary}
    La somme et le produit de deux éléments algébriques de $L/K$ sont algébriques. L'ensemble des éléments algébriques de $L/K$ est une sous-extension, notée $\overline{K}^L$ et appelée \emph{clôture algébrique de $K$ dans $L$}. De plus, $\overline{K}^L / K$ est algébrique.
\end{corollary}

\begin{proof}
    Soient $x,y \in L$ algébriques sur $K$. Alors, $K(x,y)/K$ est une extension algébrique par ce qui précède. Comme $x-y, xy \in K(x,y)$ (et $y^{-1} \in K(x,y)$ si $y \neq 0$), ces éléments sont algébriques sur $K$. Le reste en découle.
\end{proof}

\begin{example} \leavevmode
    \begin{enumerate}
        \item L'extension $\R/\Q$ n'est ni algébrique, ni de type fini.
        \item L'extension $K(X)/K$ est de type fini mais pas algébrique.
        \item L'extension $\Q(\sqrt[3]{2},j)/\Q$ est algébrique et finie.
        \item L'extension $\overline{\Q}/\Q$ où $\overline{\Q}$ est la clôture algébrique de $\Q$ dans $\C$ n'est pas de type fini ou finie mais est algébrique.
    \end{enumerate}
\end{example}

\begin{definition}
    Soit $L/K$ une extension. Si $\overline{K}^L = K$, on dit que \emph{$K$ est algébriquement clos dans $L$}. Remarquons qu'alors l'extension $L/\overline{K}^L$ est totalement transcendante.
\end{definition}

\begin{corollary}
    Une extension de corps engendrées par des éléments algébriques est algébrique.
\end{corollary}

\begin{proof}
    Soit $L/K$ une extension de corps et $S \subseteq L$ où $S$ contient uniquement des éléments algébriques sur $K$ tel que $L = K(S)$. Alors, $S\subseteq \overline{K}^L$ et $L/K$ est algébrique.
\end{proof}

\begin{proposition}
    Soit $M/L/K$ une tour de corps. Alors, $M/K$ est algébrique si et seulement si $M/L$ et $L/K$ le sont.
\end{proposition}

\begin{proof}
    Le sens direct est clair. Supposons donc $M/L$ et $L/K$ algébriques. Soit $\alpha \in M$. On dispose de $P = \sum\limits_{i=0}^n a_i X^i \in L[X]$ non nul tel que $P(\alpha) = 0$. Alors, $\alpha$ est algébrique dans $M/K(a_1,\dots,a_n)$. Alors, $[K(a_0,\dots,a_n)(\alpha) : K(a_0,\dots,a_n)] < \infty$. De plus, les éléments $a_i$ sont algébriques sur $K$ car $L/K$ est algébrique, donc $[K(a_0,\dots,a_n) : K] < \infty$. Le théorème \ref{thmdeg} donne $[K(a_0,\dots,a_n)(\alpha) : K] < \infty$, ce qui démontre que $\alpha$ est algébrique sur $K$. Ainsi, $M/K$ est algébrique.
\end{proof}

\subsection{Degré de transcendance}

\begin{definition}
    Soit $L/K$ une extension de corps. Une famille $(\alpha_i)_{i\in I}$ d'éléments de $L$ est dite \emph{algébriquement indépendante} sur $K$ si le morphisme de $K$ algèbres $\applic{K[X_i \mid i\in I]}{L}{X_i}{\alpha_i}$ est injectif. L'extension $L/K$ est dite \emph{transcendante pure} si elle est engendrée par une famille algébriquement indépendante. Cela est équivalent à demander que $L$ soit $K$-isomorphe à $K(X_i \mid i\in I)$.
\end{definition}

\begin{remark} \leavevmode
    \begin{enumerate}
        \item La famille d'éléments $(\alpha_i)_{i\in I}$ de $L/K$ est algébriquement indépendante si et seulement si pour tout $\{i_1,\dots,i_n\} \subseteq I$ et pour tout $P(X_1,\dots,X_n) \in K[X_1,\dots,X_n]$, on a $P(\alpha_{i_1},\dots,\alpha_{i_n}) \ne 0$.
        \item Si $(\alpha_i)_{i\in I}$ est algébriquement indépendante dans $L/K$, le $K$-morphisme $\applic{K[X_i \mid i\in I]}{L}{X_i}{\alpha_i}$ est injectif et se prolonge donc à un $K$-morphisme de corps $\applic{K(X_i \mid i\in I)}{L}{\frac{P(X_{i_1}, \dots,X_{i_n})}{Q(X_{j_1},\dots,X_{j_m})}}{\frac{P(\alpha_{i_1}, \dots,\alpha_{i_n})}{Q(\alpha_{j_1},\dots,\alpha_{j_m})}}$ qui a pour image $K(\alpha_i \mid i \in I)$.
    \end{enumerate}
\end{remark}

\begin{lemma} \label{lemalgind}
    Soit $L/K$ une extension de corps et $S\subseteq L$ un ensemble algébriquement indépendant dans $L/K$. Soit $x \in L\setminus S$. Alors $S \cup \{x\}$ est algébriquement indépendant dans $L/K$ si et seulement si $x$ est transcendant dans $L/K(S)$.
\end{lemma}

\begin{proof}
    Supposons $x$ algébrique dans $L/K(S)$. Soit $P \in K(S)[X] \setminus \{0\}$ tel que $P(x) = 0$. On écrit $P = \sum\limits_{i=0}^n a_i X^i$ avec $a_i \in K(S)$. On dispose alors de $\exists b_i, c_i \in K[S]$ tels que $a_i = \frac{b_i}{c_i}$ avec les $c_i$ non nuls. En multipliant $P$ par $\prod\limits_{i} c_i$, on obtient un polynôme $P_1 \in K[S][X] \setminus\{0\}$ qui annule $x$. Comme $S$ est algébriquement indépendant, le morphisme $\applic{K[X_s \mid s \in S]}{K[S]}{X_s}{s}$ est injectif. Par l'absurde, si $S \cup \{x\}$ est algébriquement indépendant, on a
    \begin{center}
        \begin{tikzcd}[row sep = 0ex,
            /tikz/column 1/.append style={anchor=base east},
            /tikz/column 3/.append style={anchor=base west},
            ]
            K[X,X_s \mid s \in S] \arrow[r,hookrightarrow] & K[S][X] \arrow[r,hookrightarrow] & L \\
            X_s \arrow[r,mapsto] & s \arrow[r,mapsto] & s \\
            X \arrow[r,mapsto] & X \arrow[r,mapsto] & x
        \end{tikzcd}
    \end{center}
    Ce qui contredit l'existence de $P_1 \in K[S][X] \setminus\{0\}$ qui annule $x$. \\
    Réciproquement, supposons $x$ transcendant dans $L/K(S)$. Comme $S$ est algébriquement indépendant, le $K$-morphisme $\applic{K[X_s \mid s\in S]}{K(S)}{X_s}{s}$ est injectif. La transcendance de $x$ implique que le $K(S)$-morphisme $\applic{K(S)[X]}{L}{X}{x}$ est injectif. Donc, le $K$-morphisme
    \[
        \begin{array}[t]{rcl}
            K[X,X_s \mid s\in S] & \to & L \\
            X_s & \mapsto & s \\
            X & \mapsto & x
            \end{array}
    \]
    est injectif, d'où $S \cup \{x\}$ est algébriquement indépendant dans $L/K$.
\end{proof}

\begin{example} \leavevmode
    \begin{enumerate}
        \item L'ensemble $\{\pi,\pi-1\}$ n'est pas algébriquement indépendant dans $\R/\Q$.
        \item On ne sait pas si $\{\pi,e\}$ est algébriquement indépendant dans $\R/\Q$ et donc on ne sait pas si $\Q(\pi,e)/\Q$ est purement transcendante.
        \item $\{\sqrt{2},\pi\}$ n'est pas algébriquement indépendant dans $\R/\Q$ et $\Q(\sqrt{2},\pi)/\Q$ n'est pas transcendante pure.
        \item $K(X_1,\dots,X_n)/K$ est transcendante pure.
        \item $\R(X,\sqrt{1-X^2})/\R$ est transcendante pure car égale à $\R\left(\frac{1-X}{\sqrt{1-X^2}}\right)/\R$ (paramétrisation du cercle).
        \item $\Q(X,\sqrt{X^3 - X})/\Q$ n'est pas transcendante pure, même si elle est transcendante.
        \item Dans la tour $\R(X,Y)/\R(X+Y)/\R$, l'extension $\R(X,Y)/\R(X+Y)$ est transcendante pure.
    \end{enumerate}
\end{example}

\begin{definition}
    Une \emph{base de transcendance pour $L/K$} est un sous-ensemble algébriquement indépendant dans $L/K$ maximal.
\end{definition}

\begin{lemma} \label{lembasetrans}
    Un ensemble algébriquement indépendant $S \subseteq L$ est une base de transcendance de $L/K$ si et seulement si $L/K(S)$ est algébrique.
\end{lemma}

\begin{proof}
    Le lemme précédent nous dit qu'on peut ajouter un élément $x$ à $S$ en préservant l'indépendance algébrique si et seulement si $x$ est transcendant dans $L/K(S)$, ce qui conclut.
\end{proof}

\begin{theorem}
    Soit $L/K$ une extension. Elle contient une base de transcendance. Toute les bases de transcendance de $L/K$ ont même cardinal.
\end{theorem}

\begin{definition}
    Le cardinal commun des bases de transcendance d'une extension $L/K$ est appelé le \emph{degré de transcendance de $L/K$} et noté $\degtr(L/K)$.
\end{definition}

\begin{proof}[Preuve de l'existence]
    Soit $E \subseteq L$ un ensemble algébriquement indépendant dans $L/K$. Soit $S \subseteq L$ un ensemble générateur de $L/K$ tel que $E \subset S$. Soit
    \[\mathcal{B} = \{U \subseteq L \mid U \text{ algébriquement indépendant}, E  \subseteq U \subseteq S \}\]
    C'est un ensemble non vide ordonné pour l'inclusion. Soit $(T_i)_{i\in I}$ une chaîne dans $\mathcal{B}$. Alors, $\bigcup\limits_i T_i \in \mathcal{B}$ est un majorant de la chaîne. En effet, si des éléments $a_1,\dots,a_n$ dans l'union sont annulés par un polynôme $P$, comme $(T_i)$ est totalement ordonné pour l'inclusion, ces éléments sont contenus dans un certain $T_j$ qui est algébriquement indépendant, donc $P = 0$, donc l'union est bien dans $\mathcal{B}$. Par le lemme de Zorn, on dispose d'un ensemble algébriquement indépendant maximal $B$ contenu dans $S$. \\
    S'il existe $t \in S \setminus B$ tel que $t$ est transcendant sur $L/K(B)$, alors $B \cup \{t\} \subseteq S$ est algébriquement indépendant par le lemme \ref{lemalgind}. Cela contredit la maximalité de $B$, donc tout $t \in S \setminus B$ est algébrique sur $L/K(B)$, d'où $\underbrace{K(B)(S\setminus B)}_L/K(B)$ est algébrique et on conclut que $B$ est une base de transcendance par le lemme \ref{lembasetrans}.
\end{proof}

\begin{remark}
    La preuve précédente montre que tout ensemble générateur de l'extension $L/K$ contient une base de transcendance.
\end{remark}

\begin{proof}[Preuve de l'unicité des cardinaux]
    Soient $B_1,B_2 \subseteq L$ deux bases de transcendance de $L/K$. Quitte à permuter, on suppose $\Card(B_1) \le \Card(B_2)$. \\
    Supposons d'abord $\Card(B_2)$ infini. Pour $\alpha \in B_1$, il existe $B_\alpha \subseteq B_2$ fini (il suffit de prendre les coefficients d'un polynôme annulateur de $\alpha$ et de remarquer qu'ils s'expriment chacun à partir d'un nombre fini d'éléments de $B_2$) tel que $\alpha$ est algébrique sur $K(B_\alpha)$. Soit $B' = \bigcup\limits_{\alpha \in B_1} B_\alpha$. Soit $\beta \in B_2 \setminus B'$. Alors $\beta$ est algébrique dans $L/K(B_1)$. De plus, par construction de $B'$, $K(B_1)/K(B')$ est algébrique. Ainsi $\beta$ est algébrique dans $L/K(B')$, ce qui est absurde car $B'\cup \{\beta\} \subseteq B_2$ n'est pas algébriquement indépendante. Ainsi $B_2 = B'$, d'où $B_1$ est infinie (sinon $B' = B_2$ serait finie) et l'arithmétique des cardinaux nous dit que $\Card(B_1) = \Card(B_2)$. \\
    Supposons maintenant $\Card(B_2)$ fini. Alors, $\Card(B_1)$ est fini aussi par hypothèse. On explicite $B_1 = \{\beta_1,\dots,\beta_m\}$ et $B_2 = \{\alpha_1,\dots,\alpha_n\}$, alors $m \le n$. On raisonne par récurrence sur $m$. Si $m=0$, alors $L/K$ est algébrique, donc forcément $B_2 = \varnothing$ et $n=0$. Si $m > 0$, on dispose d'un polynôme $P \in K[X,Y_1,\dots,Y_n]$ tel que $P(\beta_1,\alpha_1,\dots,\alpha_n) = 0$ avec $\beta_1$ et au moins un des $\alpha_i$ qui apparaissent. Supposons quitte à permuter que c'est $\alpha_1$ qui apparaît. Soit $B' = \{\beta_1,\alpha_2,\dots,\alpha_n\}$. Alors, $K(B',\alpha_1)/K(B')$ est algébrique. Si $B'$ n'était pas algébriquement indépendant, on trouverait un polynôme $Q$ à coefficients dans $K$ tel que $Q(\beta_1,\alpha_2,\dots,\alpha_n) = 0$ où forcément $\beta_1$ apparaîtrait, d'où $\beta_1$ serait algébrique sur $K(\alpha_2,\dots,\alpha_n)$ et donc $\alpha_1$ aussi, ce qui est absurde. Alors, $B'$ est algébriquement indépendant dans $L/K$. Ainsi $\{\alpha_2,\dots,\alpha_n\}$, $\{\beta_2,\dots,\beta_m\}$ sont des bases de transcendance de $L/K(\beta_1)$, et on conclut avec l'hypothèse de récurrence.
\end{proof}

\begin{remark}
    On a $\degtr(L/K) = 0$ si et seulement si $L/K$ est algébrique.
\end{remark}

\begin{remark}
    Soit $L = K(\alpha)/K$ avec $\alpha$ transcendant. Soit $M/K$ une extension. Alors, on a une bijection
    \begin{center}
        \begin{tikzcd}[row sep = 0ex,
            /tikz/column 1/.append style={anchor=base east},
            /tikz/column 2/.append style={anchor=base west},
            ]
            \{\text{$K$-morphismes } L = K(\alpha) \to M\} \arrow[r, leftrightarrow,"\text{bij}"] &
            \{\text{transcendants de } M/K\} \\
            (\varphi : \alpha \mapsto \beta) \arrow[r, mapsto] & \beta
        \end{tikzcd}
    \end{center}
\end{remark}

\begin{remark}
    Pour $L/K$, on peut toujours prendre une sous-extension maximale $E$ telle que $E/K$ est purement transcendante et $L/E$ algébrique.
\end{remark}

\begin{example} \leavevmode
    \begin{enumerate}
        \item $L = K(X)/K$, $B_1 = \{X\}$, $B_2 = \{X^2\}$. $K(B_1) = K(X)$, $K(B_2) = K(X^2)$.
        \item $\degtr(K(X_i \mid i \in I)/K) = \Card(I)$.
        \item $1 \le \degtr(\Q(\pi,e)/\Q) \le 2$.
        \item $\degtr(\R(X,\sqrt{1-X^2}/\R)) = 1$
        \item $\degtr(\Q(X,\sqrt{X^3 - X})/\Q) = 1$
        \item $\degtr(\C/\Q) = \infty$
    \end{enumerate}
\end{example}

\begin{corollary}
    Soit $M/L/K$ une tour de corps. Alors
    \[\degtr(M/K) = \degtr(M/L) + \degtr(L/K)\]
\end{corollary}

\begin{proof}
    Soient $B_1,B_2$ des bases de transcendance de $M/L$ et $L/K$, respectivement. On a $B_1 \cap B_2 = \varnothing$ car $B_1 \cap L =\varnothing$ et $B_2 \subseteq L$. L'ensemble $B_1 \cup B_2$ est algébriquement indépendant dans $M/K$. Sinon, on obtiendrait une relation de la forme
    \[P(\alpha_1,\dots,\alpha_n,\beta_1,\dots,\beta_m) = 0\]
    où $\alpha_1,\dots,\alpha_n \in B_1$ et $\beta_1,\dots,\beta_m \in B_2$ sont distincts deux à deux et $P$ est un polynôme à coefficients dans $K$. Mais alors, par indépendance algébrique de $B_1$, aucun $\alpha_i$ ne peut apparaître dans le polynôme (car les $\beta_i$ sont dans $L$), d'où $P(\beta_1,\dots,\beta_m) = 0$ et contredit l'indépendance algébrique de $B_2$. De plus, $M/K(B_1 \cup B_2)$ est algébrique. En effet, $M/L(B_1)$ et $L/K(B_2)$ le sont, donc $M/L(B_1 \cup B_2)$ est algébrique et les éléments de $L,B_1\cup B_2$ sont algébriques dans $M/K(B_1 \cup B_2)$.
\end{proof}

\begin{proposition}[Critère d'Eisenstein]
    Soit $P(X) = \sum\limits_{i=0}^n a_i X^i \in \Z[X]$. On suppose qu'il existe $p$ premier tel que $p \mid a_0,\dots,p\mid a_{n-1}$, $p \nmid a_n$ et $p^2 \nmid a_0$. Alors, $P$ est irréductible dans $\Q[X]$.
\end{proposition}

\begin{proof}
    Soit une factorisation $P = QR$ avec $Q,R \in \Q[X]$ de degrés non nuls. \todo
\end{proof}

\begin{corollary}
    L'extension $\overline{\Q}/\Q$ n'est pas finie. De même, $\overline{\Q}^\R = \overline{\Q} \cap \R$ n'est pas une extension finie de $\Q$.
\end{corollary}

\begin{proof}
    Pour $n \in \N^*$ et $p$ premier fixé, $P_n(X) = p + pX + \cdots pX^{n-1} + X^n \in \Z[X]$ est irréductible par le critère d'Eisenstein. Ainsi, $[\Q[X]/(P_n(X)) : \Q] = n$. De plus, si $\alpha_n \in \C$ est une racine de $P_n$, on a $\Q[X]/(P_n(X)) \underset{\Q}{\simeq} \Q(\alpha_n)$. Comme $\Q(\alpha_n) \subseteq \Qbar$, $\Qbar$ contient des extensions de $\Q$ de degré arbitrairement grand, ce qui prouve $[\Qbar : \Q] = \infty$. Pour $\Qbar^\R$, on peut prendre $\alpha_n \in \R$ pour $n$ impair.
\end{proof}

\begin{remark}
    On a $\Qbar = \bigcup_{P \in \Q[X]} \{\text{racines de } P\}$ donc $\Qbar$ est dénombrable comme union dénombrable d'ensembles finis.
\end{remark}

\begin{proposition}
    On a $\degtr(\C/\Q) = \degtr(\C/\Qbar) = \infty$ non dénombrable.
\end{proposition}

\begin{proof}
    \todo
\end{proof}

\begin{remark}
    De manière similaire, $\degtr(\Qbar^\R/\Q) = \infty$ non dénombrable.
\end{remark}

\begin{example}
    Soit $B$ une base de transcendance de $\C/\Q$. Soit $\varphi : B \to B$ une application quelconque injective. Elle induit un $\Qbar$-morphisme $\Qbar(B) \xrightarrow{\overline{\varphi}} \Qbar(B)$ qui peut toujours se prolonger (a priori non uniquement) à $\C \xrightarrow{\varphi_0} \C$. Si $\varphi$ n'est pas surjective, alors $\varphi_0$ sera un morphisme de corps non surjectif.
\end{example}

Références :
\begin{itemize}
    \item Fields and Galois Theory, James Stuart Milne
    \item Polycopiés du cours algèbre 2 de l'ENS en français de Jean-François Dat, Olivier Debarre et Ariane Mézard\footnote{Son poly n'est pas sur son site à ce moment mais est accessible à travers la \emph{Wayback Machine}.}
    \item Algebra, Serge Lang
\end{itemize}

\subsection{Racines de polynômes}

Soit $K$ un corps.

\begin{lemma}
    Si $P \in K[X]$ a une racine $\alpha \in K$, alors $(X-\alpha) \mid P$.
\end{lemma}

\begin{proof}
    La division euclidienne fonctionne dans $K[X]$ car $K$ est un corps, donc $P = (X-\alpha)Q + c$ avec $c \in K$ et évaluer en $\alpha$ donne le résultat.
\end{proof}

\begin{corollary}
    Si $P \in K[X]$ est de degré $d$, alors $P$ a au plus $d$ racines dans $K$.
\end{corollary}

\begin{proof}
    Se déduit du lemme précédent avec une récurrence.
\end{proof}

\begin{definition}
    On dit que $P$ est \emph{scindé} si $P = c \prod\limits_{i=1}^d (X-\alpha_i)$ dans $K[X]$. Autrement dit, les facteurs irréductibles de $P$ sont de degré $1$ (ou $P \in K$). Ici, $c \in K$.
\end{definition}

\begin{example}
    $X^4 - 1$ est scindé dans $\C[X]$ mais pas dans $\R[X]$.
\end{example}

\begin{definition}
    Soit $L/K$ une extension de corps et $P \in K[X]$ irréductible. On dit que $L$ est \emph{un corps de rupture de $P$} s'il existe $\alpha \in L$ tel que $P(\alpha) = 0$ et $L=K(\alpha)$.
\end{definition}

\begin{example} \leavevmode
    \begin{enumerate}
        \item $\C$ est un corps de rupture de $X^2 + 1 \in \R[X]$.
        \item $\Q(\sqrt{2})$ est un corps de rupture de $X^2 - 2 \in \Q[X]$.
        \item $\Q(\sqrt[3]{2})$ et $\Q(j\sqrt[3]{2})$ sont des corps de rupture de $X^3 - 2 \in \Q[X]$.
        \item $\Q(\sqrt[5]{3})$ est un corps de rupture de $X^5 - 3 \in \Q[X]$.
    \end{enumerate}
\end{example}

\begin{remark}
    Un corps de rupture n'est pas en général un corps dans lequel le polynôme est scindé, c'est-à-dire qu'en général il ne contient pas \og toutes \fg les racines du polynôme.
\end{remark}

\begin{proposition}
    Soit $P \in K[X]$ irréductible. Pour toute extension $L/K$ et toute racine $\alpha \in L$ de $P$, on a un unique $K$-morphisme de corps $K_P = K[X]/(P) \hookrightarrow L$ tel que $X + (P)$ soit envoyé sur $\alpha$. En particulier, si $L$ est un corps de rupture de $P$, alors ce morphisme est un isomorphisme. On a alors $[L : K] = \deg P$.
\end{proposition}

\begin{proof}
    Prenons le $K$-morphisme $E_\alpha : \applic{K[X]}{L}{Q}{Q(\alpha)}$. Comme $\alpha$ est racine de $P$, $(P) \subset \ker E_\alpha$, d'où $E_\alpha$ induit un $K$-morphisme $K[X]/(P) \hookrightarrow L$. Pour l'unicité, si $\phi : K[X]/(P) \to L$ est un autre $K$-morphisme tel que $\phi(X+(P)) = \alpha$, alors on a $\phi \circ \pi = E_\alpha$ où $\pi$ est la projection canonique $K[X] \to K[X]/(P)$, donc par la propriété universelle du quotient $\phi = E_\alpha$. Si $L$ est un corps de rupture de $P$, alors $L = K(\alpha) = K[\alpha] = \im E_\alpha$, donc $E_\alpha$ est surjectif et le $K$-morphisme $K_P \hookrightarrow L$ induit est un isomorphisme. On a alors $[L : K] = [K_P : K] = \deg P$ (voir le théorème \ref{thmpolmin}).
\end{proof}

\begin{corollary} \label{nbmorcorpsrupture}
    Soit $P \in K[X]$ irréductible et $L/K$ un corps de rupture. Soit $M/K$ une extension dans laquelle $P$ a une racine. Il existe un $K$-morphisme $L \hookrightarrow M$ et le nombre de tels morphismes est le nombre de racines distinctes de $P$ dans $M$. C'est donc au plus $\deg P = [L : K]$, avec égalité si $P$ a $\deg P$ racines distinctes dans $M$.
\end{corollary}

\begin{proof}
    Par la proposition précédente, on a un morphisme $\varphi : L \to M$. Pour $\alpha \in L$ une racine de $P$, comme $\varphi$ est un morphisme de corps, $P(\varphi(\alpha)) = \varphi(P(\alpha)) = 0$. Ainsi $\varphi(\alpha)$ est une racine de $P$ dans $M$. Comme $\varphi$ est un $K$-morphisme et qu'on peut supposer $L=K(\alpha)$, on déduit que $\varphi$ est uniquement déterminé par $\varphi(\alpha)$, donc le nombre de morphismes est bien le nombre de racines de $P$ dans $M$.
\end{proof}

\begin{theorem} \label{thmmorpolscinde}
    Soit $L/K$ une extension de corps engendrée par des racines d'un polynôme $P \in K[X]$. Soit $M/K$ une extension dans laquelle $P$ est scindé. Il existe un $K$-morphisme de corps $L \hookrightarrow M$. Il y a au plus $[L : K]$ tels morphismes avec égalité quand $P$ a $\deg P$ racines dans $M$.
\end{theorem}

\begin{proof}
    Soit $L=K(\alpha_1,\dots,\alpha_n)$, où $\alpha_1,\dots,\alpha_n \in L$ sont des racines de $P$. Soit $P_{\alpha_1} \in K[X]$ le polynôme minimal de $\alpha_1$ dans $L/K$. Comme $P(\alpha_1) = 0$, $P_{\alpha_1} \mid P$ dans $K[X]$. Comme $K \subseteq M$ et $P$ est scindé dans $M[X]$, $P_{\alpha_1}$ est scindé dans $M[X]$, avec $\deg P_{\alpha_1}$ racines si $P$ a $\deg P$ racines dans $M$. Par le corollaire précédent, il existe un $K$-morphisme de corps $K(\alpha_1) \overset{\varphi_1}{\hookrightarrow} M$ et le nombre de tels morphismes est au plus $[K(\alpha_1) : K]$ avec égalité dans le cas où $P$ (et donc $P_{\alpha_1}$) a $\deg P$ (resp. $\deg P_{\alpha_1}$) racines dans $M$. \\
    On considère maintenant $P_{\alpha_2}$ le polynôme minimal de $\alpha_2$ dans $L/K(\alpha_1)$. On a encore $P_{\alpha_2} \mid P$ dans $K(\alpha_1)[X]$, donc $\varphi_1 (P_{\alpha_2})$ est scindé dans $M[X]$ avec $\deg P_{\alpha_2}$ racines si $P$ a $\deg P$ racines dans $M$, car $P_{\alpha_2} \mid P$ dans $K(\alpha_1)[X]$ implique $\varphi_1(P_{\alpha_2}) \mid \varphi_1(P) = P$ dans $\varphi_1 (K(\alpha_1))$. Alors, on a un $K(\alpha_1)$-morphisme $\varphi_2 : K(\alpha_1, \alpha_2) \hookrightarrow M$ (on injecte $K(\alpha_1)$ dans $M$ par $\varphi_1$) et il y a au plus $[K(\alpha_1, \alpha_2) : K(\alpha_1)]$ tels morphismes avec égalité quand $P$ a $\deg P$ racines dans $M$. Ainsi, en itérant ce procédé, on obtient qu'il y a un $K$-morphisme $L \hookrightarrow M$ et au plus
    \[[K(\alpha_1,\dots,\alpha_n) : K(\alpha_1,\dots,\alpha_{n-1})] \cdots [K(\alpha_1) : K] = [L : K]\]
    tels morphismes, avec exactement ce nombre quand $P$ a $\deg P$ racines dans $M$.
\end{proof}

\begin{definition}
    Soit $P \in K[X]$. Une extension $L/K$ est dite \emph{un corps de décomposition de $P$} si $P$ est scindé dans $L$ et que $L$ est engendré par les racines de $P$.
\end{definition}

\begin{corollary}
    Soit $P \in K[X]$ un polynôme. Il existe une extension $L/K$ qui est un corps de décomposition de $P$. Deux corps de décomposition de $P$ sont $K$-isomorphes. De plus, $[L : K] \le (\deg P)!$.
\end{corollary}

\begin{proof}
    SOit $L_1/K$ un corps de rupture pour un facteur irréductible de $P \in K[X]$. Soit $L_1 = K(\alpha_1)$ avec $P(\alpha_1) = 0$. Alors, $\frac{P}{X-\alpha_1} \in L_1[X]$ et on construit un corps de rupture $L_2 = L_1(\alpha_2)$ d'un facteur irréductible de $\frac{P}{X-\alpha_1} \in L_1 [X]$. On continue ainsi jusqu'à ce que $\frac{P}{Q} \in L_{\deg P} [X]$ soit de degré $1$. On a alors $[L_1 : K] \le \deg P$, $[L_2 : L_1] \le \deg P - 1$, \dots, $[L_{\deg P} : L_{\deg P - 1}] \le 2$, d'où $[L_{\deg P} : K] \le (\deg P)!$. Soit maintenant $M/K$ un autre corps de décomposition de $P$. Par le théorème précédent, il existe des $K$-morphismes $L \hookrightarrow M$ et $M \hookrightarrow L$, d'où $[M : K] = [L : K]$ et $M \simeq L$.
\end{proof}

\begin{example} \leavevmode
    \begin{enumerate}
        \item $\Q(\sqrt[3]{2},j)/\Q$ est un corps de décomposition de $X^3 - 2$ de degré $3! = 6$.
        \item $\Q(i)/\Q$ est un corps de décomposition de $X^4 - 1$ de degré $2 < 4!$.
        \item Soit $P = X^3 + a X^2 + b X + c \in \Q[X]$, irréductible et $\alpha_1,\alpha_2,\alpha_3 \in \C$ ses racines. Alors pour deux de ces racines, par exemple $\alpha_1,\alpha_2$, $\Q(\alpha_1,\alpha_2)/\Q$ est un corps de décomposition de $P$. On a aussi $[\Q(\alpha_1) : \Q] = 3$, d'où $3 = [\Q(\alpha_1,\alpha_2) : \Q] \le 3! = 6$ et $3 \mid [\Q(\alpha_1,\alpha_2) : \Q]$, donc $[\Q(\alpha_1,\alpha_2) : \Q] \in \{3,6\}$.
        \item Pour $P = 1 + X + \cdots + X^{p-1} \in \Q[X]$ avec $p$ premier, corps de décomposition de $P$ est $\Q(\mu_p)/\Q$ une extension cyclotomique.
        \item Pour $P = X^3 - 2$, $[\Q(\alpha_1, \alpha_2) : \Q] = 6$, pour $P = X^3 + X^2 - 2X - 1 \in \Q[X]$, $[\Q(\alpha_1,\alpha_2) : \Q] = 3$.
    \end{enumerate}
\end{example}

\begin{corollary}
    Soit $L/K$ une extension finie et $M/K$ une extension. Alors le nombre de $K$-morphismes $L \hookrightarrow M$ est au plus $[L : K]$. Il existe une extension finie $F/M$ tel qu'il existe un $K$-morphisme $L \hookrightarrow F$.
    \begin{center}
        \begin{tikzcd}[row sep=scriptsize, column sep=scriptsize]
            & F \\
            L \arrow[ru,"\exists"] & & M \arrow[lu,"\mathrm{finie}" sloped] \\
            & K \arrow[lu,"\mathrm{finie}" sloped] \arrow[ru]
        \end{tikzcd}
    \end{center}
\end{corollary}

\begin{proof}
    Soit $L = K(\alpha_1,\dots,\alpha_n)$ où $\alpha_1,\dots,\alpha_n$ sont algébriques dans $L/K$. Soit $P \in K[X]$ le produit des polynômes minimaux de $\alpha_1,\dots,\alpha_n$ sur $K$. Soit $F/M$ un corps de décomposition de $P \in K[X] \subseteq M[X]$. Par le théorème, il existe un $K$-morphisme $L \hookrightarrow F$ et le nombre de tels morphismes est au plus $[L : K]$. Tout $K$-morphisme $L \hookrightarrow M$ induit un $K$-morphisme $L \hookrightarrow F$, donc il y a au plus $[L : K]$ tels morphismes.
\end{proof}

\begin{corollary}
    Soient $L_i/K$, $i = 1,2,\dots,n$ des extensions finies et $M/K$ une extension. Il existe une extension finie $F/M$ telle qu'il existe des $K$-morphismes $L_i \hookrightarrow F$, $i = 1,2,\dots,n$. 
\end{corollary}

\begin{proof}
    Par le corollaire précédent, il existe $M_1/M$ finie telle qu'il existe un $K$-morphisme $L_1 \hookrightarrow M_1$. On applique maintenant ce même corollaire aux extensions $L_2/K$ et $M_1/K$ pour obtenir une extension $M_2/K$ finie et l'existence d'un $K$-morphisme $L_2 \hookrightarrow M_2$. En continuant ce procédé, nous arrivons à une extension $F = M_n/M$ finie qui satisfait l'énoncé.
\end{proof}

\begin{definition}
    Soit $(P_i)_{i\in I}$ une famille de polynômes dans $K[X]$. Une extension $L/K$ est dite \emph{un corps de décomposition pour la famille $(P_i)_{i\in I}$} si pour tout $i \in I$, $P_i$ est scindé dans $L$ avec pour ensemble de racines $R_i \subseteq L$ et que $L = K \left(\bigcup\limits_{i\in I} R_i\right)$.
\end{definition}

\subsection{Clôture algébrique}

Soit $K$ un corps.

\begin{proposition}
    Les assertions suivantes sont équivalentes :
    \begin{enumerate}
        \item Tout polynôme $P \in K[X] \setminus K$ a au moins une racine dans $K$.
        \item Tout $P \in K[X] \setminus K$ est scindé dans $K$.
        \item Les polynômes irréductibles dans $K[X]$ sont ceux de degré $1$.
        \item Si $L/K$ est une extension algébrique, alors $L=K$.
    \end{enumerate}
\end{proposition}

\begin{proof}
    \todo
\end{proof}

\begin{definition} \leavevmode
    \begin{enumerate}
        \item Un corps $K$ satisfaisant les propriétés équivalentes de la proposition est dit \emph{algébriquement clos}.
        \item Un corps $L$ est dit la \emph{clôture algébrique} d'un sous-corps $K$ si $L$ est algébriquement clos et que $L/K$ est algébrique.
    \end{enumerate}
\end{definition}

\begin{example} \leavevmode
    \begin{enumerate}
        \item $\C$ est algébriquement clos. $\R$ et $\Q$ ne le sont pas. $K(T)$ ne l'est pas non plus.
        \item $\C$ est une clôture algébrique de $\R$ mais pas une clôture algébrique de $\Q$.
        \item Si $L/K$ est une extension de corps avec $L$ algébriquement clos, alors $\overline{K}^L$ est une clôture algébrique de $K$, donc $\Qbar$ est algébriquement clos et c'est une clôture algébrique de $\Q$.
    \end{enumerate}
\end{example}

\begin{proposition} \label{propclotalgpolscinde}
    Soit une extension $L/K$. Alors $L$ est une clôture algébrique de $K$ si et seulement si l'extension $L/K$ est algébrique et tout polynôme de $K[X]$ est scindé sur $L$.
\end{proposition}

\begin{proof}
    Le sens direct est immédiat. Pour le sens réciproque, soit $P \in L[X]$ un polynôme irréductible. Soit $M/L$ un corps de rupture de $P$. Alors, $M/L$, $L/K$ sont algébriques donc $M/K$ est algébrique. Soit $\alpha \in M$ une racine de $P$ et $P_\alpha$ son polynôme minimal sur $K$. Alors, $P_\alpha \mid P$ dans $L[X]$, mais $P$ est irréductible, donc $c P = P_\alpha$ pour un certain $c\in L$. Or, $P_\alpha \in K[X]$ est scindé dans $L$ par hypothèse,donc $P$ aussi. Comme $P$ est irréductible, forcément $\deg P = 1$, ce qui démontre que $L$ est algébriquement clos. Comme $L/K$ est algébrique, on obtient le résultat.
\end{proof}

\begin{proposition} \label{prolongmorcac}
    Soit $L/K$ algébrique et $M$ un corps algébriquement clos. Tout morphisme $i : K \to M$ se prolonge en un morphisme $\varphi : L \to M$.
    \begin{center}
        \begin{tikzcd}[column sep = tiny]
            L \arrow[rr,"\varphi"] & & M \\
            & K \arrow[lu, hookrightarrow] \arrow[ru, "i" ']
        \end{tikzcd}
    \end{center}
\end{proposition}

\begin{proof}
    \todo zorn
\end{proof}

\begin{proposition}
    Soit $M/K$ algébrique.
    \begin{enumerate}
        \item Si $M$ est algébriquement clos, toute extension algébrique $L/K$ est $K$-isomorphe à un sous-corps de $M$ (une sous-extension de $M/K$). De plus, $M$ est une clôture algébrique de $L$.
        \item Si toute extension finie $L/K$ est isomorphe à une sous-extension de $M/K$, alors $M$ est algébriquement clos.
    \end{enumerate}
\end{proposition}

\begin{proof}
    Pour le (1), on prolonge l'inclusion $K \hookrightarrow M$ en un morphisme $L \hookrightarrow M$ par la proposition précédente. Soit $P \in K[X]$. $P$ est scindé dans un de ses corps de décomposition $L/K$ qui est une extension finie. Ainsi, par hypothèse, $L/K$ est isomorphe à une sous-extension de $M/K$, donc $P$ est scindé sur $M$. On conclut par la proposition \ref{propclotalgpolscinde}.
\end{proof}

\begin{remark}
    Soit $M/K$ une extension avec $M$ algébriquement clos. Alors, par la proposition, pour tout polynôme $P \in K[X]$, $M$ contient \emph{exactement} (\todo : comprendre pourquoi) un corps de décomposition de $P$. De même, pour tout famille de polynôme $(P_i)_{i\in I}$, $M$ contient exactement un corps de décomposition de $(P_i)$.
\end{remark}

\begin{theorem}[Steinitz 1910]
    Soit $K$ un corps. Il existe une clôture algébrique de $K$. Deux clôtures algébriques de $K$ sont $K$-isomorphes.
\end{theorem}

\begin{proof}
    \todo
\end{proof}

\begin{remark}
    Si $\varphi : K \to K'$ est un isomorphisme de corps et que $M$ et $M'$ sont des clôtures algébriques de $K$, $K'$ respectivement, alors $\varphi$ se prolonge en un isomorphisme de corps $\psi : M \to M'$ (non unique en général).
\end{remark}

\begin{example}
    Le morphisme $\varphi : \applic{\R}{\R}{x}{x}$ peut se prolonger en deux morphismes par $\applic{\R(i)}{\R(i)}{i}{i}$ et $\applic{\R(i)}{\R(i)}{i}{-i}$.
\end{example}

\begin{definition}
    Le théorème de Steinitz nous autorise à faire un abus de langage et parler de \og la\fg clôture algébrique d'un corps $K$. On la notera $\overline{K}$.
\end{definition}

\begin{example} \leavevmode
    \begin{enumerate}
        \item $\Qbar \subsetneq \overline{\Q(\pi)} \subsetneq \overline{\C}$. On sait que la deuxième inclusion est stricte car $\overline{\Q(\pi)}$ est dénombrable.
        \item En général, il est difficile de décrire $\overline{K}$ pour un corps $K$. Par exemple, $\overline{K(T)}$ pour un corps $K$ arbitraire.
    \end{enumerate}
\end{example}

\subsection{Éléments conjugués}

\begin{definition}
    Soit $L/K$ une extension de corps. Deux éléments algébriques $\alpha,\beta \in L$ sont dits \emph{conjugués} sur $K$ si leur polynômes minimaux sur $K$ sont les mêmes.
\end{definition}

\begin{example} \leavevmode
    \begin{enumerate}
        \item $i,-i$ sont conjugués dans $\C/\R$.
        \item $\sqrt[3]{2},j\sqrt[3]{2}$ sont conjugués dans $\C/\Q$.
        \item $\sqrt{3},i\sqrt{3}$ ne sont pas conjugués dans $\C/\Q$.
    \end{enumerate}
\end{example}

\begin{proposition}
    Soit $L/K$ une extension de corps. Les énoncés suivants sont équivalents :
    \begin{enumerate}
        \item $\alpha,\beta$ sont conjugués dans $L/K$
        \item Il existe un $K$-isomorphisme $\applic{K(\alpha)}{K(\beta)}{\alpha}{\beta}$.
    \end{enumerate}
\end{proposition}

\begin{proof}
    Si $\alpha,\beta$ sont conjugués dans $L/K$, alors les isomorphismes
    \[\applic{K(\alpha)}{K[X]/(P_\alpha)}{\alpha}{X + (P_\alpha)}\quad\applic{K(\beta)}{K[X]/(P_\beta)}{\beta}{X + (P_\beta)}\]
    montrent $(2)$. Réciproquement, si $\applic{K(\alpha)}{K(\beta)}{\alpha}{\beta}$ est un $K$-isomorphisme, on a que $P_\alpha(\beta) = P_\alpha(\alpha) = 0$ et symétriquement $P_\beta(\alpha) = 0$, d'où $P_\beta \mid P_\alpha$ et $P_\alpha \mid P_\beta$, d'où $P_\alpha = P_\beta$.
\end{proof}

\begin{remark}
    Dans $\overline{K}$, les conjugués de $\alpha \in \overline{K}$ sont les racines de $P_\alpha \in K[X]$. Si $L/K$ est une extension et $\alpha \in L$ est algébrique dans $L/K$, alors les conjugués de $\alpha$ dans $L$ sont les racines de $P_\alpha \in K[X]$ et il y en a au plus $\deg P_\alpha$.
\end{remark}

\begin{example} \leavevmode
    \begin{enumerate}
        \item Dans $\F_p(T)$, le polynôme minimal de $\sqrt[p]{T} \in \overline{\F_p(T)}$ est $X^P - T = (X - \sqrt[p]{T})^p$, donc le seul conjugué de $\sqrt[p]{T}$ est lui-même.
        \item $\sqrt[3]{2}$ est son seul conjugué dans $\Q(\sqrt[3]{2})/\Q$.
    \end{enumerate}
\end{example}

\begin{proposition}
    Deux éléments $\alpha,\beta \in \overline{K}$ sont conjugués sur $K$ si et seulement s'il existe un $K$-isomorphisme $\overline{K} \to \overline{K}$ qui envoie $\alpha$ sur $\beta$.
\end{proposition}

\begin{proof}
    Si $\alpha,\beta$ sont conjugués, ils sont algébriques, donc $\overline{K}$ est un clôture algébrique de $K(\alpha)$ et $K(\beta)$. Ainsi, l'isomorphisme de corps $K(\alpha) \xrightarrow{\sim} K(\beta)$ qui envoie $\alpha$ sur $\beta$ se prolonge en un isomorphisme $\overline{K} \to \overline{K}$. La réciproque est immédiate.
\end{proof}

\subsection{Extension normales}

\begin{definition}
    Une extension algébrique $L/K$ est dite \emph{normale} si tout polynôme $P \in K[X]$ irréductible qui a une racine dans $L$ est scindé sur $L$.
\end{definition}

\begin{example} \leavevmode
    \begin{enumerate}
        \item $\overline{K}/K$ est normale.
        \item $\Q(\sqrt[3]{2})/\Q$ n'est pas normale, $\Q(\sqrt[4]{2})/\Q$ non plus.
        \item $\Q(i)/\Q$, $\Q(\sqrt[3]{2}, j)/\Q$ sont normales.
    \end{enumerate}
\end{example}

\begin{proposition} \label{caracextnormale}
    Soit $L/K$ algébrique, $\overline{K}$ une clôture algébrique de $K$ et $L \hookrightarrow \overline{K}$. Les énoncés suivants sont équivalents :
    \begin{enumerate}
        \item $L/K$ est normale.
        \item Pour $\alpha \in L$, les conjugués de $\alpha$ dans $\overline{K}/K$ sont des éléments de $L$.
        \item Tout $K$-isomorphisme de $\overline{K}$ envoie $L$ sur lui-même.
        \item $L$ est le corps de décomposition d'une famille de polynômes à coefficients dans $K$.
    \end{enumerate}
\end{proposition}

\begin{proof}
    Supposons $L/K$ normale. Alors, $L$ est le corps de décomposition de la famille $(P_\alpha)_{\alpha \in L}$ des polynômes minimaux des éléments de $L$ sur $K$. \\
    Supposons que $L$ est le corps de décomposition de $(P_i)_{i\in I} \in K[X]^I$, et soit $\varphi : \overline{K} \xrightarrow{\sim}\overline{K}$ un $K$-isomorphisme. Soit $R_i \subseteq \overline{K}$ l'ensemble des racines de $P_i$ dans $\overline{K}$. Alors, $L = K(\bigcup_{i\in I} R_i)$. De plus, $\varphi(R_i) = R_i$ pour tout $i$. Ainsi, $\varphi(K(\bigcup_{i\in I} R_i)) = K(\bigcup_{i\in I} R_i)$. \\
    Supposons $(3)$. Soit $\alpha \in L$ et $\beta \in \overline{K}$ un conjugué de $\alpha$ dans $\overline{K}/K$. Alors, il existe un $K$-isomorphisme $\applic{\overline{K}}{\overline{K}}{\alpha}{\beta}$, d'où $\beta = \varphi(\alpha) \in \varphi(L) = L$. \\
    Supposons $(2)$. Soit $P \in K[X]$ irréductible qui possède une racine $\alpha \in L$. Quitte à multiplier par un élément de $K^\times$ pour rendre $P$ unitaire, on peut supposer $P = P_\alpha$. Alors, par $(2)$, les conjugués de $\alpha$ dans $\overline{K}/K$, qui sont les racines de $P_\alpha$ dans $\overline{K}$, sont dans $L$, d'où $P$ est scindé sur $L$.
\end{proof}

\begin{corollary}
    Soit $M/L/K$ une tour de corps. Si $M/K$ est normale, alors $M/L$ l'est aussi.
\end{corollary}

\begin{proof}
    Immédiat par l'équivalence $(1) \iff (4)$ de la proposition \ref{caracextnormale}.
\end{proof}

\begin{corollary}
    Une extension $L/K$ est normale et finie si et seulement si $L$ est le corps de décomposition d'un polynôme sur $K$.
\end{corollary}

\begin{proof}
    Le sens inverse est vrai par le $(4)$ de la proposition \ref{caracextnormale}. Supposons donc $L/K$ normale et finie. Raisonnons par récurrence sur $[L : K]$. Si $[L : K] = 1$, le résultat est clair. Supposons le résultat vrai pour $[L : K] < n$, $n \in \N^*$. Soit $L/K$ tel que le résultat est vrai $[L : K] = n$. Soit $\alpha_0 \in L/K$ et $P_{\alpha_0} \in K$ son polynôme minimal. Alors, $P_{\alpha_0}$ est scindé sur $L$. Soit $K_0 \subseteq L$ sont corps de décomposition. Comme $K \subsetneq K_0 \subseteq L$, $[L : K_0] < n$. Comme $L/K_0$ est normale par le corollaire précédent, par hypothèse de récurrence on dispose de $Q \in K_0[X]$ dont $L$ est le corps de décomposition. Soient $\alpha_1,\dots,\alpha_n \in L$ les racines de $Q$. On a $L = K_0(\alpha_1,\dots,\alpha_n)$, donc $L = K(\alpha_0,\alpha_1\dots,\alpha_n)$. Alors, $L$ est le corps de décomposition du polynôme obtenu comme produit des polynômes minimaux des $\alpha_i$ sur $K$.
\end{proof}

\begin{example} \leavevmode
    \begin{enumerate}
        \item Dans la tour $\Q(\sqrt[3]{2},j)/\Q(\sqrt[3]{2})/\Q$, l'extension $\Q(\sqrt[3]{2},j)/\Q$ est normale mais pas $\Q(\sqrt[3]{2})/\Q$.
        \item Dans la tour $\Q(\sqrt[4]{2})/\Q(\sqrt{2})/\Q$, les extensions $\Q(\sqrt[4]{2})/\Q(\sqrt{2})$ et $\Q(\sqrt{2})/\Q$ sont normales, mais $\Q(\sqrt[4]{2})/\Q$ ne l'est pas.
    \end{enumerate}
\end{example}

\begin{corollary}
    Soit $L/K$ algébrique et $M$ un corps algébriquement clos tel que $M/K$ est algébrique (?? \todo)
\end{corollary}

\begin{proof}
    \todo
\end{proof}

\begin{corollary}
    Soit $M/L/K$ une tour de corps avec $M/K$ normale. Alors, $L/K$ est normale si et seulement si pour tout $K$-automorphisme $\varphi$ de $M$, on a $\varphi(L) = L$.
\end{corollary}

\begin{proof}
    \todo
\end{proof}

\begin{corollary}
    Soit $L/K$ normale. Tout automorphisme de $K$ se prolonge en un automorphisme de $L$. Cela est faux. Je sais pas trop quoi en faire du coup. \todo
\end{corollary}

\begin{proof}
    \todo
\end{proof}

\begin{theorem}
    Soit $L/K$ algébrique et $M$ un corps algébriquement clos tel que $L \subseteq M$. Il existe une plus petite extension normale $E/K$ telle que $L \subseteq E \subseteq M$. On dira que $E$ est la \emph{clôture normale de $L$ dans $M$}.
\end{theorem}

\begin{proof}
    \todo
\end{proof}

\subsection{Polynômes séparables}

\begin{definition}
    Soit $K$ un corps. Soit $\theta$ l'unique morphisme d'anneaux $\Z \to K$ donné par $\theta(1) = 1_K$. Le noyau de $\theta$ est de la forme $n\Z$. Si $\theta$ est injectif, $n = 0$ et on dit que \emph{$K$ est de caractéristique nulle}. Si $\theta$ n'est pas injectif, alors $\ker \theta$ est un idéal premier de $\Z$ car $K$ est un corps, donc $n = p$ est premier et on dit que \emph{$K$ est de caractéristique $p$}. \\
    On note la caractéristique de $K$ par $\car(K)$.
\end{definition}

\begin{proposition}
    Si un corps $K$ est de caractéristique nulle, alors le morphisme $\Z \to K$ se prolonge en un morphisme de corps $\Q \to K$. Si $K$ est de caractéristique $p > 0$, alors le morphisme $\Z \to K$ se factorise en un morphisme de corps $\F_p = \Z/p\Z \hookrightarrow K$ et l'application
    \[\Fr_K : \applic{K}{K}{x}{x^p}\]
    est un automorphisme de $K$ appelé \emph{le morphisme de Frobenius}.
\end{proposition}

\begin{proof}
    On démontre seulement le fait que le Frobenius est un morphisme.
    On a $\Fr_K(1) =1$ et $\Fr_K(xy) = (xy)^p = x^p y^p = \Fr_K(x) \Fr_K(y)$. De plus
    \[(x+y)^p = \sum_{k=0}^p \binom{p}{k} x^k y^{p-k}\]
    Comme $p$ est premier, $p \mid \binom{p}{k}$ pour $0 < k < p$. Comme \og $p = 0$ \fg dans $K$, on obtient
    \[(x+y)^p = y^p + x^p\]
    D'où $\Fr_K$ est bien un automorphisme de $K$ (il est en particulier injectif !).
\end{proof}

\begin{example} \leavevmode
    \begin{enumerate}
        \item La caractéristique de $\Q,\R,\C, \Q(T),\R(T),\C(T)$ est nulle.
        \item La caractéristique de $\F_p,\F_p(T), \F_p(T_1,\dots,T_n)$ est $p$ pour $p$ premier quelconque.
    \end{enumerate}
\end{example}

\begin{definition}
    Pour un polynôme $P = \sum a_i X^i \in K[X]$, on définit son \emph{polynôme dérivé} par $P' = \sum i a_i X^{i-1} \in K[X]$ (où $i$ est en fait $\theta(i)$ où $\theta$ est l'unique morphisme $\Z \to K$).
\end{definition}

\begin{proposition}
    L'application dérivée
    \[\partial : \applic{K[X]}{K[X]}{P}{P'}\]
    est une application $K$-linéaire. C'est l'unique application $K$-linéaire de $K[X]$ dans $K[X]$ qui satisfait la relation de Leibniz $\partial(PQ) = \partial(P) Q + P \partial(Q)$.
\end{proposition}

\begin{proposition}
    Soit $P \in K[X]$ tel que $P' = 0$.
    \begin{itemize}
        \item Si $\car(K) = 0$, alors $\deg P = 0$ ou $P = 0$.
        \item Si $\car(K) = p > 0$, alors il existe un unique $Q \in K[X]$ tel que $P = Q(X^p)$.
    \end{itemize}
\end{proposition}

\begin{proof}
    On explicite $P = \sum a_i X^i$. Supposons $\car (K) = 0$. Alors, $ia_i = 0$ pour tout $i$. Ainsi, $a_i = 0$ pour tout $i > 0$, donc $\deg P = 0$ (ou $P = 0$). Supposons maintenant $\car(K) = p > 0$. Il subsiste que $a_i = 0$ pour tout $i$ non divisible par $p$, donc les seuls monômes non nuls dans $P$ sont de la forme $a_{pk} X^{pk} = (X^p)^k$, ce qui conclut.
\end{proof}

\begin{definition}
    Soit $L/K$ une extension de corps et $P \in K[X]$. Alors, $\alpha \in L$ est dit \emph{une racine de multiplicité $m$ de $P$} si $P(\alpha) = 0$ et $(X-\alpha)^m \mid P$ dans $L[X]$ mais que $(X-\alpha)^{m+1} \nmid P$ dans $L[X]$. Si $m=1$, $\alpha$ est dite \emph{une racine simple de $P$}, sinon on parle de \emph{racine multiple}. 
\end{definition}

\begin{remark}
    Si $P$ est scindé sur $L$, alors $\alpha$ est une racine de multiplicité $m$ de $P$ si et seulement si
    \[P = (X-\alpha)^m c \prod_{i=1}^s (X-\beta_i)^{r_i}\]
    avec $\beta_i \ne \alpha$ pour tout $i$ et $c \in K^\times$.
\end{remark}

\begin{definition}
    Un polynôme $P \in K[X]$ est dit \emph{séparable} si l'idéal $(P,P')$ est $K[X]$, c'est-à-dire si $P$ et $P'$ sont premiers entre eux.
\end{definition}

\begin{example} \leavevmode
    \begin{enumerate}
        \item Si $\deg P = 1$, alors $P$ est séparable.
        \item Dans $\C[X]$, $X^2 + 1$ est séparable.
        \item Dans $\F_p(T)[X]$, $X^p - T$ n'est pas séparable.
    \end{enumerate}
\end{example}

\begin{lemma}
    Un polynôme est séparable si et seulement s'il n'a pas de racines multiples dans son corps de décomposition.
\end{lemma}

\begin{proof}
    Par contraposée, supposons que $P \in K[X]$ ait une racine multiple $\alpha$ dans son corps de décomposition $L/K$. Alors, $(X-\alpha)^2 \mid P$, donc il existe $Q \in L[X]$ tel que $P=(X-\alpha)^2 Q$, d'où
    \[P' = ((X-\alpha)^2 Q)' = 2 (X-\alpha)Q + (X-\alpha)^2 Q'\]
    Donc $\alpha$ est aussi racine de $P'$. Ainsi, en notant $\Pi_\alpha \in K[X]$ le polynôme minimal de $\alpha$ sur $K$ (existe car $L/K$ algébrique), $(P,P') \supset (\Pi_\alpha)$, donc $P$ n'est pas séparable. \\
    Supposons maintenant toutes les racines de $P$ simples dans son corps de décomposition $L/K$. Soit $R \in K[X]$ irréductible tel que $(R) = (P,P')$. Comme $R \mid P$ dans $L[X]$ et que $P$ est scindé dans $L$, $R$ est aussi scindé dans $L[X]$. Une racine que $P$ et $P'$ ont en commun est une racine double de $P$, donc nécessairement $R \in K$ et $(P,P') = K[X]$.
\end{proof}

\begin{remark}
    Le polynôme $P \in K[X]$ est séparable si et seulement si $P$ a exactement $\deg P$ racines dans $\overline{K}$.
\end{remark}

\begin{corollary} \leavevmode
    \begin{enumerate}
        \item Un polynôme $P \in K[X]$ irréductible est séparable si et seulement si $P' \ne 0$.
        \item Si $\car (K) = 0$, tout polynôme irréductible est séparable.
        \item Si $\car(K) = p > 0$, alors $P \in K[X]$ irréductible est séparable si et seulement si $P \notin K[X^p]$.
    \end{enumerate}
\end{corollary}

\begin{proof}
    $(1)$ : l'anneau $K[X]$ est principal, donc $(P,P') = (R)$ pour un $R \in K[X]$, d'où $R \mid P$. Comme $P$ est irréductible, $R = \alpha P$ avec $\alpha \in K^\times$ ou $R \in K^\times$. Si $P$ est séparable, $R \in K^\times$ et $P' \ne 0$ car $(P,0) = (P) \ne K[X]$. Si $P' \ne 0$, comme $R \mid P'$, forcément $R \in K^\times$ (car sinon $\deg R = \deg P > \deg P'$), donc $(P,P') = K[X]$ ie $P$ est séparable. \\
    $(2)$ : en caractéristique $0$, $P' = 0$ implique $P \in K$ ce qui empêche que $P$ soit irréductible, donc par contraposée, tout polynôme irréductible est séparable.
    $(3)$ : en caractéristique $p > 0$, on a $P' = 0$ si et seulement si $P \in K[X^p]$, d'où le résultat.
\end{proof}

\begin{corollary}
    $P \in K[X]$ est séparable si et seulement si c'est le produit de polynômes irréductibles séparables deux à deux distincts.
\end{corollary}

\begin{proof}
    \todo
\end{proof}

Étant donné ce qu'il se passe en caractéristique zéro, on peut se poser la question : sur quels corps les polynôme irréductibles sont-ils tous séparables ?

\begin{definition}
    Un corps $K$ est dit \emph{parfait} si tout polynôme irréductible de $K[X]$ est séparable.
\end{definition}

\begin{proposition}
    Un corps $K$ est parfait si $\car(K) = 0$ ou $\car(K) = p > 0$ et le morphisme de Frobenius $\Fr_K$ est surjectif.
\end{proposition}

\begin{proof}
    Commençons par le sens inverse. On a déjà vu que les corps de caractéristique nulle sont parfaits. Supposons $\car(K) = p>0$ et $\Fr_K$ surjectif. Soit $P \in K[X]$ irréductible. Supposons par l'absurde $P \in K[X^p]$. Comme $\Fr_K$ est surjectif, les coefficients de $P$ sont des puissances $p$-ièmes, donc en fait $P = Q^p$. Ainsi, $P$ ne peut pas être irréductible, ce qui conclut pour le sens inverse. \\
    Supposons maintenant $K$ parfait et $\car(K) = p > 0$. Soit $\alpha \in K \setminus K^p$ un élément non atteint par le morphisme de Frobenius. On considère le polynôme non séparable $X^p - \alpha$. Montrons qu'il est irréductible. Soit $Q \in K[X]$ tel que $Q \mid X^p - \alpha$. Alors, dans un corps de rupture $L/K$ de $X^p -\alpha$, on dispose de $\beta \in L$ tel que $\beta^p = \alpha$. Ainsi, $Q \mid (X-\beta)^p$ dans $L[X]$, donc $Q = (X-\beta)^m$ pour un certain $m$. Si $m$ et $p$ sont premiers entre eux, alors on dispose de $u,v \in \Z$ tels que $um + vp = 1$. Comme $Q \in K[X]$, $Q(0) \in K[X]$, donc $\beta^m \in K$. Comme $\beta^p = \alpha \in K$, on en déduit $\beta^{um+vp} = \beta \in K$. Cela est absurde car $\alpha = \beta^p$ est alors atteint par le morphisme de Frobenius.
\end{proof}

\begin{example} \leavevmode
    \begin{enumerate}
        \item Tout corps fini est parfait (le morphisme de Frobenius est injectif donc surjectif).
        \item Tout corps algébriquement clos est parfait.
        \item $\F_p(T)$ n'est pas parfait.
    \end{enumerate}
\end{example}

\subsection{Extensions séparables}

\begin{definition}
    Une extension de corps $L/K$ algébrique est dite \emph{séparable} si pour tout $\alpha \in L$, le polynôme minimal de $\alpha$ sur $K$ est séparable. Le \emph{degré séparable} d'une extension algébrique $M/K$ est défini par
    \[[M : K]_s = \Card(\{\sigma : M \hookrightarrow \overline{K} \mid \sigma \text{ prolonge } K \hookrightarrow \overline{K}\})\]
    où $\overline{K}$ est une clôture algébrique de $K$.
\end{definition}

\begin{remark}
    Le degré séparable est bien défini : par le théorème de Steinitz, $[M : K]_s$ ne dépend pas de la clôture algébrique $\overline{K}/K$. De plus, on sait que $[M : K]_s > 0$.
\end{remark}

\begin{example} \leavevmode
    \begin{enumerate}
        \item $\C/\R$ est séparable et $[\C : \R]_s = 2$.
        \item Soient $p$ premier, $L = \F_p (T)$ et $K = L^p = \F_p(T^p)$. Alors, $[L : K] = p$. Si $\sigma : L \to \overline{K}$ est un prolongement de $i : K \hookrightarrow \overline{K}$, alors $\sigma(T) = \sigma(T^p)^{1/p} = i(T^p)^{1/p}$, d'où $\sigma$ est uniquement déterminé par $i$, car $X^p - i(T^p) = (X-\sigma(T))^p$ dans $\overline{K}[X]$. Donc, $[L:K]_s = 1 < p = [L:K]$.
    \end{enumerate}
\end{example}

\begin{remark} \leavevmode
    \begin{enumerate}
        \item Soient $\sigma_1 : K \hookrightarrow \overline{K_1}$ et $\sigma_2 : K \hookrightarrow \overline{K_2}$ deux clôtures algébriques de $K$. Par la proposition \ref{prolongmorcac}, comme $\overline{K_1}/K$ est algébrique, il existe $\varphi : \overline{K_1} \hookrightarrow \overline{K_2}$ qui prolonge $\sigma_2$. Soit $\alpha \in \overline{K_2}$ et $P_\alpha \in K[X]$ son polynôme minimal sur $K$. En enlevant les identifications, on a $(\sigma_2 P_\alpha) (\alpha) = 0$, où $(\sigma_2 P_\alpha) \in \overline{K_2} [X]$. Alors, $(X-\alpha) \mid (\sigma_2 P_\alpha)$ dans $\overline{K_2}[X]$. Comme $\overline{K_1}$ est algébriquement clos, $\varphi(\overline{K_1})$ l'est aussi, donc $\sigma_2 P_\alpha = (\varphi \circ \sigma_1) P_\alpha \in \varphi(\overline{K_1})[X]$ est scindé dans $\varphi(\overline{K_1})[X] \subseteq \overline{K_2}$, d'où $X-\alpha \in \varphi(\overline{K_1})[X]$ et $\alpha \in \varphi(\overline{K_1})$. Ainsi, $\varphi$ est surjectif donc un $K$-isomorphisme.
        \item Soit $L/K$ une extension de corps algébrique. Soient $\sigma_1 : K \to \overline{K_1}$ et $\sigma_2 : K \to \overline{K_2}$ deux clôtures algébriques de $K$. Alors, en prenant $\varphi$ comme ci-dessus un $K$-isomorphisme entre $\overline{K_1}$ et $\overline{K_2}$, l'application
        \[\applic{\{\pi : L \to \overline{K_1} \mid \pi_{\mid K} = \sigma_1\}}{\{\psi : L \to \overline{K_2} \mid \psi_{\mid K} = \sigma_2\}}{\pi}{\varphi \circ \pi}\]
        est une bijection, d'où $[L : K]_s$ ne dépend pas de la clôture algébrique de $K$.
        \item Si $L/K$ n'est pas algébrique, alors $[L : K]_s = 0$. Sinon, $[L : K]_s > 0$ par la proposition \ref{prolongmorcac}.
    \end{enumerate}
\end{remark}

\begin{lemma}
    Soit $M/L/K$ une tour de corps algébrique. Alors,
    \[[M : K]_s = [M : L]_s [L : K]_s\]
\end{lemma}

\begin{proof}
    Soit $\overline{K}/K$ une clôture algébrique de $K$. Soit $(\sigma_i)_{i\in I}$ la famille des $K$-plongements $L \hookrightarrow \overline{K}$. Par définition, $[L : K]_s = \Card(I)$. Pour $i\in I$ fixé, $\sigma_i : L \hookrightarrow \overline{K}$ est une clôture algébrique de $L$. Ainsi, par la remarque précédente, il existe $[M : L]_s$ morphismes $M \hookrightarrow \overline{K}$ qui fixent $L$. Notons-les $(\varphi_j^i)_{j \in J}$ avec $[M : L]_s = \Card(J)$. Alors, les prolongements de $K \hookrightarrow \overline{K}$ à $M \hookrightarrow \overline{K}$ sont en bijection avec $(\varphi_j^i)_{(i,j) \in I \times J}$, d'où la conclusion.
\end{proof}

\begin{theorem}
    Soit $L/K$ une extension finie.
    \begin{enumerate}
        \item S'il existe $S \subseteq L$ fini et contenant uniquement des éléments séparables dans $L/K$, alors $L/K$ est une extension séparable. 
        \item On a $[L : K]_s \le [L : K]$ avec égalité si et seulement si $L/K$ est séparable.
    \end{enumerate}
\end{theorem}

\begin{proof}
    Si $L = K(\alpha)$ avec $P_\alpha \in K[X]$ le polynôme minimal de $\alpha$ sur $K$, alors $[L : K]_s$ est le nombre de racines distinctes de $P_\alpha$ dans $\overline{K}$ une clôture algébrique de $K$. Il y en a au plus $[L : K]$ par le corollaire \ref{nbmorcorpsrupture}, d'où $[L : K]_s \le [L : K]$ avec égalité si et seulement si $P_\alpha$ a exactement $\deg P_\alpha$ racines dans $\overline{K}$ c'est-à-dire si $P_\alpha$ est séparable.\\
    Soit maintenant $L = K(S)$. Comme $L/K$ est finie, on peut supposer $S$ fini quitte à passer à un sous-ensemble. Écrivons $S = \{\alpha_1,\dots,\alpha_n\}$. En appliquant la multiplication du degré et du degré séparable à la tour de corps
    \[L/K(\alpha_1,\dots,\alpha_n)/\cdots/K(\alpha_1)/K\]
    on obtient $[L : K]_s \le [L : K]$ par le paragraphe précédent. \\
    Si on suppose que $S$ ne contient que des éléments séparables dans $L/K$, alors pour tout $i$, $\alpha_i$ est séparable dans $K(\alpha_1,\dots,\alpha_i)/K(\alpha_1,\dots,\alpha_{i-1})$, car le polynôme minimal de $\alpha_i$ sur un surcorps de $K$ divise celui sur $K$, qui n'a que des racines simples dans $\overline{K}$. Ainsi, $[L : K]_s = [L : K]$ dans ce cas par le premier paragraphe. \\
    Réciproquement, si $[L : K] = [L : K]_s$, alors pour tout $\alpha \in L$, on a $[L : K(\alpha)]_s = [L : K(\alpha)]$ et $[K(\alpha) : K]_s = [K(\alpha) : K]$ d'où $\alpha$ est séparable dans $L/K$. Ainsi, $L/K$ est séparable.
\end{proof}

\begin{remark}
    Si $K = \Q$ et $L = \Q(\sqrt{p} \mid p \text{ premier})$, alors $[L : \Q]$ est infini dénombrable, mais $\{\sigma : L \to \Qbar \mid \sigma\text{ $K$-morphisme}\}$ est de cardinal $2^{\aleph_0}$ (il faut choisir $\sqrt{p}$ ou $-\sqrt{p}$ pour tout $p$), donc $[L : K]_s$ est infini non dénombrable.
\end{remark}

\begin{proposition}
    Soit $M/L/K$ une tour de corps. Si $\alpha \in M$ est séparable dans $M/L$ et que $L/K$ est séparable, alors $\alpha$ est séparable dans $M/K$. Par conséquence, $M/K$ est séparable si et seulement si $M/L$ et $L/K$ le sont.
\end{proposition}

\begin{proof}
    Soit $P_\alpha \in L[X]$ le polynôme minimal de $\alpha$ sur $L$. Soit $K'/K$ la sous-extension de $L/K$ engendrée par les coefficients de $P_\alpha$. Alors, $K'/K$ est engendrée par un nombre fini d'éléments  algébriques séparable, donc est finie, donc séparable, par le théorème précédent. Ainsi, $[K' : K]_s = [K' : K]$. Comme $P_\alpha \in K'[X]$, $P_\alpha$ est le polynôme minimal de $\alpha$ dans $M/K'$, donc $\alpha$ est séparable dans $M/K'$. Cela implique par le théorème précédent que $K'(\alpha)/K'$ est séparable, donc $[K'(\alpha) : K']_s = [K'(\alpha) : K']$. On obtient $[K'(\alpha) : K] = [K'(\alpha) : K]_s$, donc $K'(\alpha)/K$ est séparable, d'où $\alpha$ est séparable dans $M/K$. \\
    Ainsi, si $M/L$ et $L/K$ sont séparables, $M/K$ l'est aussi. Réciproquement, supposons $M/K$ séparable. Alors, $L/K$ est séparable. De plus, si $\alpha \in M$ et $P_\alpha^K, P_\alpha^L$ sont ses polynômes minimaux sur $K$ et $L$, alors $P_\alpha^L \mid P_\alpha^K$, donc si $P_\alpha^K$ est séparable, $P_\alpha^L$ aussi, d'où $M/L$ est séparable.
\end{proof}

\begin{corollary}
    L'ensemble des éléments séparables d'une extension $L/K$ est une sous-extension séparable sur $K$. On l'appelle la \emph{clôture séparable de $K$ dans $L$}.
\end{corollary}

\begin{proof}
    Pour $\alpha,\beta \in L$ séparables dans $L/K$, $K(\alpha,\beta)/K$ est séparable, donc $\alpha - \beta, \alpha \beta^{-1}$ aussi.
\end{proof}

\begin{remark} \leavevmode
    \begin{enumerate}
        \item Si $L/K$ est algébrique et $L = K(S)$ avec $S\subseteq L$ contenant uniquement des éléments séparables, alors $L/K$ est séparable.
        \item On peut définir \emph{\og la \fg clôture séparable} d'un corps $K$ comme étant la clôture séparable de $K$ dans $\overline{K}$. On la note $\overline{K}^\sep$ ou $\overline{K}^s$. Elles sont toutes $K$-isomorphes.
    \end{enumerate}
\end{remark}

\begin{definition}
    Une extension algébrique $L/K$ est dite \emph{purement inséparable} ou \emph{radicielle} si la clôture séparable de $K$ dans $L$ est $K$.
\end{definition}

\begin{remark} \leavevmode
   \begin{enumerate}
    \item L'extension $\overline{K}/\overline{K}^\sep$ est purement inséparable.
    \item S'il existe une extension $L/K$ purement inséparable avec $[L : K] > 1$ (c'est vrai ça ??? \todo), alors $\car K > 0$. En particulier, si $\car K = 0$, on a $\overline{K} = \overline{K}^\sep$.
   \end{enumerate} 
\end{remark}

\begin{example}
    L'extension $\F_p(T)/\F_p(T^p)$ est purement inséparable.
\end{example}

\begin{proposition} \leavevmode
    \begin{enumerate}
        \item Soit $K$ un corps. Alors, $K$ est parfait si et seulement si $\overline{K} = \overline{K}^\sep$.
        \item Si $L/\overline{K}^\sep$ est séparable, on a $L=\overline{K}^\sep$.
        \item Si $\alpha \in \overline{K} \setminus \overline{K}^\sep$, $\alpha$ n'est pas séparable dans $\overline{K}/\overline{K}^\sep$ et pas séparable dans $\overline{K}/K$.
    \end{enumerate}
\end{proposition}

\begin{proof} \leavevmode
    \begin{enumerate}
        \item $K$ est parfait si et seulement si tous les polynômes irréductibles de $K[X]$ sont séparables si et seulement si tous les polynômes minimaux sur $K$ d'éléments de $\overline{K}$ sont séparables si et seulement si $\overline{K} = \overline{K}^\sep$.
        \item $L/\overline{K}^\sep$ est algébrique donc il existe un plongement $L\hookrightarrow \overline{K}$. Comme $L/K$ est séparable, on a $L \subseteq \overline{K}^\sep$ et donc $L = \overline{K}^\sep$.
        \item Voir (2). (rendre ça plus sympathique au lecteur \todo)
    \end{enumerate}
\end{proof}

\begin{definition}
    Un corps $K$ est dit \emph{séparablement clos} si $K = \overline{K}^\sep$.
\end{definition}

\begin{proposition} \leavevmode
    \begin{enumerate}
        \item Un corps $K$ est séparablement clos si et seulement si tout polynôme irréductible séparable sur $K$ est de degré $1$.
        \item Toute extension séparable d'un corps $K$ se plonge dans $\overline{K}^\sep$. (déjà fait juste au-dessus non ? \todo)
    \end{enumerate}
\end{proposition}

\begin{proof}
    \todo
\end{proof}

\subsection{Théorème de l'élément primitif}

\begin{theorem}
    Soit $K$ un corps et $L = K(\alpha_1,\dots,\alpha_n)$ une extension finie de $K$. Si $\alpha_2,\dots,\alpha_n$ sont séparables dans $L/K$, alors il existe $\beta \in L$ tel que $L = K(\beta)$. On dit dans ce cas que $L/K$ est une \emph{extension simple}.
\end{theorem}

\begin{proof}
    Supposons d'abord $K$ fini. Alors, $L$ est fini comme extension finie de $K$. Alors, $L^\times$ est cyclique \footnote{\todo mettre une ref}, donc $L = K(\alpha)$ pour $\alpha \in L$ un générateur de $L^\times$. \\
    Supposons maintenant $K$ infini. Il suffit de montrer le résultat pour $n=2$ puis de faire une récurrence (c'est ici que le choix de ne pas supposer $\alpha_1$ séparable prend son sens). Soient $P_{\alpha_1}, P_{\alpha_2}$ les polynômes minimaux de $\alpha_1,\alpha_2$ sur $K$, respectivement. Soit $M/L$ un corps de décomposition de $P_{\alpha_1}P_{\alpha_2}$. Alors, dans $M[X]$, on a
    \[P_{\alpha_1} = \prod_{i=1}^s (X-x_i) \quad \text{et} \quad P_{\alpha_2} = \prod_{j=1}^t (X-y_j) \]
    avec $x_1 = \alpha_1$, $y_1 = \alpha_2$ et les $y_j$ distincts deux à deux (car $\alpha_2$ est séparable). On choisit
    \[\gamma \in K \setminus \left\{\frac{x_i-\alpha_i}{\alpha_2-y_j} \; \Big| \; 2 \le j \le t, 1 \le i \le s \right\}\]
    Un tel $\gamma$ existe car $K$ est supposé infini. Soit $\beta = \alpha_1 + \gamma \alpha_2 \in K$. Montrons que $K(\beta) = K(\alpha_1,\alpha_2)$. Par construction, on a $P_{\alpha_1} (\beta - \gamma \alpha_2) = P_{\alpha_1} (\alpha_1) = 0$ et $P_{\alpha_2} (\beta - \gamma y_j) \ne 0$ pour $j \ne 1$. Ainsi, les polynômes $P_{\alpha_2}$ et $R = P_{\alpha_1} (\beta - \gamma X)$ dans $K(\beta)[X]$ ont $\alpha_2 \in L$ pour seule racine commune. Ainsi, $\pgcd(P_{\alpha_2},R) = X-\alpha_2$. Comme le PGCD ne dépend pas du corps ambiant, $X-\alpha_2 \in K(\beta)[X]$ d'où $\alpha_2 \in K(\beta)$, d'où $\alpha_1 = \beta - \gamma\alpha_2 \in K(\beta)$. Ainsi, $K(\beta) = K(\alpha_1,\alpha_2)$.
\end{proof}

\begin{example} \leavevmode
    \begin{enumerate}
        \item $\Q(\sqrt[3]{2},j)/\Q$ est séparable, donc simple. Elle est engendrée par $\sqrt[3]{2} + j$. En général, il n'est pas évident de trouver un $x \in L$ tel que $L = K(x)$.
        \item Soit $L = \F_p (X,Y)$ et $K = L^p = \F_p(X^p,Y^p)$. Alors, $L/K$ n'est pas simple :
        \begin{itemize}
            \item $\{X^i Y^j \mid 0\le i,j\le p-1\}$ est une base de $L/K$, donc $[L : K] = p^2$.
            \item Tout $\alpha \in L$, est racine de $T^p - \alpha^p \in K[T]$ d'où son polynôme minimal sur $K$ divise $T^p - \alpha^p$ donc est degré au plus $p$. Ainsi, $[K(\alpha) : K] \le p$ donc $L \ne K(\alpha)$.
        \end{itemize}
    \end{enumerate}
\end{example}

\begin{theorem}
    Soit $L/K$ une extension finie. Sont équivalents :
    \begin{enumerate}
        \item Il n'y a qu'un nombre fini de corps intermédiaires $E$ entre $K$ et $L$.
        \item Il existe $\alpha \in L$ tel que $L = K(\alpha)$ i.e. $L/K$ est simple.
    \end{enumerate}
\end{theorem}

\begin{proof}
    Dans le cas où $K$ est fini, l'énoncé est clair. Supposons donc $K$ infini. \\
    Supposons $(1)$. Soit $L = K(\alpha_1,\dots,\alpha_n)$. Il suffit de traiter le cas $n=2$ par récurrence. La famille d'extensions donnée par $(K(\alpha_1 + \beta \alpha_2))_{\beta \in K}$ étant finie, on dispose de $u,v \in K$ distincts tels que $K(\alpha_1 + u \alpha_2) = K(\alpha_1 + v \alpha_2)$. Alors, $\alpha_2 = \frac{(\alpha_1+v\alpha_2) - (\alpha_1 + u \alpha_2)}{v-u} \in K(\alpha_1 + u \alpha_2)$, d'où $\alpha_1 \in K(\alpha_1 + u \alpha_2)$ et on obtient $K(\alpha_1,\alpha_2) = K(\alpha_1 + u\alpha_2)$. \\
    Supposons $(2)$. Soit $L = K(\alpha)$ avec $P_\alpha$ le polynôme minimal de $\alpha$ sur $K$. Soit $E$ un corps tel que $K \subseteq E \subseteq L$. Alors, le polynôme minimal $P_{\alpha,E}$ de $\alpha$ sur $E$ satisfait $P_{\alpha, E} \mid P_\alpha$ dans $E[X] \subseteq L[X]$. On remarque que $P_\alpha$ n'a qu'un nombre fini de diviseurs unitaires dans $L[X]$. Soit $E_0$ le corps engendré sur $K$ par les coefficients de $P_{\alpha, E}$. Alors, $E \supseteq E_0 \supseteq K$ et $P_{\alpha, E}$ est irréductible sur $E_0 [X]$, donc c'est le polynôme minimal de $\alpha$ sur $E_0$. On a donc $[L : E_0] = \deg P_{\alpha, E} = [L : E]$ d'où $E = E_0$. Ainsi, on peut retrouver $E$ à partir de $P_{\alpha, E}$. On obtient une injection
    \[\applic{\{\text{corps intermédiaires dans }L/K\}}{\{\text{facteurs irréductibles unitaires de $P_\alpha$ dans $L[X]$}\}}{E}{P_{\alpha,E}}\]
    d'où le résultat.
\end{proof}

\subsection{Corps finis}

Un corps $K$ est dit fini si son cardinal est fini. On remarque qu'alors $\car K > 0$ car $1$ est d'ordre fini dans $(K,+)$. Ainsi, il existe $p$ premier tel que $\car K = p$. Soit $\F_p = \Z/p\Z$. Alors, $\F_p$ est un sous-corps de $K$. On a donc $\Card(K) = p^{[K : \F_p]}$ où $[K : \F_p] < \infty$ car $K$ est fini et une extension de $\F_p$.

\begin{proposition}
    Pour tout $p$ premier et tout $n \in \N^*$, il existe un corps $K$ à $p^n$ éléments. Ce corps est un corps de décomposition du polynôme $X^{p^n} - X \in \F_p [X]$, d'où tous deux corps à $p^n$ éléments sont isomorphes.
\end{proposition}

\begin{proof}
    Soit $K$ un corps de décomposition de $X^{p^n} - X \in \F_p [X]$. Comme ce polynôme est séparable, ses racines dans $K$ sont simples et sont distinctes donc il y en a $p^n$. De plus, le sous-ensemble $K_0 = \{ x\in K \mid x^{p^n} = x\}$ est un sous-corps de $K$, car
    \[(xy)^{p^n} = x^{p^n} y^{p^n} \quad (x+y)^{p^n} = x^{p^n}+y^{p^n}\]
    comme on travaille en caractéristique $p$. Ainsi, $K_0$ est un corps de décomposition de $X^{p^n} - X$ lui-même. Comme deux corps de décomposition d'un même polynôme sont isomorphes, $K_0 \simeq K$. Le corps $K_0$ est fini comme ensemble des racines d'un polynôme, donc $K = K_0$ et $\Card(K) = p^n$. \\
    Maintenant, si $K$ est n'importe quel corps à $p^n$ éléments, on sait que $K^\times$ est cyclique d'ordre $p^n - 1$. Ainsi, le théorème de Lagrange assure que si $x \in K^\times$, $x^{p^n - 1} = 1$. En particulier, pour tout $x \in K$, $x^{p^n} = x$. Comme $\car K = p$ et que $\F_p \hookrightarrow K$, $K$ est en fait un corps de décomposition de $X^{p^n} - X \in \F_p[X]$, car il est minimal pour la propriété de posséder toutes les racines de $X^{p^n} - X$.
\end{proof}

Dans la suite, on fait l'abus de notation de noter $\F_q$ un corps à $q$ éléments. On a $q=p^n$ pour $p$ premier et $n\in \N^*$.

\begin{proposition}
    Soit $p$ premier et $n,m \in \N^*$. Alors, il existe un morphisme $\F_{p^m} \hookrightarrow \F_{p^n}$ si et seulement si $m \mid n$.
\end{proposition}

\begin{proof}
    S'il existe un morphisme de corps $\F_{p^m} \hookrightarrow \F_{p^n}$, alors en notant $d = [\F_{p^n} : \F_{p^m}]$, on a $\Card(\F_{p^n}) = p^n = (\Card(\F_{p^m}))^d = (p^m)^d = p^{md}$ d'où $m \mid n$. \\
    Réciproquement, supposons $n = dm$ pour $d \in \N^*$. Alors, $p^m - 1 \mid p^n - 1$. Écrivons $p^n - 1 = (p^m - 1)A$ pour $A \in \N^*$. On remarque que $\F_{p^n}$ est un corps de décomposition de $X^{p^n - 1} - 1 \in \F_p [X]$. Comme
    \[X^{p^n - 1} - 1 = X^{A(p^m - 1)} - 1 = ((X^{p^m - 1}) - 1)Q\]
    avec $Q \in \F_p [X]$ et $X^{p^n - 1} - 1$ est séparable, on déduit que $X^{p^m - 1} - 1$ est scindé dans $\F_{p^n}$. Ainsi, par le théorème \ref{thmmorpolscinde}, il existe un morphisme $\F_{p^m} \hookrightarrow \F_{p^n}$.
\end{proof}

\begin{proposition}
    Soit $p$ premier. Soit $\overline{\F_p}$ une clôture algébrique de $\F_p$. Alors, pour tout $n \in \N^*$, $\overline{\F_p}$ contient exactement un corps à $p^n$ éléments.
\end{proposition}

\begin{proof}
    Comme $\F_{p^n}$ est le corps de décomposition de $X^{p^n} - X \in \F_p [X]$, la clôture algébrique $\overline{\F_p}$ n'en contient qu'une seule copie.
\end{proof}

\begin{remark}
    On a en fait $\overline{\F_p} = \bigcup\limits_{n\ge 1} \F_{p^n}$. Montrons l'inclusion $\subseteq$. Soit $x\in \overline{\F_p}$. Alors, $\F_p (x) / \F_p$ est algébrique et de type finie donc finie d'où $\Card(\F_p) < \infty$ avec $\car \F_p (x) = p$ d'où $\F_p (x) \simeq \F_{p^n}$.
\end{remark}

\begin{example}
    Une extension algébrique infinie de $\F_p$ qui n'est pas sa clôture algébrique est $\bigcup\limits_{n\ge 1} \F_{p^{2n}}$.
\end{example}

\subsection{Théorie de Galois}



\end{document}