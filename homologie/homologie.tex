\documentclass{article}
\usepackage{amsmath, amssymb, amsfonts, amsthm}
\usepackage[english]{babel}
\usepackage{geometry}[textwidth=10cm]
\usepackage{tikz-cd}
\usepackage{tikz}
\usepackage{fontawesome5}
\usepackage{nicefrac}
\usetikzlibrary{positioning}
\usetikzlibrary{decorations.pathmorphing}

\newcommand{\id}{\mathrm{id}}
\newcommand{\op}{\mathrm{op}}
\newcommand{\N}{\mathbb{N}}
\newcommand{\Z}{\mathbb{Z}}
\newcommand{\Q}{\mathbb{Q}}
\newcommand{\R}{\mathbb{R}}
\newcommand{\C}{\mathbb{C}}
\newcommand{\K}{\mathbb{K}}
\newcommand{\U}{\mathbb{U}}
\newcommand{\F}{\mathbb{F}}
\renewcommand{\H}{\mathbb{H}}

\renewcommand{\P}{\mathbb{P}}
\renewcommand{\d}{{\rm \, d}}

\newcommand{\asympt}{\mathop{\sim}}
\newcommand{\lin}{\mathcal{L}}
\newcommand{\M}{\mathcal{M}}
\newcommand{\cat}{\mathcal{C}}
\newcommand{\catt}{\mathcal{D}}
\newcommand{\rel}{\mathcal{R}}
\newcommand{\Jcat}{\mathcal{J}}

\newcommand{\Set}{\mathbf{Set}}
\newcommand{\Top}{\mathbf{Top}}
\newcommand{\Grp}{\mathbf{Grp}}
\newcommand{\Mod}{\mathbf{Mod}}
\newcommand{\Ab}{\mathbf{Ab}}
\newcommand{\Vectcat}{\mathbf{Vect}}
\newcommand{\Rep}{\mathbf{Rep}}
\newcommand{\Ring}{\mathbf{Ring}}

%Operators
\DeclareMathOperator{\Acc}{Acc}
\DeclareMathOperator{\arccosh}{arccosh}
\DeclareMathOperator{\Card}{Card}
\DeclareMathOperator{\GL}{GL}
\DeclareMathOperator{\SL}{SL}
\DeclareMathOperator{\PSL}{PSL}
\DeclareMathOperator{\PGL}{PGL}
\DeclareMathOperator{\SO}{SO}
\DeclareMathOperator{\pgcd}{pgcd}
\DeclareMathOperator{\Tr}{Tr}
\DeclareMathOperator{\rg}{rg}
\DeclareMathOperator{\ch}{ch}
\DeclareMathOperator{\sh}{sh}
\DeclareMathOperator{\Vect}{Vect}
\DeclareMathOperator{\Gal}{Gal}
\DeclareMathOperator{\rk}{rk}
\DeclareMathOperator{\Fix}{Fix}
\DeclareMathOperator{\Ob}{Ob}
\DeclareMathOperator{\Aut}{Aut}
\DeclareMathOperator{\Int}{Int}
\DeclareMathOperator{\Mor}{Mor}
\DeclareMathOperator{\Hom}{Hom}
\DeclareMathOperator{\Fun}{Fun}
\DeclareMathOperator{\Nat}{Nat}
\DeclareMathOperator{\colim}{colim}
\renewcommand{\th}{\mathop{\rm th}}
\renewcommand{\Re}{\mathop{\rm Re}}
\renewcommand{\Im}{\mathop{\rm Im}}

\newcommand{\todo}{\textbf{TODO}}

\renewcommand{\epsilon}{\varepsilon}
\newcommand{\invlim}{\varprojlim}
\newcommand{\dirlim}{\varinjlim}

\newcommand{\esh}{{\textstyle \int}}

%Macros
\newcommand{\applic}[4]{\begin{array}[t]{rcl}
#1 & \to & #2 \\
#3 & \mapsto & #4
\end{array}}
\newcommand{\warning}{\faExclamationTriangle \hspace{3pt}}

\setlength{\parindent}{0pt}

\theoremstyle{plain}
\newtheorem{theorem}{Theorem}[section]
\newtheorem{proposition}[theorem]{Proposition}
\newtheorem*{corollary}{Corollary}

\theoremstyle{definition}
\newtheorem{definition}[theorem]{Definition}
\newtheorem{example}[theorem]{Example}
\newtheorem{examples}[theorem]{Examples}


\theoremstyle{remark}
\newtheorem*{remark}{Remark}
\newtheorem*{exercise}{Exercise}

\title{Homological Algebra}
\author{Original lectures notes by Baptiste Rognerud, but now typeset in \LaTeX}
\date{}

\begin{document}
\maketitle

\section{Introduction to category theory}

References:
\begin{itemize}
    \item Emily Riehl, Category Theory in Context (chapter I)
    \item Saunders Mac Lane, Categories for the Working Mathematician
    \item Ibrahim Assem, Introduction au langage catégorique (chapters I, II)
\end{itemize}

\faChevronRight \space Near 1945 Eilenberg and Mac Lane gave the good formalism for a ``natural isomorphism'' (the general theory of natural transformations). For instance, if $V$ is a finite-dimensional vector space, $V \simeq V^*$ and $V \simeq V^{**}$, but the first isomorphism is not natural (``a choice needs to be made''), while the second is. But why? \\
It turns out solving this question gave a formalism, category theory, that unified already existing mathematical concepts, gave new links between different notions and also gave new questions! \\
\warning Category theory is not a theory that trivializes mathematics. \\
It is used today by (almost) everyone: algebraic geometry, algebra, representation theory, topology, combinatorics, \dots

\subsection{Categories and functors}

\begin{definition}
    A \emph{category} $\cat$ is the data of
    \begin{itemize}
        \item A collection of \emph{morphisms} $\Mor(\cat)$
        \item A collection of \emph{objects} $\Ob(\cat)$
    \end{itemize}
    such that
    \begin{enumerate}
        \item Every morphism $f \in \Mor(\cat)$ has a specified domain $X \in \Ob(\cat)$ and codomain $Y \in \Ob(\cat)$. We write $f : X \to Y$.
        \item For every object $X \in \Ob(\cat)$ there exists a morphism $1_X : X \to X$ (the \emph{identity} of $X$), also written $\id_X$
        \item For any three objects $X,Y,Z \in \Ob(\cat)$ and morphism $f : X \to Y$ and $g : Y \to Z$ there exists a morphism $g\circ f : X \to Z$ (we often omit $\circ$ and just write $gf$)
    \end{enumerate}
    satisfying
    \begin{description}
        \item[(Identity)] $\forall f : X \to Y, 1_Y f = f = f 1_X$
        \item[(Associativity)] $\forall f : W \to X, g : X \to Y, h : Y \to Z$, $h(gf) = (hg)f$
    \end{description}
\end{definition}

\begin{remark} \leavevmode
    \begin{enumerate}
        \item We use the term ``collection'' because we don't want to worry about set-theoretical issues
        \item If $\Mor(\cat)$ is a set, we say that $\cat$ is \emph{small}
        \item We denote by $\Hom_\cat (X,Y)$ (or $\cat(X,Y)$) the collection of $f : X \to Y \in \Mor(\cat)$
    \end{enumerate}
\end{remark}

\begin{examples}[Concrete categories] \leavevmode
    \begin{enumerate}
        \item The category $\Set$, where objects are sets and morphisms are just maps.
        \item $\Top$, where objects are topological spaces and morphisms are continuous maps.
        \item Groups together with group homomorphisms form a category called $\Grp$. The same can be said about rings, fields\dots
        \item $k$-vector spaces, or more generally left/right $R$-modules, together with linear maps, form a category denoted $R\Mod$ or $\Mod R$ (for left or right $R$-modules).
    \end{enumerate}
\end{examples}

In these previous examples, objects are sets with additional structure, and morphisms between two objects are in particular maps between the two underlying sets. Such categories are called \emph{concrete categories} (a rigorous definition will be given later). However, a category need not be concrete.

\begin{examples}[Abstract categories] \leavevmode
    \begin{enumerate}
        \item Let $k$ be a field. There exists a category $\mathbf{Mat}_k$ where objects are the natural numbers $\N$ and morphisms are $\Hom(m,n) = \mathrm{Mat}_{n,m} (k)$, where composition is given by matrix multiplication.
        \item If $G$ is a group, there exists a category $BG$ which has only one object $\bullet$, and morphisms $\Hom(\bullet, \bullet) = G$, where composition is multiplication in $G$.
        \item If $(P,\le)$ is a \emph{poset} (a partially ordered set, that is a set $P$ together with a reflexive, transitive relation $\le$), then one can construct a category $\hat{P}$ by setting $\Ob(\hat{P}) = P$ and $|\Hom(x,y)| = \begin{cases}
            1 & \text{if $x\le y$} \\ 0 & \text{otherwise}
        \end{cases}$, where composition is defined in the only possible way.
        \item The homotopy category of topological spaces: objects are topological spaces, and $\Hom(X,Y)$ is $\Hom_\Top(X,Y)/\sim$ where $\sim$ is homotopy of continuous maps.
    \end{enumerate}
\end{examples}

\begin{exercise}
    Check the categories defined above really are categories. In (2), what are the minimal hypotheses needed on $G$ for $BG$ to be a category? In (3), what are the minimal hypotheses needed on $\le$ for $\hat{P}$ to be a category?
\end{exercise}

\begin{examples}[Categories constructed from categories] \leavevmode
    \begin{enumerate}
        \item If $\cat$ is a category, one can construct its \emph{opposite category} $\cat^\op$, defined by $\Ob(\cat^\op) = \Ob(\cat)$ and $\Hom_{\cat^\op}(X,Y) = \Hom_\cat(Y,X)$, with composition described by the following diagram:
        \begin{center}
            \begin{tikzcd}
                X \arrow[d, "f"] \arrow[dd, "gf", bend right, '] \arrow[dd,phantom,bend left,""{name=U}] & X \\
                Y \arrow[d, "g"] & Y \arrow[u,"f^\text{op}"] \\
                Z & Z \arrow[u, "g^\text{op}"] \arrow[uu, "f^\text{op}g^\text{op}", ', bend right] \arrow[uu,bend left,""{name=V},phantom]
                \arrow[from=U,to=V,squiggly]
            \end{tikzcd}
        \end{center}
        \item Let $\cat$ be a category. A \emph{subcategory} $\catt$ of $\cat$ is another category such that $\Ob(\catt) \subset \Ob(\cat)$ and $\Mor(\catt) \subset \Mor(\cat)$ and the composition in $\catt$ is induced by the one in $\cat$. For instance, $\Ab$, the category of abelian groups and group homomorphisms, is a subcategory of $\Grp$.
        \item Let $\cat$ and $\catt$ be categories. The \emph{product category} of $\cat$ and $\catt$ is the category $\cat \times \catt$ defined by $\Ob(\cat \times \catt) = \Ob(\cat) \times \Ob(\catt)$ and $\Mor(\cat \times \catt) = \Mor(\cat) \times \Mor(\catt)$, composition and identities being defined componentwise.
    \end{enumerate}
\end{examples}

\begin{exercise}
    Describe $(BG)^\op$ for $G$ a group and $\hat{P}^\op$ for $(P,\le)$ a poset.
\end{exercise}

\warning $\Set^\op$ is not $\Set$. \todo

\begin{remark}
    In a category $\cat$ the objects can be anything, so saying $x \in X$ for $X \in \Ob(\cat)$ doesn't make sense. Hence, categorical notions are defined using arrows and not elements.
\end{remark}

\begin{definition}
    Let $\cat$ be a category.
    \begin{enumerate}
        \item $f : X \to Y$ is an \emph{isomorphism} if there exists $g : Y \to X$ such that $gf = \id_X$ and $fg = \id_Y$.
        \item $f : X \to Y$ is a \emph{monomorphism} if for all $g,h : W \to X$ such that $fg = fh$, $g=h$ ($f$ is left-cancellable).
        \item $f : X \to Y$ is an \emph{epimorphism} if for all $g,h : Y \to Z$ such that $gf = hf$, $g=h$ ($f$ is right-cancellable).
    \end{enumerate}
\end{definition}

\warning Being both a mono and an epi doesn't imply being an iso. \todo

\begin{definition}
    Let $\cat,\catt$ be two categories. A \emph{(covariant) functor} $F : \cat \to \catt$ is the data of
    \begin{itemize}
        \item An object $F(X) \in \Ob(\catt)$ for all $X \in \Ob(\cat)$
        \item A morphism $F(f) \in \Hom_\catt(F(X), F(Y))$ for all $f \in \Hom_\cat (X,Y)$
    \end{itemize}
    such that $F(\id_X) = \id_{F(X)}$ for all $X \in \Ob(\cat)$ and $F(gf) = F(g)F(f)$ whenever $f,g \in \Mor(\cat)$ are composable. 
\end{definition}

\begin{definition}
    A \emph{contravariant} functor from $\cat$ to $\catt$ is a functor from $\cat^\op$ to $\catt$ (so composition is reversed, i.e. $F(gf) = F(f)F(g)$).
\end{definition}

\begin{examples} \leavevmode
    \begin{enumerate}
        \item $U : \Grp \to \Set, U(G) = G, U(f) = f$ the functor that to a group assigns it its underlying set and to a homomorphism the underlying map. It is called the \emph{forgetful functor} from groups to sets, because it forgets the group structure.
        \item $U : \applic{\mathbf{Ass}}{\mathbf{Lie}}{(A,+,\cdot)}{(A,+,[-,-])}$ the forgetful functor from the category of associative algebras to the category of Lie algebras. It forgets the ``associative structure'' but remembers the underlying abelian group.
        \item $F : \Set \to \Ab, X \mapsto \Z[X], f \mapsto \Z[f]$, which to a set assigns the free abelian group with basis $X$ (the group of finite linear combinations of elements of $X$). A map $f : X \to Y$ can then be uniquely extended to a linear map $\Z[f] : \Z[X] \to \Z[Y]$ that agrees with $f$ on the bases of $\Z[X]$ and $\Z[Y]$.
        \item Suppose $\cat$ is locally small (i.e. for any $X,Y$, $\Hom_\cat(X,Y)$ is a set). For all $X \in \cat$, $\Hom(X,-)$ is a functor $\cat \to \Set$. Similarly, $\Hom_\cat(-,X)$ is a contravariant functor $\cat \to \Set$. $\Hom_\cat(-,-)$ is a functor $\cat \times \cat^\op \to \Set$.
        \item Functors $\cat \xrightarrow[F]{} \catt \xrightarrow[G]{} \mathcal{E}$ can be composed in the obvious sense.
    \end{enumerate}
\end{examples}

\todo : DRAW DIAGRAMS

\begin{definition}
    Let \begin{tikzcd}
        \cat \arrow[r,bend left, "F"] \arrow[r,bend right, "G",'] & \catt
    \end{tikzcd} be two functors. A \emph{natural transformation} $\eta$ from $F$ to $G$ is the data of morphisms $\eta_X : F(X) \to G(X) \in \Mor(\catt)$ for all $X \in \Ob(\cat)$ such that for all $f : X \to Y \in \Mor(\cat)$, the diagram
    \begin{center}
        \begin{tikzcd}
            F(X) \arrow[r, "\eta_X"] \arrow[d, "F(f)"] & G(X) \arrow[d, "G(f)"] \\
            F(Y) \arrow[r, "\eta_Y"] & G(Y)
        \end{tikzcd}
    \end{center}
    commutes, that is $G(f)\eta_X = \eta_Y F(f)$. We write $\eta : F \Rightarrow G$ or draw \begin{tikzcd}
        \cat \arrow[r,bend left, "F"] \arrow[r,bend left, phantom, ""{name=U}] \arrow[r,bend right, "G"{name=V},'] \arrow[r,bend right, phantom, ""{name=V},'] & \catt \arrow[Rightarrow, from=U,to=V,"\eta"]
    \end{tikzcd}
\end{definition}

\begin{example}
    Let $V$ be a $k$-vector space. $\id_{\mathbf{Vect}_k}$ and $D^2 = \Hom_{\mathbf{Vect}_k}(\Hom_{\mathbf{Vect}_k}(-,k),k)$ are two endofunctors of $\mathbf{Vect}_k$. $\mathrm{ev}_- : \applic{V}{V^{**}}{v}{\applic{\Hom(V,k)}{k}{\phi}{\phi(v)}}$ defines a natural transformation between them:
    \begin{center}
        \begin{tikzcd}
            V \arrow[r, "\mathrm{ev}"] \arrow[d, "f",'] & V^{**} \arrow[d, "D^2(f)"] \\
            W \arrow[r, "\mathrm{ev}",'] & W^{**}
        \end{tikzcd}
    \end{center}
    For $a \in V$, $D^2 (f) \circ \mathrm{ev}_a : \applic{W^*}{k}{\phi}{\phi(f(a))} \in W^{**}$ and in the other direction $(\mathrm{ev} \circ f)(a) = \mathrm{ev}_{f(a)}$. \\
    However, there is no natural transformation from $\id_{\Vectcat_k}$ to $D$. For one, the first is covariant and the second is contravariant. To get a natural transformation from a covariant to a contravariant functor, we can modify the definition of naturality to be that \begin{tikzcd}[column sep=tiny, row sep = small]
        V \arrow[r] \arrow[d] & V^* \\ W \arrow[r] & W^* \arrow[u]
    \end{tikzcd}
    commutes, but even such natural transformations do not exist from $\id_{\Vectcat_k}$ to $D$.
\end{example}

\begin{definition}
    A natural transformation \begin{tikzcd}
        \cat \arrow[r,bend left, "F"] \arrow[r,bend left, phantom, ""{name=U}] \arrow[r,bend right, "G"{name=V},'] \arrow[r,bend right, phantom, ""{name=V},'] & \catt \arrow[Rightarrow, from=U,to=V,"\eta"]
    \end{tikzcd} is a \emph{natural isomorphism} if $\eta_X$ is an isomorphism for all $X \in \Ob(\cat)$.
\end{definition}

\begin{remark}
    One can compose natural transformations in two ways, ``vertical composition'':
    \begin{center}
        \begin{tikzcd}[row sep=huge]
            \mathcal{C}
             \arrow[r, bend left=65, "F"{name=F}]
             \arrow[r, "G"{inner sep=0,fill=white,anchor=center,name=G}]
             \arrow[r, bend right=65, "H"{name=H, swap}]
             \arrow[from=F.south-|G,to=G,Rightarrow,shorten=2pt,"\alpha"] 
             \arrow[from=G,to=H.north-|G,Rightarrow,shorten=2pt,"\beta"] &
           \mathcal{D} \arrow[r,squiggly] &
           \mathcal{C} \arrow[r, bend left=30,"F"{name=U}] \arrow[r,bend right=30, "H"{name=D},'] \arrow[from=U.south,to=D.north,Rightarrow, "\beta \circ \alpha",shorten=2pt] &
           \mathcal{D}
        \end{tikzcd}
        where $(\beta\circ \alpha)_X = \beta_X \circ \alpha_X$
    \end{center}
    or ``horizontal composition'':
    \begin{center}
        \begin{tikzcd}[row sep=huge]
            \mathcal{C}
             \arrow[r, bend left=30, "F_1"{name=F}]
             \arrow[r, bend right=30, "G_1"{name=H, swap}]
             \arrow[from=F.south-|H,to=H.north,Rightarrow,shorten=2pt,"\alpha_1"] &
           \mathcal{D}
           \arrow[r, bend left=30, "F_2"{name=FF}]
             \arrow[r, bend right=30, "G_2"{name=HH, swap}]
             \arrow[from=FF.south-|HH,to=HH.north,Rightarrow,shorten=2pt,"\alpha_2"] &
            \mathcal{E} \arrow[r,squiggly] &
            \mathcal{C} \arrow[rr, bend left=30,"F_2\circ F_1"{name=U}] \arrow[rr,bend right=30, "G_2\circ G_1"{name=D},'] \arrow[from=U.south,to=D.north,Rightarrow, "\alpha_2 * \alpha_1",shorten=2pt] & &
           \mathcal{E}
        \end{tikzcd}
        where $(\alpha_2 * \alpha_1)_X = G_2({(\alpha_1)}_{X}) \circ {(\alpha_2)}_{F_1(X)} $
    \end{center}
    Horizontal composition can also be defined in another equivalent way using commutativity of
    \begin{center}
        \begin{tikzcd}
            F_2 F_1(X) \arrow[r,"(\alpha_2)_{F_1(X)}"] \arrow[d,"F_2((\alpha_1)_X)",'] & G_2 F_1(X) \arrow[d,"G_2((\alpha_1)_X)"]  \\
            F_2 G_1(X) \arrow[r,"(\alpha_2)_{G_1(X)}"] & G_2 G_1 (X)
        \end{tikzcd}
    \end{center}
    The diagram commutes by naturality of $\alpha_2$, so $(\alpha_2 * \alpha_1) = (\alpha_2)_{G_1(X)} \circ F_2((\alpha_1)_X)$.
\end{remark}

\begin{definition}
    Let $\cat,\catt$ be two categories. Then the \emph{functor category from $\cat$ to $\catt$} written $\Fun(\cat,\catt)$ or $\catt^\cat$ is the category whose objects are functors from $\cat$ to $\catt$ and morphisms are natural transformations.
\end{definition}

\begin{remark}
    Categories, together with functors and natural transformations between them is the prototypal example of a 2-category.
\end{remark}

\subsection{Equivalences of categories}

\begin{definition}
    Let $\cat$ and $\catt$ be two categories. An \emph{equivalence of categories} from $\cat$ to $\catt$ is the data of
    \begin{enumerate}
        \item $F : \cat \to \catt$ and $G : \catt \to \cat$ wo functors
        \item Natural isomorphisms $\eta : 1_\cat \Rightarrow GF$ and $\epsilon : FG \Rightarrow 1_\catt$ where $1_\cat$ and $1_\catt$ are the identity functors of $\cat$ and $\catt$.
    \end{enumerate}
\end{definition}

\begin{remark} \leavevmode
    \begin{enumerate}
        \item $G$ is called a \emph{quasi-inverse} of $F$.
        \item Most of the time we say that $F$ is an equivalence if there exists $G$ such that $(F,G)$ is an equivalence.
        \item If $F,G$ are contravariant, we speak of \emph{duality} between $\cat$ and $\catt$.
        \item If two categories are equivalent, every property that can be expressed ``in terms of arrows'' is preserved.
    \end{enumerate}
\end{remark}

\begin{definition}
    Let $F : \cat \to \catt$ be a functor. Then, we say
    \begin{enumerate}
        \item $F$ is \emph{faithful} if $\forall X,Y \in \Ob(\cat)$, $F : \applic{\Hom_\cat(X,Y)}{\Hom_\catt(F(X),F(Y))}{f}{F(f)}$ is injective.
        \item $F$ is \emph{full} if the previous map is surjective.
        \item $F$ is \emph{essentially surjective} if for all $ Y \in \Ob(\catt)$ there is $X \in \Ob(\cat)$ such that $F(X) \simeq Y$ in $\catt$.
    \end{enumerate}
\end{definition}

\begin{theorem}
    Let $F : \cat \to \catt$ be a functor. Then $F$ is an equivalence of categories if and only if it is fully faithful and essentially surjective. 
\end{theorem}

\begin{proof}
    \warning There is a little set-theoretic issue: an equivalence of categories is always fully faithful and essentially surjective, but the converse requires the axiom of choice for the class $\Ob(\cat)$. \\
    Suppose $F : \cat \to \catt$ is an equivalence of categories, and let $G : \catt \to \cat$ be a quasi-inverse of $F$, together with natural isomorphisms $\eta : 1_\cat \to GF$ and $\epsilon : 1_\catt \to FG$. If $Y$ is an object of $\catt$, then $Y \simeq FG(Y)$, so $F$ is essentially surjective. Let $X,Y$ be objects of $\cat$. To show $F$ is fully faithful we will construct an inverse to $F : \Hom_\cat (X,Y) \to \Hom_\catt (F(X),F(Y))$. For any $f \in \Hom_\catt(F(X),F(Y))$, we have a commutative diagram
    \begin{center}
        \begin{tikzcd}[cramped]
            X \arrow[r, "\eta_X"] \arrow[d, "f",'] & GF(X) \arrow[d, "GF(f)"] \\
            Y \arrow[r, "\eta_Y"] & GF(Y)
        \end{tikzcd}
    \end{center}
    which prompts us to define $\phi : \applic{\Hom_\catt (F(X),F(Y))}{\Hom_\cat (X,Y)}{g}{\eta_Y^{-1} \circ G(g) \circ \eta_X}$. We now check it is the map we're looking for. If $f : X \to Y$, since the above diagram commutes and $\eta_Y$ is invertible, we get that $\phi(F(f)) = f$, so $\phi \circ F = \id_{\Hom_\cat (X,Y)}$, which means $F$ is faithful. We have two commutative diagrams, by definition of $\phi$ and by naturality of $\eta$:
    \begin{center}
        \begin{tikzcd}[cramped]
            X \arrow[r, "\eta_X"] \arrow[d, "\phi(g)",'] & GF(X) \arrow[d, "G(g)"] \\
            Y \arrow[r, "\eta_Y"] & GF(Y)
        \end{tikzcd}
        \hspace{1.5cm}
        \begin{tikzcd}[cramped]
            X \arrow[r, "\eta_X"] \arrow[d, "\phi(g)",'] & GF(X) \arrow[d, "GF(\phi(g))"] \\
            Y \arrow[r, "\eta_Y"] & GF(Y)
        \end{tikzcd}
    \end{center}
    therefore, $G(g) \circ \eta_X = GF(\phi(g)) \circ \eta_X$.
    Since $\eta_X$ is invertible, $G(g) = GF(\phi(g))$. The previous point shows that $G$ is faithful, so $g = F(\phi(g))$, hence $F$ is full. \\
    Now suppose $F$ is fully faithful and essentially surjective. Our goal is to construct $G$. For any $Y \in \Ob(\catt)$, since $F$ is essentially surjective, there exists $X_Y \in \Ob(\cat)$ and an isomorphism $\epsilon_Y : Y \to F(X_Y)$. Therefore, for any $Y,Z\in \Ob(\catt)$ and $f:Y \to Z$, we have a commutative diagram
    \begin{center}
        \begin{tikzcd}
            Y \arrow[d, "\epsilon_Y"] \arrow[r, "f"] & Z \arrow[d, "\epsilon_Z"] \\
            F(X_Y) \arrow[r, "\epsilon_Z \circ f \circ \epsilon_Y^{-1}",'] & F(X_Z)
        \end{tikzcd}
    \end{center}
    Which leads us to define $G(Y) = X_Y$ and $G(f)$ to be the unique morphism $m_f : X_Y \to X_Z$ such that $F(m_f) = \epsilon_Z \circ f \circ \epsilon_Y^{-1}$ (this works because $F$ is fully faithful). We have $G(\id_Y) = \id_{X_Y}$ since $\varepsilon_Y \circ \id_Y \circ \varepsilon_Y^{-1} = \id_Y$ and $F(\id_{X_Y}) = \id_Y$. The next diagram shows $G(g\circ f) = G(g)\circ G(f)$:
    \begin{center}
        \begin{tikzcd}
            W \arrow[d, "\epsilon_W"] \arrow[r,"f"] \arrow[rr, bend left, "g\circ f"] & Y \arrow[d, "\epsilon_Y"] \arrow[r, "g"] & Z \arrow[d, "\epsilon_Z"] \\
            F(X_W) \arrow[r,"F(m_f)",'] \arrow[rr, bend right,',"F(m_g \circ m_f) = F(m_g) \circ F(m_f)"] & F(X_Y) \arrow[r, "F(m_g)",'] & F(X_Z)
        \end{tikzcd}
    \end{center}
    By this construction, $\epsilon$ is a natural isomorphism $\id_\catt \Rightarrow FG$ (look at the above diagrams). Now, pick $Y,Z\in \Ob(\cat)$ and $f : Y \to Z$. We have $GF(Y) = X_{F(Y)}$ and $\varepsilon_Y : F(Y) \xrightarrow{\sim} F(X_{F(Y)})$. Since $F$ is fully faithful, there exists a unique $\eta_Y : Y \to X_{F(Y)} = GF(Y)$ such that $F(\eta_Y) = \varepsilon_Y$. Here, $\eta_Y = G(\epsilon_Y)$, which means that $\eta_Y$ is an isomorphism since functors preserve isomorphisms. We obtain the diagram
    \begin{center}
        \begin{tikzcd}
            Y \arrow[d,"f"] \arrow[r, "\eta_Y"] & GF(Y) \arrow[d,"GF(f)"] \\
            Z \arrow[r,"\eta_Z"] & GF(Z)
        \end{tikzcd}
    \end{center}
    The diagram commutes because $GF(f)$ is the unique morphism such that
    \[F(GF(f)) = \varepsilon_Z \circ F(f) \circ \varepsilon_Y^{-1} = F(\eta_Z \circ f \circ \eta_Y^{-1})\] and $F$ is faithful. $\eta$ is then a natural isomorphism $\id_\cat \Rightarrow GF$.
\end{proof}

\begin{example}
    $\Vectcat_k \simeq \mathbf{Mat}_k$ through the functor $n \mapsto k^n$ and $(A : n \to m) \mapsto (f_A : k^n \to k^m)$.
\end{example}
\newpage
\section{Universal properties}
References:
\begin{itemize}
    \item Riehl (Chapters II, III, IV)
    \item Mac Lane (Chapters III, IV, V)
    \item Assem (Chapters III, IV, V)
\end{itemize}

\faChevronRight \space Let $S$ be a set together with an equivalence relation $\sim$. Let $\nicefrac{S}{\sim}$ be the quotient set, and $\pi : S \to \nicefrac{S}{\sim}$ be the projection. For any $f : S \to X$ compatible with $\sim$, there exists a unique map $\bar{f} : \nicefrac{S}{\sim} \to X$ such that $f = \bar{f} \circ \pi$. This is represented by the following commutative diagram : \begin{center}\begin{tikzcd}
    S \arrow[d,"\pi",'] \arrow[r,"f"] & X \\ \nicefrac{S}{\sim} \arrow[ur, "\exists!\bar{f}", dashed,']
\end{tikzcd}\end{center}
We say that $S \xrightarrow{\pi} \nicefrac{S}{\sim}$ is a solution to the universal problem posed by the compatible maps. Such a solution is unique up to unique isomorphism: if $S \xrightarrow{p} S'$ is another solution, then we get the three commutative diagrams
\begin{center}
    \begin{tikzcd}
    S \arrow[d,"\pi",'] \arrow[r,"p"] & S' \\ \nicefrac{S}{\sim} \arrow[ur, "\exists!a", dashed,']
    \end{tikzcd}
    \begin{tikzcd}
        S \arrow[d,"p",'] \arrow[r,"\pi"] & \nicefrac{S}{\sim} \\ S' \arrow[ur, "\exists!b", dashed,']
    \end{tikzcd}
    \begin{tikzcd}
        S \arrow[d,"p",'] \arrow[r,"p"] & S' \\
        S' \arrow[r, "b",'] \arrow[ur, "\id_{S'}",'] & \nicefrac{S}{\sim} \arrow[u, "a",']
    \end{tikzcd}
\end{center}
then $abp = a\pi = p$. The identity of $S'$ also makes this diagram commute so by uniqueness $ab = \id_{S'}$ and similarly $ba = \id_{\nicefrac{S}{\sim}}$.

\subsection{Initial and final objects}
\begin{definition}
    Let $\cat$ be a category. An object $c \in \Ob(\cat)$ is \emph{initial} (\emph{final}) if for all $d \in \Ob(\cat)$ there exists a unique morphism $c \to d$ (a unique morphism $d \to c$).
\end{definition}
\begin{proposition}
    If an initial/final object exists, then it is unique up to unique isomorphism.
\end{proposition}
\begin{proof}
    Let $c,c'$ be two initial objects. Then there exists a unique morphism $f : c \to c'$ and a unique morphism $g : c' \to c$. There also exists a unique morphism $c \to c$, that is $\id_c$. Therefore, $gf = \id_c$. In the same way, $fg = \id_{c'}$. Therefore, $c$ and $c'$ are isomorphic and the isomorphism is unique.
\end{proof}
\begin{examples} \leavevmode
    \begin{enumerate}
        \item $\varnothing$ is initial in $\Set$ and any singleton is final.
        \item $\{0\}$ is both initial and final in $\Vectcat_k$ (or $R\Mod$).
        \item The category of fields does not have initial/final objects (reason on field characteristics).
    \end{enumerate}
\end{examples}

We want to say a universal object is an initial or final object. A category has at most 2, so this may seem a little restrictive, but this is solved by thinking of a good category.
\begin{definition}
    Let $F : \cat \to \Set$ be a functor. Let $\esh F$ be the category defined by
    \begin{align*}
        & \Ob(\esh F) = \{(c,x) \mid c \in \Ob(\cat) \text{ and } x \in F(c)\} \\
        & \Hom((c,x),(c',x')) = \{f \in \Hom(c,c') \mid F(f)(x) = x'\}
    \end{align*}
    where composition is composition in $\cat$, and $\id_{(c,x)} = \id_c$ for all $x$. If $F$ is contravariant, let $\esh F$ have the same objects and morphisms $\Hom((c,x),(c',x')) = \{f \in \Hom(c,c') \mid F(f)(x') = x\}$.
\end{definition}

\begin{proposition}
    There is a forgetful functor $\pi : \esh F \to \cat$ defined by $\pi(c,x) = c$ and $\pi(f : (c,x) \to (c',x')) = f : c\to c'$.
\end{proposition}

\begin{example}
    Let $S$ be a set, and $\sim$ an equivalence relation on $S$. Let $F : \Set \to \Set$ be defined by $F(X) = \{f : S \to X \mid x \sim y \Rightarrow f(x) = f(y)\}$ and $F(\alpha : X \to Y) = \alpha \circ -$. \\
    $\esh F$ has for objects $(X, S\xrightarrow{f}X)$ where $f$ is compatible with $\sim$, and for morphisms $\alpha$ that makes this diagram commute: \begin{tikzcd}
        S \arrow["f'", r] \arrow["f",d,'] & X' \\ X \arrow[ur, "\alpha",']
    \end{tikzcd} \\
    $(\nicefrac{S}{\sim}, S\xrightarrow{\pi} \nicefrac{S}{\sim})$ is an initial object of $\esh F$.
\end{example}

\begin{definition}
    Let $F : \cat \to \Set$ be a functor. A \emph{universal element} for $F$ is an initial object of $\esh F$, that is a pair $(c,x)$ with $c \in \Ob(\cat)$ and $x \in F(c)$ such that
    \[\forall (d,y), d \in \Ob(\cat), y \in F(d), \exists! \alpha : c\to d, y= F(\alpha)(x)\]
\end{definition}

\begin{definition}
    Let $F : \cat \to \catt$ be a functor and $d \in \Ob(\catt)$. A \emph{universal arrow from $d$ to $F$} is a pair $(c,f)$ where $c \in \Ob(\cat)$ and $f \in \Hom_{\catt} (d,F(c))$, such that
    \[\forall (c',f'), c' \in \Ob(\cat), f' : d \to F(c'), \exists! \alpha \in \Hom_\cat(c,c'), F(\alpha) \circ f = f'\]
    \begin{center}
        \begin{tikzcd}[column sep=tiny]
            & d \arrow[ld, "f",'] \arrow[rd, "\forall f'"] \\
            F(c) \arrow[rr, dashed, "F(\alpha)",'] & & F(c') \\
            c \arrow[rr, "\exists ! \alpha"] & & c'
        \end{tikzcd}
    \end{center}
\end{definition}

\begin{exercise}
    Define a category $d \downarrow F$ such that a universal arrow is an initial object of $d \downarrow F$.
\end{exercise}

\begin{example}
    Let $U : \Vectcat_k \to \Set$ be the forgetful functor. Let $X \in \Set$. A universal arrow from $X$ to $U$ is the ``best'' $k$-vector space $V_X$ with a map $X \to V_X$. Set $V_X = k[X]$ the $k$-vector space with basis $X$, and $i : X \to V_X$ that maps $x \in X$ to the corresponding basis element. Then, for any vector space $V$ and map $f : X \to U(V)$, $f$ can be extended by linearity into a linear map $\tilde{f} : k[X] \to V$, which makes this diagram commute:
    \begin{center}
        \begin{tikzcd}[column sep=tiny]
            & X \arrow[ld, "i",'] \arrow[rd, "f"] \\
           k[X] \arrow[rr, dashed, "\tilde{f}",'] & & U(V)
        \end{tikzcd}
    \end{center}
    If $\alpha$ is another map that makes the diagram commute then $\alpha$ and $\tilde{f}$ coincide on a basis of $k[X]$ and therefore are equal.
\end{example}

\begin{proposition}
    Universal elements and arrows are two equivalent notions.
\end{proposition}

\begin{proof}
    Let $F : \cat \to \Set$ be a functor and $(c,x)$ a universal element  for $F$. Consider $f_x : \applic{\{*\}}{F(c)}{*}{x}$. Then, $(c,f_x)$ is a universal arrow $*\to F$, because $F(\alpha)(x)=y$ iff $F(\alpha)\circ f_x = f_y$.
    \begin{center}
        \begin{tikzcd}[column sep=tiny]
            & \{*\} \arrow[ld, "f_x",'] \arrow[rd, "f_y"] \\
           F(c) \arrow[rr, dashed, "F(\alpha)",'] & & F(c')
        \end{tikzcd}
    \end{center}
    Conversely, if $F : \cat \to \catt$ is a functor and $(c,f)$ is a universal arrow $d \to F$, then consider the functor $\Hom_\catt (d,F(-)) : \applic{\cat}{\Set}{x}{\Hom_\catt (d,F(x))}$ (we need to assume $\catt$ is locally small so the functor is set-valued). Then, $f \in \Hom_\catt (d,F(c))$ is a universal element for this functor.
\end{proof}

\subsection{Representable functors}

\begin{definition}
    Let $\cat$ be a (locally small) category, and $F : \cat \to \Set$ a functor.
    \begin{enumerate}
        \item We say that $F$ is \emph{representable} if there is some $c \in \Ob(\cat)$ such that $F$ and $\Hom_\cat (c,-)$ are naturally isomorphic (if $F$ is contravariant, use $\Hom_\cat(-,c)$ instead).
        \item A \emph{representation} of $F$ is the data of $c \in \Ob(\cat)$ and a natural isomorphism $\eta : \Hom(c,-) \Rightarrow F$.
    \end{enumerate}
\end{definition}

\begin{example}
    The forgetful functor $U : \Grp \to \Set$ is representable since $\Hom_\Grp (\Z,-) \simeq U$. The natural isomorphism is given by $\alpha \in \Hom_\Grp (\Z,G) \mapsto \alpha(1) \in G$.
\end{example}

The following theorem explains how to find the natural isomorphism $\alpha : \Hom_\cat (c,-) \Rightarrow F$ in general.

\begin{theorem}[Yoneda lemma]
    Let $F : \cat \to \Set$ be a functor with $\cat$ locally small, and $c \in \Ob(\cat)$. Then,
    \[\begin{array}[t]{rcl}
        \Nat(\Hom(c,-),F) & \xrightarrow{\sim} & F(c)\\
        \alpha & \mapsto & \alpha_c(\id_c)
        \end{array}\]
    and this isomorphism is natural in $c$ and in $F$.
\end{theorem}

\begin{proof}
    Let $\alpha \in \Nat(\Hom(c,-),F)$. Let $d \in \cat$ and $f : c \to d$. By naturality, we have a commutative diagram
    \begin{center}
        \begin{tikzcd}
            \Hom(c,c) \arrow[r,"\alpha_c"] \arrow[d,"f\circ -"] & F(c) \arrow[d,"F(f)"] \\
            \Hom(c,d) \arrow[r,"\alpha_d"] & F(d)
        \end{tikzcd}
    \end{center}
    This means that $F(f) \circ \alpha_c = \alpha_d \circ (f\circ -)$. Evaluating at $\id_c$, we get $F(f) \circ \alpha_c(\id_c) = \alpha_d(f)$. This shows that the natural transformation $\alpha$ is entirely determined by the value of $\alpha_c(\id_c)$, which shows the map defined above is injective. Conversely, if $e \in F(c)$, then we define $\alpha^e : \Hom(c,-) \Rightarrow F$ by $\alpha_d^e : g \mapsto F(g)(e)$. We check it is a natural transformation:
    \begin{center}
        \begin{tikzcd}[column sep=large]
            \Hom(c,c) \arrow[r,"g \mapsto F(g)(e)"] \arrow[d,"f\circ -"] & F(c) \arrow[d,"F(f)"] \\
            \Hom(c,d) \arrow[r,"h \mapsto F(h)(e)"] & F(d)
        \end{tikzcd}
    \end{center}
    and this diagram commutes since for $g : c\to c$ we have
    \[F(f)(F(g)(e)) = F(f \circ g)(e) = F((f\circ -)(g))(e)\]
    This shows the map in the theorem is surjective, therefore an isomorphism, and its inverse is given by $e \in F(c) \mapsto \alpha^e$.
    We now check naturality. We first need to understand what it means to say the isomorphism is natural in $c$. Let $f : c \to d$. $\Nat(\Hom(c,-),F)$ is functorial in $c$, as it is the composition of two contravariant hom-functors. More concretely:
    \[c \xrightarrow{f} d \rightsquigarrow \Hom(d,-) \xrightarrow{- \circ f}\Hom(c,-) \rightsquigarrow \Nat(\Hom(c,-),F) \xrightarrow{- \circ (- \circ f)} \Nat(\Hom(d,-),F)\]
    ($\Nat$ is the hom-functor of the functor category $\cat^\Set$). One thing to note is that the morphism $f : c\to d$ induces a natural transformation $\Hom(d,-) \xrightarrow{-\circ f} \Hom(c,-)$, and this makes the whole thing work. This is in general a property of functors defined on a product category, see the remark below. Now, saying the isomorphism, which we'll write $\Phi_{d,F}$, is natural means that the square
    \begin{center}
        \begin{tikzcd}
            \Nat(\Hom(c,-),F) \arrow[r,"\Phi_{c,F}"] \arrow[d,"-\circ(-\circ f)"] & F(c) \arrow[d,"F(f)"]\\
            \Nat(\Hom(d,-),F) \arrow[r,"\Phi_{d,F}"] & F(d)
        \end{tikzcd}
    \end{center}
    commutes. And indeed, if $\alpha : \Hom(c,-)\Rightarrow F$ is a natural transformation,
    \begin{align*}
        \Phi_{d,F}(\alpha \circ (- \circ f)) & = (\alpha \circ (-\circ f))_d (\id_d) = [\alpha_d \circ (- \circ f)] (\id_d) = \alpha_d (f) \\
        F(f)(\Phi_{c,F}(\alpha)) & = F(f)(\alpha_c(\id_c)) = \alpha_d(f\circ \id_c) = \alpha_d(f)
    \end{align*}
    The second to last equality comes from the naturality of $\alpha$. \\
    We now turn to naturality in $F$. Let $G$ be another functor $\cat \to \Set$ and $\beta : F \Rightarrow G$ be a natural transformation. We check that
    \begin{center}
        \begin{tikzcd}
            \Nat(\Hom(c,-),F) \arrow[r,"\Phi_{c,F}"] \arrow[d,"\beta \circ -"] & F(c) \arrow[d, "\beta_c"] \\
            \Nat(\Hom(c,-),G) \arrow[r,"\Phi_{c,G}"] & G(c)
        \end{tikzcd}
    \end{center}
    commutes. For $\alpha : \Hom(c,-) \Rightarrow F$, we have
    \[\beta_c (\Phi_{c,F}(\alpha)) = \beta_c(\alpha_c(\id_c)) = (\beta \circ \alpha)_c (\id_c) = \Phi_{c,G}(\beta \circ \alpha)\]
    which completes the proof of naturality.
\end{proof}
\newpage

\begin{remark} \leavevmode
    \begin{enumerate}
        \item If $F : \cat \to \Set$, then $(c,x)$ is a universal element for $F$ if and only if the natural transformation $\alpha_x : \Hom(c,-) \Rightarrow F$ induced by $x$ is an isomorphism. Indeed, $\alpha_x$ is an isomorphism iff $\forall c' \in \cat$, $(\alpha_x)_{c'} : \applic{\Hom(c,c')}{F(c')}{f}{F(f)(x)}$ is bijective iff
        \[\forall c' \in \cat, \forall y \in F(c'), \exists ! f : c \to c', F(f)(x) = y\]
        \item For universal arrows, use $\Hom_\catt (d,F(-))$ as before.
        \item Let $\cat,\catt$ and $\mathcal{E}$ be categories, and $F : \cat \times \catt \to \mathcal{E}$ be a functor. Let $c,d \in \Ob(\cat)$, $x,y \in \Ob(\catt)$ and morphisms $f : c \to d$, $g : x\to y$. The morphism $f$ induces a natural transformation $F(f,\id_-) : F(c,-) \Rightarrow F(d,-)$, see the commutative square:
        \begin{center}
            \begin{tikzcd}[column sep=large]
                F(c,x) \ar{r}{F(f,\id_x)} \ar{d}{F(\id_c,g)} & F(d,x) \ar{d}{F(\id_d,g)} \\
                F(c,y) \ar{r}[']{F(f,\id_y)} & F(d,y)
            \end{tikzcd}
        \end{center}
    \end{enumerate}
\end{remark}

\subsection{Examples of objects defined by universal properties}

\subsubsection{Products, coproducts}
Let $\cat$ be a small category and $X,Y \in \Ob(\cat)$. We define a category $\cat_{X,Y}$ whose objects are tuples $(Z,f,g)$ where $Z \in \Ob(\cat)$ and $f : Z \to X$, $g : Z \to Y$ and morphisms are maps $\alpha : Z \to Z'$ in $\cat$ such that the following diagram commutes:
\begin{center}
    \begin{tikzcd}[column sep=tiny, row sep=small]
        & Z \arrow[dl,"f",'] \arrow[dd,"\alpha"] \arrow[dr,"g"] \\
        X & & Y \\
        & Z' \arrow[ul,"f'"] \arrow[ur,"g'",']
    \end{tikzcd}
\end{center}

\begin{definition}
    A \emph{product} of $X$ and $Y$ is a final object in $\cat_{X,Y}$. Concretely, it is an object $X \times Y$ together with two maps $\pi_X : X \times Y \to X$ and $\pi_Y : X \times Y \to Y$ such that for any $(Z,f,g) \in \Ob(\cat_{X,Y})$, we have a commutative diagram
    \begin{center}
        \begin{tikzcd}[column sep=scriptsize]
            & Z \arrow[dl,"\forall f",'] \arrow[dr,"\forall g"] \arrow[d,"\exists!\alpha", dashed] \\
            X & X \times Y \arrow[l,"\pi_X"] \arrow[r,"\pi_Y",'] & Y
        \end{tikzcd}
    \end{center}
\end{definition}

Since it is defined as being a final object, if it exists, a product is unique up to unique isomorphism.

\begin{examples}
    In $\Set$, the product of $X$ and $Y$ is the cartesian product. In $\Grp$, it is the product group. In $\Top$, it is the cartesian product equipped with the product topology. In these examples, the maps in the definition are the canonical projections.
\end{examples}

The notion dual to the one of a product is called a coproduct.

\begin{definition}
    A \emph{coproduct} of $X$ and $Y$ is a product in $\cat^\op$. Concretely, it satisfies the universal property expressed by this commutative diagram:
    \begin{center}
        \begin{tikzcd}
            X \arrow[r,"i_X"] \arrow[dr, "\forall f",'] & X \sqcup Y \arrow[d,"\exists!\alpha",dashed] & Y \arrow[dl,"\forall g"] \arrow[l,"i_Y",'] \\
            & Z
        \end{tikzcd}
    \end{center}
\end{definition}

\begin{examples}
    In $\Set$, the coproduct of $X$ and $Y$ is the disjoint union together with canonical inclusion. In $\Top$, the coproduct of $X$ and $Y$ is their disjoint union equipped with the disjoint union topology. However, in $\Grp$, the underlying set of the coproduct of two groups is not the disjoint union.
\end{examples}

\subsubsection{Equalizers and coequalizers}

\begin{definition}
    Let $\cat$ be a category and $X,Y \in \Ob(\cat)$, $f,g : X \to Y$. Consider the contravariant functor $F : \cat \to \Set$ defined by $F(c) = \{\alpha : c \to X \mid f\alpha = g \alpha\}$ and $F(\beta) = - \circ \beta$. An \emph{equalizer} in $\cat$ is a representation of the contravariant functor $F$.
\end{definition}

By the Yoneda lemma, a natural transformation $\Hom(-,c) \Rightarrow F$ is the same as an element of $F(c)$, so a representation of $F$ is a pair $(c,e)$ with $c \in \Ob(\cat)$ and $e \in F(c)$ such that the natural transformation given by the Yoneda lemma is an isomorphism. Concretely, we want $\eta_e : \applic{\Hom(d,c)}{F(d)}{h}{F(h)(e)}$ to be an isomorphism for all $d \in \Ob(c)$. This translates into the follwing diagram:
\begin{center}
    \begin{tikzcd}
        & d \arrow[dl, "\exists!\alpha",dashed,'] \arrow[d,"\forall h"] \\
        c \arrow[r,"e"] & X \arrow[r,"f",shift left] \arrow[r,"g",shift right,'] & Y
    \end{tikzcd}
\end{center}

\begin{example}
    In $\Set$, $E = \{x \in X \mid f(x) = g(x)\} \hookrightarrow X$ is an equalizer.
\end{example}

The dual notion is that of a coequalizer.

\begin{definition}
    A \emph{coequalizer} of \begin{tikzcd}
        X \arrow[r,"f",shift left] \arrow[r,"g",shift right,'] & Y
    \end{tikzcd} is an object $Z \in \Ob(\cat)$ together with a morphism $\pi : Y\to Z$ such that $\pi f = \pi g$ and that universal to this property:
    \begin{center}
        \begin{tikzcd}
            X \arrow[r,"f",shift left] \arrow[r,"g",shift right,'] & Y \arrow[r, "\pi"] \arrow[d, "\forall h"] & Z \arrow[dl, dashed, "\exists! \alpha"] \\
            & Z'
        \end{tikzcd}
    \end{center}
\end{definition}

\begin{example}
    In $\Set$, consider the equivalence relation $\sim$ on $Y$ generated by $f(x)\sim g(x)$ (the smallest equivalence relation on $Y$ with this property). Then $y \xrightarrow{\pi} \nicefrac{Y}{\sim}$ is a coequalizer.
\end{example}

\subsection{Adjoint functors}

This notion was introduced by Kan in 1958.

\begin{definition}
    An \emph{adjunction} $(G,D)$ is a pair of functors $G : \cat \to \catt$ and $D : \catt \to \cat$ together with an isomorphism $\Hom_\catt (G(c),d) \simeq \Hom_\cat (c,D(d))$ which is natural in both $c$ and $d$. \\
    We write $G \dashv D$ and say $G$ is left adjoint to $D$ and $D$ is right adjoint to $G$.
\end{definition}

If $G \dashv D$ we have $\forall c,d \in \Ob(\cat)\times \Ob(\catt)$,
\[\Hom_\catt (G(c),d) \xrightarrow[\alpha_{c,d}]{\sim} \Hom_\cat (c,D(d))\]
and in particular when $d=G(c)$ we get $\Hom_\catt (G(c),G(c)) \xrightarrow[\alpha_{c,G(c)}]{\sim} \Hom_\cat (c,DG(c))$. \\
Let $\eta_c : c \to DG(c)$ be the image of $\id_{G(c)}$. This gives a collection of morphisms $- \to DG(-)$.
\begin{proposition}
    These morphisms make up a natural transformation $\id_\cat \Rightarrow DG$.
\end{proposition}
\begin{proof}
    Let $f : c\to d$. We want to show that
    \begin{center}
        \begin{tikzcd}[column sep=large]
            c \arrow[d,"f"] \arrow[rr,"\eta_c = \alpha_{c,G(c)} (\id_{G(c)})"] & & DG(c) \arrow[d,"DG(f)"] \\
            d \arrow[rr, "\eta_d = \alpha_{d,G(d)} (\id_{G(d)})"] & & DG(d)
        \end{tikzcd}
    \end{center}
    commutes. By naturality of the isomorphism $\alpha$ given by the adjunction, we get the following commutative diagram
    \begin{center}
        \begin{tikzcd}
            \Hom(G(c),G(c)) \arrow[d, "G(f) \circ -",'] \arrow[r,"\sim","\alpha_{c,G(c)}"'] & \Hom(c,DG(c)) \arrow[d,"DG(f)\circ -",']\\  
            \Hom(G(c),G(d)) \arrow[r,"\sim","\alpha_{c,G(d)}"'] & \Hom(c,DG(d)) \\
            \Hom(G(d),G(d)) \arrow[u, "- \circ G(f)"] \arrow[r,"\sim","\alpha_{d,G(d)}"'] & \Hom(d,DG(d)) \arrow[u,"- \circ f"]
        \end{tikzcd}
    \end{center}
    which gives us these equations:
    \begin{align*}
        DG(f) \circ \eta_c & = DG(f) \circ \alpha_{c,G(c)}(\id_{G(c)}) = \alpha_{c,G(d)}(G(f)\circ \id_{G(c)}) = \alpha_{c,G(d)} (G(f)) \\
        \eta_d \circ f & = \alpha_{d,G(d)}(\id_{G(d)}) \circ f = \alpha_{c,G(d)}(\id_{G(c)} \circ G(f)) = \alpha_{c,G(d)} (G(f))
    \end{align*}
    which completes the proof.
\end{proof}

We also get a natural transformation $\epsilon : GD \Rightarrow \id_\catt$ when $c=D(d)$ by setting $\epsilon_d = \alpha_{D(d),d}^{-1}(\id_{D(d)})$.

\begin{definition}
    The natural transformation $\eta : \id_\cat \Rightarrow DG$ is called the \emph{unit} of the adjunction. The natural transformation $\epsilon : GD \Rightarrow \id_\catt$ is called its $\emph{counit}$.
\end{definition}

\begin{proposition}
    Let \begin{tikzcd}
        \cat \arrow[r, bend left, "G"] & \catt \arrow[l,bend left,"D"]
    \end{tikzcd} be two functors. Then, $G \dashv D$ if and only if there are natural transformations $\eta : \id_\cat \Rightarrow DG$ and $\epsilon : GD \Rightarrow \id_\catt$ such that the following diagrams commute:
    \begin{center}
        \begin{tikzcd}
            G \arrow[r, "G\eta", Rightarrow] \arrow[dr, "\id_G", equal,'] & GDG \arrow[d, "\epsilon G", Rightarrow] & & D \arrow[r,"\eta D",Rightarrow] \arrow[dr, "\id_D",equal,'] & DGD \arrow[d, "D \epsilon", Rightarrow] \\
            & G & & & D
        \end{tikzcd}
    \end{center}
    where $G\eta$ is the natural transformation given by the morphisms $G(\eta_c)$ and $\epsilon G$ is the one give by morphisms $\epsilon_{G(c)}$ (and similarly for $\eta D$ and $D\epsilon$).
\end{proposition}

\begin{proof}
    Suppose $G \dashv D$. Let $\eta : \id_\cat \Rightarrow DG$ and $\epsilon : GD \Rightarrow \id_\catt$ be the unit and counit of the adjunction. Let $c \in \cat$. We have
    \[(\epsilon G)_c \circ (G\eta)_c = \varepsilon_{G(c)} \circ G(\eta_c) = \alpha_{DG(c),G(c)}^{-1} (\id_{DG(c)}) \circ G(\alpha_{c,G(c)}(\id_{G(c)}))\]
    and the naturality of $\alpha$ gives the following commutative diagram
    \begin{center}
        \begin{tikzcd}[column sep=huge]
            \Hom(G(c),G(c))
            &
            \Hom(c,DG(c))
                \arrow[l,"\sim"',"\alpha_{c,G(c)}^{-1}"]
            \\

            \Hom(GDG(c), G(c))
                \arrow[u, "- \circ G(\alpha_{c,G(c)}(\id_{G(c)}))"]
            &
            \Hom(DG(c),DG(c))
                \arrow[u,"- \circ \alpha_{c,G(c)}(\id_{G(c)})"']
                \arrow[l,"\sim"',"\alpha_{DG(c),G(c)}^{-1}"]
        \end{tikzcd}
    \end{center}
    which shows that $(\epsilon G)_c \circ (G\eta)_c = \id_{G(c)}$, hence $\epsilon G \circ G\eta = \id_G$. The commutativity of the other triangle is treated in a similar way. \\
    Now assume that there are natural transformations $\eta$ and $\epsilon$ that make both triangles commute. \\
    We define two maps
    \begin{align*}
        \alpha_{c,d} & : \applic{\Hom(G(c),d)}{\Hom(c,D(d))}{f}{D(f)\circ \eta_c} \\
        \beta_{c,d} & : \applic{\Hom(c,D(d))}{\Hom(G(c),d)}{g}{\epsilon_d \circ G(g)}
    \end{align*}
    and we show these are natural isomorphisms that give us the adjunction. First we check naturality of $\alpha$. Let $f : c \to c' \in \Mor(\cat)$ and $g : d\to d' \in \Mor(\catt)$. We need to check that the diagrams
    \begin{center}
        \begin{tikzcd}
            \Hom_\catt (G(c),d)
                \arrow[r, "\alpha_{c,d}"]
            &
            \Hom_\cat (c,D(d))
            \\
            \Hom_\catt (G(c'),d)
                \arrow[u, "- \circ G(f)"]
                \arrow[r,"\alpha_{c',d}"]
            &
            \Hom_\cat(c',D(d))
                \arrow[u,"-\circ f"]
        \end{tikzcd}
        \hspace{1cm}
        \begin{tikzcd}
            \Hom_\catt (G(c),d)
                \arrow[r, "\alpha_{c,d}"]
                \arrow[d, "g \circ -"]
            &
            \Hom_\cat (c,D(d))
                \arrow[d, "D(g) \circ -"]
            \\
            \Hom_\catt (G(c),d')
                \arrow[r,"\alpha_{c,d'}"]
            &
            \Hom_\cat(c,D(d'))
        \end{tikzcd}
    \end{center}
    commute. We only check the left diagram and leave the right to the reader (sorry). We have
    \begin{align*}
        \alpha_{c,d} \circ (- \circ G(f)) & = (D(-) \circ \eta_c) \circ (- \circ G(f)) = D(-\circ G(f)) \circ \eta_c = D(-) \circ DG(f) \circ \eta_c \\
        (-\circ f) \circ \alpha_{c',d} & = (-\circ f) \circ (D(-)\circ \eta_{c'}) = D(-) \circ \eta_{c'} \circ f = D(-) \circ DG(f) \circ \eta_c
    \end{align*}
    One shows $\beta$ is natural in $c$ and $d$ in a similar way. We leave it to the reader (sorry again). Now we need to check that $\alpha$ and $\beta$ are inverses of each other, and that's where the triangle diagrams come into play.
    \[\alpha_{c,d} \circ \beta_{c,d} = D(\epsilon_d \circ G(-)) \circ \eta_c = D(\varepsilon_d) \circ DG(-) \circ \eta_c = D(\varepsilon_d) \circ \eta_{D(d)} \circ - = -\]
    We used the definitions of $\alpha$ and $\beta$, the functoriality of $D$, the naturality of $\eta$ and the second triangle diagram. We leave to the reader (sorry) to check that $\beta_{c,d} \circ \alpha_{c,d}$ is also the identity.
\end{proof}

\begin{examples} \leavevmode
    \begin{enumerate} 
        \item The forgetful functor $\Ab \to \Set$ is right adjoint to the free abelian group functor $\Set \to \Ab$.
        \item The forgetful functor $\Ab \to \Grp$ is right adjoint to the abelianization functor $\Grp \to \Ab$ that sends a group $G$ to its abelianization $G^{ab} = G/[G,G]$ and a morphism $f : G \to H$ to the induced morphism $f^{ab} : G^{ab} \to H^{ab}$.
        \item The forgetful functor $\Top \to \Set$ is right adjoint to the functor $\Set \to \Top$ that takes a set and equips it with the coarse topology. The forgetful functor $\Top \to \Set$ is also left adjoint to the functor $\Set \to \Top$ that equips a set with the discrete topology.
        \item Let $G$ be a group, $H$ one of its subgroups and $k$ be a field. We have a functor from the category $\Rep_k(G)$ of representations of $G$ on $k$-vector spaces to the category $\Rep_k(H)$ of representations of $H$ on $k$-vector spaces. It is the restriction functor $\mathrm{Res}_H^G$. Its left adjoint is $\mathrm{Ind}_H^G$, the induced representation functor.
    \end{enumerate}
\end{examples}

\begin{theorem} \label{thmadjoints}
    Let $F : \cat \to \catt$ be a functor. The following are equivalent:
    \begin{enumerate}
        \item $F$ admits a left adjoint
        \item For all $X \in \Ob(\catt)$, $\Hom_\catt (X,F(-))$ is representable
        \item For all $X \in \Ob(\catt)$, there exists a universal arrow $X \to F$
    \end{enumerate}
\end{theorem}

\begin{corollary}
    If they exist, adjoints are unique up to isomorphism.
\end{corollary}

\begin{proof}
    $2 \iff 3$ was the subject of a previous remark right after the Yoneda lemma. We prove $1 \iff 2$. Suppose $F$ admits a left adjoint $G$. Let $X \in \Ob(\catt)$. Then for all $Y \in \Ob(\cat)$ we have a bijection $\Hom_\catt(X,F(Y)) \simeq \Hom_\cat(G(X),Y)$ which is natural in $Y$, so $G(X)$ represents $\Hom_\catt(X,F(-))$. For the converse, suppose all functors $\Hom_\catt(X,F(-))$ are representable. We define $G(X)$ to be an object of $\cat$ that represents $\Hom_\catt(X,F(-))$. Now choose $X,Y \in \Ob(\catt)$ and $f : X \to Y$. We need to define $G(f)$. We wish to have a commuting square
    \begin{center}
        \begin{tikzcd}
            \Hom(G(X),-)
                \arrow[r, "\sim"]
            &
            \Hom(X,F(-))
            \\
            \Hom(G(Y),-)
                \arrow[r,"\sim"]
                \arrow[u,"\exists! \gamma", dashed]
            &
            \Hom(Y,F(-))
                \arrow[u,"- \circ f"]
        \end{tikzcd}
    \end{center}
    We need to recover a map $G(X) \to G(Y)$ such that composing with it gives us $\gamma$. This works by the Yoneda lemma, which tells us that the natural transformation $\gamma$ comes from an element $\Hom(G(X),G(Y))$. Call it $G(f)$. It remains to check this does define a functor. Using this diagram with $X=Y$ and $f = \id_X$ shows that $G(\id_X)= \id_{G(X)}$. Let $X \xrightarrow{f} Y \xrightarrow{g} Z$ in $\cat$. Then we draw
    \begin{center}
        \begin{tikzcd}
            \Hom(G(Z),-)
                \arrow[r,"- \circ G(g)" ']
                \arrow[rr,bend left=10,"- \circ G(g\circ f)"]
                \arrow[d,"\sim" sloped]
            &
            \Hom(G(Y),-)
                \arrow[r,"- \circ G(f)" ']
                \arrow[d,"\sim" sloped]
            &
            \Hom(G(X),-)
                \arrow[d,"\sim" sloped]
            \\
            \Hom(Z,F(-))
                \arrow[r,"- \circ g"]
                \arrow[rr,bend right=10,"- \circ (g \circ f)",',]
            &
            \Hom(Y,F(-))
                \arrow[r,"- \circ f"]
            &
            \Hom(X,F(-))
        \end{tikzcd}
    \end{center}
    and this diagram shows that $G(g\circ f) = G(g) \circ G(f)$ (because the map $\gamma$ above is unique).
\end{proof}

This theorem shows there is a deep link between universal properties and adjoint functors.

\subsection{Limits and colimits}
(This subsection may be skipped on a first reading.) \\
Let us recall the definition of a functor category.
\begin{definition}
    Let $\cat,\catt$ be two categories. Then $\Fun(\cat,\catt)$, also written $\catt^\cat$, is the category whose objects are functors $\cat \to \catt$ and morphisms are natural transformations between such functors, with composition given by vertical composition. It is called the \emph{functor category category from $\cat$ to $\catt$}. When $\Jcat$ is a small category we also say that $\Fun(\cat,\catt)$ is the category of \emph{diagrams of shape $\Jcat$ in $\cat$}.
\end{definition}

\begin{examples} \leavevmode
    \begin{enumerate}
        \item Let $\mathbf{2}$ be the category \begin{tikzcd}
            \bullet \arrow[r] & \bullet
        \end{tikzcd} which has two objects $1$ and $2$ and three morphisms (two of them being identities). \\
        Then $\mathbf{2\times 2}$ is the category \begin{tikzcd}
            \bullet
                \arrow[r] \arrow[d] \arrow[rd]
            &
            \bullet
                \arrow[d]
            \\
            \bullet
                \arrow[r]
            &
            \bullet
        \end{tikzcd} which has 4 objects and 9 morphisms (4 of them being identities). Then, a functor from  $\mathbf{2 \times 2}$ to $\cat$ is a commutative diagram of this shape in $\cat$.
        \item If $\Jcat$ is a small category, there is a functor $\Delta : \Ob(\cat) \to \Fun(\Jcat,\cat)$ where $\Delta(c)$ is the constant functor at $c$, that is the functor that sends all objects to $c$ and all morphisms to $\id_c$, and $\Delta(f) = f$, which works since a natural transformation $\Delta(c) \Rightarrow \Delta(d)$ is just the data of one morphism $c\to d$.
    \end{enumerate}
\end{examples}

\begin{definition}
    A \emph{cone above a diagram $F : \Jcat \to \cat$ with summit $c\in \cat$} is a natural transformation $\lambda : \Delta(c) \Rightarrow F$. Dually, a cone \emph{under} $F$ with summit $c$, also called a \emph{cocone}, is a natural transformation $\lambda : F \Rightarrow \Delta(c)$.
\end{definition}

Let us unwrap this definition. A cone is a collection of maps $\lambda_j : c \to F(j)$ for all $j \in \Ob(\Jcat)$, such that for any morphism $f : i \to j \in \Mor(\Jcat)$, this diagram commutes:
\begin{center}
    \begin{tikzcd}[column sep=tiny]
    &
    c
        \arrow[dl,"\lambda_i" '] \arrow[dr,"\lambda_j"]
    \\
    F(i)
        \arrow[rr, "F(f)" ']
    & &
    F(j)
    \end{tikzcd}
\end{center}

\begin{definition}
    Let $F : \Jcat \to \cat$ be a diagram. A \emph{limit} (or \emph{projective limit} or \emph{inverse limit}) of $F$ is a universal cone above $F$, in the sense that it is a final object in the category of cones above $F$. Dually, a \emph{colimit} (or \emph{inductive limite} or \emph{direct limit}) is a universal cocone, that is an initial object in the category of cones under $F$.
\end{definition}

Concretely, a limit of $F : \Jcat \to \cat$ is a pair $(\lim F, \phi)$ with $\lim F \in \Ob(\cat)$ and $\phi : \Delta(\lim F) \Rightarrow F$ is such that for any cone $\lambda : \Delta(c) \Rightarrow F$, there exists a unique morphism $f : X \to \lim F \in \Mor(\cat)$, such that the diagram on the left commutes:
\begin{center}
    \begin{tikzcd}[column sep=tiny]
        \Delta(c)
            \arrow[rr, "\Delta(f)",dashed]
            \arrow[dr, "\lambda" ', Rightarrow]
        &&
        \Delta(\lim F)
            \arrow[dl, "\phi", Rightarrow]
        \\
        &
        F
    \end{tikzcd}
    \hspace{1cm}
    which is equivalent to
    \hspace{1cm}
    $\forall j \in \Jcat$,
    \begin{tikzcd}[column sep=tiny]
        c
            \arrow[rr, "f",dashed]
            \arrow[dr, "\lambda_j" ', Rightarrow]
        &&
        \lim F
            \arrow[dl, "\phi_j", Rightarrow]
        \\
        &
        F(j)
    \end{tikzcd}
\end{center}

In compact form, $\Hom_\cat(-, \lim F) \simeq \Hom_{\Fun(\Jcat,\cat)} (\Delta(-),F)$.

\begin{exercise}
    Do the same for colimits.
\end{exercise}

\begin{remark}\leavevmode
    \begin{enumerate}
        \item If a limit exists it is unique up to isomorphism (unique isomorphism that commutes with the legs of the cone)
        \item If all limits exist, then $\lim$ becomes a functor $\lim : \Fun(\Jcat, \cat) \to \cat$ in the following way. Recall that theorem \ref{thmadjoints} says a functor $D$ admits a left adjoint iff for all objects $X$ in its codomain, $\Hom(X,D(-))$ is representable. The compact form of the definition of a limit says that the functor $\Hom(\Delta(-),F)$ is representable for all $F$ (since we assume all limits exist). A dual version of the theorem gives that $\Delta$ admits a right adjoint, which is $\lim$ since $\Hom(c,\lim F) \simeq \Hom(\Delta(c),F)$. If $\eta : F \Rightarrow G$ is a natural transformation, then $\lim(\eta)$ can be constructed in the following way: $\lim F \Rightarrow F \underset{\eta}{\Rightarrow} G$ is a cone above $G$, and $\lim(\eta) : \lim F \to \lim G$ comes from the universality of $\lim G$.
    \end{enumerate}
\end{remark}

\begin{corollary} \leavevmode
    \begin{enumerate}
        \item If $\cat$ has all $\Jcat$-limits, then $\lim : \Fun(\Jcat,\cat) \to \cat$ is a right adjoint to $\Delta$.
        \item If $\cat$ has all $\Jcat$-colimits, then $\colim : \Fun(\Jcat,\cat) \to \cat$ is a left adjoint to $\Delta$.
    \end{enumerate}
\end{corollary}

\begin{example} \leavevmode
    \begin{enumerate}
        \item If $\Jcat$ is discrete, that is has no morphisms other than identities, then a functor $F : \Jcat \to \cat$ is the same as a collection $(X_i)_{i \in \Jcat}$ of objects of $\cat$. Then, a limit of $F$ is an object $\lim F \in \Ob(\cat)$ with morphisms $f_i : \lim F \to X_i$ such that for all objects $X \in \Ob(\cat)$ with morphisms $p_i : X \to X_i$, we have a unique map $\alpha : X \to \lim F$ that makes this diagram commute for all $i,j \in \Jcat$:
        \begin{center}
            \begin{tikzcd}
                &
                X
                    \arrow[dl,"p_i" ']
                    \arrow[d,"\alpha",dashed]
                    \arrow[dr,"p_j"]
                \\
                X_i
                &
                \lim F
                    \arrow[l,"f_i"]
                    \arrow[r,"f_j" ']
                &
                X_j
            \end{tikzcd}
        \end{center}
        We write $\lim F = \prod\limits_{j \in \Jcat} F(j)$ and call it the \emph{product of the $F(j)$s}. Morphisms $f_i$ are written $\pi_i$ and called \emph{canonical projections}. \\
        Dually, the colimit of $F$ is called a coproduct and written $\bigsqcup\limits_{j \in \Jcat} F(j)$.
        \item If $\Jcat =$ \begin{tikzcd}[column sep=small]
            \bullet \arrow[r,shift left] \arrow[r, shift right] & \bullet
        \end{tikzcd}, then a functor $F : \Jcat \to \cat$ is the data of two parallel morphisms in $\cat$. A limit is an equalizer and a colimit is a coequalizer.
        \item If $\Jcat =$ \begin{tikzcd}[column sep=small,row sep=scriptsize]
            & \bullet \arrow[d] \\ \bullet \arrow[r] & \bullet
        \end{tikzcd} then $F : \Jcat \to \cat$ is the data of $A,B,C \in \Ob(\cat)$ with two morphisms $f : A \to C$ and $g : B \to C$. The limit $\lim F$ is called a \emph{pullback} of $f$ and $g$, with universal property depicted here:
        \begin{center}
        \begin{tikzcd}
            X
                \arrow[rrd,bend left]
                \arrow[ddr,bend right]
                \arrow[dr,dashed,"\exists !"]
            \\
            &
            A \times_C B
                \arrow[d,"\pi_B" ']
                \arrow[r,"\pi_A"]
            &
            A
                \arrow[d,"f"]
            \\
            &
            B
                \arrow[r,"g" ']
            &
            C
        \end{tikzcd}
    \end{center}
    \item If $\Jcat = \omega^\op$, that is $\Jcat = \cdots \to 2 \to 1 \to 0$, then $\lim F$ is often called the ``inverse limit'' of $F$. Concretely, $F$ is the data of $\cdots \to F(2) \xrightarrow[\alpha_2]{} F(1) \xrightarrow[\alpha_1]{} F(0)$, and a cone above $F$ looks like
    \begin{center}
        \begin{tikzcd}
            & & c
                \arrow[dl,"\lambda_2" ']
                \arrow[d,"\lambda_1"]
                \arrow[dr,"\lambda_0"]
            \\
            \cdots
                \arrow[r]
            &
            F(2)
                \arrow[r,"\alpha_2" ']
            &
            F(1)
                \arrow[r,"\alpha_1" ']
            &
            F(0)
        \end{tikzcd}
        \hspace{2em}
        we have $(\alpha_i \circ \cdots \circ \alpha_n) \circ \lambda_n = \lambda_i$.
    \end{center}
    The typical example of an inverse limit is the one given by $F(n) = \nicefrac{\Z}{p^n\Z}$ in $\Ring$ with morphisms $\nicefrac{\Z}{p^{n+1}\Z} \to \nicefrac{\Z}{p^n\Z}$ being reduction mod $p^n$. The inverse limit $\invlim \nicefrac{\Z}{p^n\Z}$ is the ring of $p$-adic integers. Concretely, $a \in \invlim \nicefrac{\Z}{p^n\Z}$ iff $a = (a_i)_{i\in \N}$ such that $a_i \equiv a_j \mod p^i \forall i \le j$.
    \item The dual notion, given by $\Jcat = 0 \to 1 \to 2 \to \cdots$, is obtained by taking the colimit. It is called a \emph{direct limit}. The typical example here is the Prüfer $p$-group $\dirlim \nicefrac{\Z}{p^n \Z} = \Z(p^\infty)$.
    \end{enumerate}
\end{example}

\end{document}