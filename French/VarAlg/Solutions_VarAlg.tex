\documentclass{article}
\usepackage{amsmath, amssymb, amsfonts, amsthm}
\usepackage[french]{babel}
\usepackage[T1]{fontenc}
\usepackage{geometry}[textwidth=10cm]
\usepackage{tikz-cd}
\usepackage{tikz}
\usepackage{nicefrac}
\usepackage{cancel}
\usepackage{url}

\newcommand{\id}{\mathrm{id}}
\newcommand{\op}{\mathrm{op}}
\newcommand{\N}{\mathbb{N}}
\newcommand{\Z}{\mathbb{Z}}
\newcommand{\Q}{\mathbb{Q}}
\newcommand{\R}{\mathbb{R}}
\newcommand{\C}{\mathbb{C}}
\newcommand{\K}{\mathbb{K}}
\newcommand{\U}{\mathbb{U}}
\renewcommand{\H}{\mathbb{H}}

\renewcommand{\P}{\mathbb{P}}
\renewcommand{\d}{{\rm \, d}}

\newcommand{\M}{\mathcal{M}}
\newcommand{\cont}{\mathcal{C}}

\newcommand{\Set}{\mathbf{Set}}
\newcommand{\Top}{\mathbf{Top}}
\newcommand{\Grp}{\mathbf{Grp}}
\newcommand{\Mod}{\mathbf{Mod}}
\newcommand{\Ab}{\mathbf{Ab}}
\newcommand{\Vectcat}{\mathbf{Vect}}
\newcommand{\Ring}{\mathbf{Ring}}

%Operators
\DeclareMathOperator{\Acc}{Acc}
\DeclareMathOperator{\Sh}{Sh}
\DeclareMathOperator{\Card}{Card}
\DeclareMathOperator{\GL}{GL}
\DeclareMathOperator{\SL}{SL}
\DeclareMathOperator{\PSL}{PSL}
\DeclareMathOperator{\PGL}{PGL}
\DeclareMathOperator{\SO}{SO}
\DeclareMathOperator{\pgcd}{pgcd}
\DeclareMathOperator{\Tr}{Tr}
\DeclareMathOperator{\rg}{rg}
\DeclareMathOperator{\Vect}{Vect}
\DeclareMathOperator{\Gal}{Gal}
\DeclareMathOperator{\Fix}{Fix}
\DeclareMathOperator{\Ob}{Ob}
\DeclareMathOperator{\Aut}{Aut}
\DeclareMathOperator{\Mor}{Mor}
\DeclareMathOperator{\Hom}{Hom}
\DeclareMathOperator{\Fun}{Fun}
\DeclareMathOperator{\Nat}{Nat}
\DeclareMathOperator{\colim}{colim}
\DeclareMathOperator{\End}{End}
\DeclareMathOperator{\coker}{coker}
\DeclareMathOperator{\im}{im}
\DeclareMathOperator{\coim}{coim}
\renewcommand{\Re}{\mathop{\rm Re}}
\renewcommand{\Im}{\mathop{\rm Im}}

\newcommand{\todo}{\textbf{TODO}}

\renewcommand{\epsilon}{\varepsilon}

%Macros
\newcommand{\applic}[4]{\begin{array}[t]{rcl}
#1 & \to & #2 \\
#3 & \mapsto & #4
\end{array}}

\newcommand{\norm}[1]{\left\lVert #1 \right\rVert}

\setlength{\parindent}{0pt}

\title{TD de variétés algébriques}
\author{Exercices par Jean-François Dat\footnote{\protect\url{https://webusers.imj-prg.fr/~jean-francois.dat/enseignement/enseignement.php}}}
\date{}

\begin{document}

\maketitle

\section{TD1}

\paragraph{1.} Pour $x \in S$ et $f \in I_S$, on a $f(x) = 0$, donc $S \subset V_{I_S}$. Par définition de la topologie de Zariski, $V_{I_S}$ est fermé donc $\overline{S} \subset V_{I_S}$. Maintenant, soit $y \in k^n$ et $U$ un ouvert de Zariski contenant $y$ qui ne rencontre pas $S$. On peut prendre $U$ principal égal à $U_f$. Alors, $U_f \cap S = \varnothing$, donc $S \subset V_{\{f\}}$, donc $f(x) = 0$ pour tout $x \in S$, donc $f \in I_S$. Comme $y \in U_f$, $f(y) \ne 0$, donc $y \notin V_{I_S}$. Par contraposée, cela conclut.

\paragraph{2.} Soient $I \lhd k[X_1,\dots,X_n]$ et $J \lhd k[X_{n+1},\dots,X_m]$ tels que $V = V_I$ et $W = V_J$. Alors,
\begin{align*}
    (x_1,\dots,x_{n+m}) \in V \times W & \iff (x_1,\dots,x_n) \in V \text{ et } (x_{n+1},\dots,x_{n+m}) \in W \\
    & \iff \forall f \in I, f(x_1,\dots,x_n) = 0 \text{ et } \forall g \in J, g(x_{n+1}, \dots, x_{n+m}) = 0
\end{align*}
En posant $f_1(x_1,\dots,x_{n+m}) = f(x_1,\dots,x_n)$ et $g_2(x_1,\dots,x_{n+m}) = g(x_{n+1},\dots,x_{n+m})$, on voit que
\[(x_1,\dots,x_{n+m}) \in V \times W \iff \forall f \in I, f_1(x_1,\dots,x_{n+m}) = 0 \text{ et } \forall g\in J, g_2(x_1,\dots,x_{n+m}) = 0\]
donc $V \times W$ est algébrique comme intersection d'ensembles qui le sont.

\paragraph{3.} On fait la preuve dans le cas $n=m=1$ pour alléger. Alors, l'ensemble $V_{X-Y}^c = \{(x,y) \in k^2 \mid x \ne y\}$ est un ouvert de Zariski de $k^2$. Soit maintenant $O_1, O_2 \subset k$ deux ouverts non vides de Zariski. Comme on est sur $k$, $O_1$ et $O_2$ sont en fait des parties de complémentaire fini. Comme $k$ est supposé infini, $O_1 \cap O_2 \ne \varnothing$. Mais alors, on dispose d'un $x \in k$ tel que $(x,x) \in O_1 \cap O_2$, ce qui implique que $V_{X-Y}^c$ ne peut pas contenir de produit d'ouverts non vides de $k$ donc nous donne le résultat. Dans le cas général, on peut faire une preuve similaire en considérant l'ouvert complémentaire des solutions de $X_1 = X_2 = \cdots = X_{n+m}$.

\paragraph{4.} Si $f_i(x) = 0$, alors $f(x) = 0$, donc $\bigcup_i V_i \subset V_f$. Réciproquement, si $f(x)=0$, alors par intégrité il existe $i$ tel que $f_i^{n_i}(x) = 0$, donc en fait $f_i(x)=0$, donc $V = \bigcup_i V_i$. \todo

\paragraph{5.} Une union de composantes connexes est ouverte, donc si $V$ avait une infinité de composantes connexes $(C_n)_{n\in \N}$ deux à deux distinctes, alors $\bigcup_{k=0}^n C_k$ serait une suite croissante d'ouverts qui ne stationne pas ce qui est impossible car $V$ muni de la topologie de Zariski est noethérien. 

\end{document}