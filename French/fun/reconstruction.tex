\documentclass{article}
\usepackage{amsmath,amssymb,amsthm,amsfonts,stmaryrd}
\usepackage{tikz}
\usepackage[french]{babel}
\usepackage[T1]{fontenc}
\usepackage{geometry}[textwidth=10cm]
\usepackage{tikz-cd}
\usepackage{nicefrac}

\newcommand{\id}{\mathrm{id}}
\newcommand{\N}{\mathbb{N}}
\newcommand{\Z}{\mathbb{Z}}
\newcommand{\Q}{\mathbb{Q}}
\newcommand{\R}{\mathbb{R}}
\newcommand{\C}{\mathbb{C}}

\renewcommand{\P}{\mathcal{P}}
\renewcommand{\d}{{\rm \, d}}

\newcommand{\M}{\mathcal{M}}

%Operators
\DeclareMathOperator{\Card}{Card}
\DeclareMathOperator{\GL}{GL}
\DeclareMathOperator{\SL}{SL}
\DeclareMathOperator{\PSL}{PSL}
\DeclareMathOperator{\PGL}{PGL}
\DeclareMathOperator{\SO}{SO}
\DeclareMathOperator{\pgcd}{pgcd}
\DeclareMathOperator{\Tr}{Tr}
\DeclareMathOperator{\Vect}{Vect}
\DeclareMathOperator{\Gal}{Gal}
\DeclareMathOperator{\Fix}{Fix}
\DeclareMathOperator{\Aut}{Aut}
\DeclareMathOperator{\Int}{Int}
\DeclareMathOperator{\Hom}{Hom}
\renewcommand{\Re}{\mathop{\rm Re}}
\renewcommand{\Im}{\mathop{\rm Im}}

\newcommand{\todo}{\textbf{TODO}}

\renewcommand{\epsilon}{\varepsilon}

%Macros
\newcommand{\applic}[4]{\begin{array}[t]{rcl}
#1 & \to & #2 \\
#3 & \mapsto & #4
\end{array}}

\setlength{\parindent}{0pt}

\theoremstyle{plain}
\newtheorem{theorem}{Théorème}
\newtheorem{proposition}[theorem]{Proposition}
\newtheorem*{corollary}{Corollaire}

\theoremstyle{definition}
\newtheorem{definition}[theorem]{Définition}
\newtheorem{example}[theorem]{Exemple}
\newtheorem{question}{Question}

\theoremstyle{remark}
\newtheorem*{remark}{Remarque}

\title{Conjecture de reconstruction}
\author{Reblax}
\date{}

\begin{document}

\maketitle

Soit $\Gamma = (V,E)$ un graphe et $G = \Aut(\Gamma)$. Alors, $G$ agit sur le multi-ensemble $D(G)$ en envoyant la carte correspondant à $v \in V$ sur celle correspondant à $\sigma(v) \in V$. Si $v,v'$ sont dans la même orbite pour l'action naturelle $G \curvearrowright V$, alors un automorphisme $g$ tel que $g\cdot v = v'$ induit un isomorphisme de graphes $\Gamma_v \xrightarrow{\sim} \Gamma_{v'}$. Comme le montre l'exemple de [dessiner], on n'obtient pas forcément tous les isomorphismes $\Gamma_v \to \Gamma_{v'}$ de cette manière. \todo fixer une identification ``canonique'' en identifiant les deux sommets enlevés et ainsi obtenir peut-être un sous-groupe de $\Aut(\Gamma_v)$

\end{document}
